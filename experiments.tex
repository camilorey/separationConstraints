% Hypothesis
% \begin{itemize}
%     \item Separability is important
%     \begin{itemize}
%         \item Separation sets
%         \item All positive network
%     \end{itemize}
%     \item Controlling separability enhances the network
%     \begin{itemize}
%         \item Depth
%         \item Dead units
%         \item Zero init
%         \item Moons data shattering
%     \end{itemize}
%     \item Separability works in real world
%     \begin{itemize}
%         \item Toy Cifar
%         \item SimpNet
%         \item ULMfit
%         \item Fixup
%     \end{itemize}
% \end{itemize}

\section{Experiments and Results}\label{sec:experiments}
After presenting the separability constraints, we test our formulation in an example of a network according to our formulation (recall the introductory paragraph of section \ref{sec:separability}). Our network has a fixed depth $D=50$ and a fixed layer width of 4 units (i.e. $m_k=4$ for $k=1,\ldots,D$. To which we refuse to perform any architectural modifications. 
\\\\
Our intention is to benchmark our proposal to the most commonly used data manipulation technique: \emph{Batch Normalization} (BN for short). As a consequence, our results entail always a comparison between: (a) no data manipulation, (b) data manipulation via BN and (c) our technique. The only addition introduced during the training is \emph{Adam} \cite{adam}, when calculating gradients \cite{Goodfellow-et-al-2016}.  
\\\\
As a \emph{main} loss we chose to do a comparison between binary cross-entropy (a classic) in \cite{LeCun06atutorial}, and compare it to the case of \emph{no loss} ($\mathcal{L}\equiv 0$ in the introduction of section \ref{sec:separability}). We chose the no loss to explore the nature of the introduction of the separation constraints \emph{by themselves}.     
\\\\
As an initialization for all our experiments we chose two schemes: the Glorot scheme \cite{Glorot10Initialization} (by default on \texttt{Keras}) and a \emph{zero} initialization scheme (setting all parameters to zero), to verify our geometric intuition regarding the distribution of the upper and lower sets of units.   
\\\\
As a dataset, we choose a small version of the \moons dataset, sampling $100$ points ($85$ for training and $15$). Our experiments were done using \texttt{Keras}\cite{keras} and \texttt{TensorFlow}\cite{tensorflow}. We chose the hyperparameters $D$ and $m_k$ to showcase the functioning of our constraints formulation in a particularly \emph{deep} network (as a ratio of depth and layer-width). 
\\\\
We chose the \moons for its \emph{simplicity} in terms of point distribution (two classes parametrically created, see for example Figure \ref{fig:zerosInput3000} on top). Also, the separating functions $F$ created can be easily visualized in a surface map as showcased in Figure \ref{fig:zerosInput3000} (recall equation \ref{eq:network}). In this sense, it is easy to tell whether a classification is \emph{well done} (graphically) and we can relate it to accuracy values. The \moons dataset promotes intuition beyond scores or metrics.  

% Beginning of the experimental section.
\subsection{Effect of the separation on the internal representations}\label{subsec:internalRepresentation}

In terms of accuracy \ReLU reaches a trivial accuracy of $0.51$ while \ReLUBN reaches $0.60$, both architectures fail to solve the problem. When comparing the results presented in Figures \ref{fig:moonsReLU} and \ref{fig:moonsReLUBN}, we find that at the fourth layer \ReLU has collapsed the points of the dataset over a line (parts \ref{fig:moonsReLU41} and \ref{fig:moonsReLU42}), while \ReLUBN still manages to warp the dataset. However, both methods at layer 25th have \emph{collapsed} the dataset into a small number of points (see Figures \ref{fig:moonsReLUBN251}, \ref{fig:moonsReLUBN252} for \ReLUBN and Figures \ref{fig:moonsReLU251} and \ref{fig:moonsReLU252} for \ReLU) that is then \emph{pushed} to the point zero for \ReLU (as shown in the feature layer: \ref{fig:moonsReLUFeature1} and \ref{fig:moonsReLUFeature2}), while \ReLUBN collapses to few points (Figures \ref{fig:moonsReLUBNFeature1} and \ref{fig:moonsReLUBNFeature2}). 
\\\\
We suggest that this collapsing is due the fact that entire portions of the data set are being mapped to the intersection of the lower sets of all the units in a layer, incrementally. Moreover, the likelihood of this collapsing (and its effect on the whole dataset) increases with depth and decreases with width (i.e. bruteforcing non trivial activations via \emph{width-increase}). 
\\\\
In addition, since the gradient is only back-propagated through the points lying on the upper sets of units, geometric collapse also stops learning. Notice that the standarization and the affine transformation (dependent on $\gamma$ and $\beta$) from \ReLUBN are not able to prevent sending the dataset to the lowers of the units. Thus, training thin and deep neural networks is difficult with \ReLU or \ReLUBN.
\\\\
On the extent of our experimentation  \ReLU networks featuring separation constraints are able to solve the problem with higher degrees of accuracy than \ReLU and \ReLUBN (see the figures in Table \ref{tab:moons}). Moreover, Figures \ref{fig:moonsUnitwiseOutput}, \ref{fig:moonsLayerwiseOutput}) \ref{fig:moonsUnitwiseOutput} and \ref{fig:moonsPointwiseOutput}), show that the separation function behaves \emph{intuitively} in the sense of Hauser \& Asok \cite{hauserAsok}. 
\\\\ 
The internal representations not only are non-trivial (as in \ReLUBN) but also preserve geometrical structures like shape and connectivity, as shown in Figures \ref{fig:moonsLayerwise251}, \ref{fig:moonsUnitwise251}, \ref{fig:moonsPointwise251} or \ref{fig:moonsUnitpointwise252}. Indeed, Figures \ref{fig:moonsUnitwise42}, \ref{fig:moonsUnitpointwise42}, \ref{fig:moonsLayerwise42} and \ref{fig:moonsPointwise42} showcase how at the 4th layers a much a solution is already found. This proves that the gradient of the main loss is back-propagated to the input, unlike \ReLU and \ReLUBN (Figures \ref{fig:moonsReLU41}, \ref{fig:moonsReLU41}, \ref{fig:moonsReLUBN41}, \ref{fig:moonsReLUBN42}).
\\\\
The constraints enforce richer representations when used without main loss (cross-entropy), see Figure \ref{fig:init}. In the case of \SepLayer the dataset is mapped into a line (Figures \ref{fig:layerwiseInit501} and \ref{fig:layerwiseInit502}) , whereas the representation reached by \SepUnitPoint is more complex, yet preserves connectivity. Furthermore, both avoid mapping the entire dataset to few values as observed with \ReLU (Figures \ref{fig:reluBNInit501} and \ref{fig:reluBNInit502}) and \ReLUBN (Figures \ref{fig:reluInit501} and \ref{fig:reluInit502}). 
\\\\
% Differences among constraints
The only exception is \SepUnit that apparently solves the problem at layer 25th on Figures \ref{fig:moonsUnitwise251} and \ref{fig:moonsUnitwise252}, yet collapses the dataset in two points at the feature layer (Figures \ref{fig:moonsUnitwiseFeature1} and \ref{fig:moonsUnitwiseFeature1}). Our intuition is  is that, although \SepUnit prevents dead/affine units, it cannot prevent points from falling into the intersection lower set of the units of a layer, construeing a \emph{dead point}). The gradient at dead points will be zero, and can only be recovered by another non-dead point moving a unit so that the dead point, \emph{revives}, when being brought back into the upper set of any unit. The \SepPoint separation constraint was designed precisely to prevent the presence of dead points. However, \SepPoint allows affine and dead units (as shown in Figures \ref{fig:moonsPointwise251} and \ref{fig:moonsPointwise252}), so the the decision function reached becomes \emph{simple} (see Figure \ref{fig:moonsPointwiseOutput}). 
\\\\
Therefore, if we combine both \SepUnit and \SepPoint we should have best of both worlds. Figure \ref{fig:moonsUnitPointwise}, shows how \SepUnitPoint is able to separate both classes perfectly (recall the $0.93$ in accuracy from Table \ref{tab:moons}). Furthermore \SepUnitPoint, produces a feature layer that does not concentrate the dataset in two points like \SepUnit, or concentrates the dataset linearly like \SepPoint. In this sense, we argue that  \SepLayer separates classes in the \texttt{MOONS} quite well, although it is a relaxation of \SepUnitPoint. We think this is due to the fact it allows affine units, that are particularly good at forwarding internal representations, as we will detail in the next subsection.

\subsection{The Role of affine, dead and repeated units}\label{sec:roleofdead}

Despite the fact that we intended to avoid dead and affine neurons with our geometrical formulation (recall the predicate $R_1(u) \wedge R_2(u)$ defined on Equation \ref{eq:separabilityDefinition}) the different constraints differ in practice. While $R_1(u) \wedge R_2(u)$ is valid within \SepUnit and extends to layers, \SepLayer and \SepPoint  allow at least one unit to be affine (recall Equation \ref{eq:affineUnit}) while allowing at least another to be dead (recall Equation \ref{eq:deadNeuronVersion2}), invalidating $R_1(u) \wedge R_2(u)$, but holding $R_1(\layer) \wedge R_2(\layer)$. 
\\\\
In this sense, neither \emph{repeated} (producing the same upper set), affine nor dead units are \emph{harmful} for network performance, beyond burdening networks with additional non used or redundant parameters. They seem ubiquitous in the extent of our experimentation as Figures \ref{fig:moonsLayerwise}, \ref{fig:moonsPointwise} and \ref{fig:moonsUnitPointwise} testify (see the axis-aligned or diagonally expanding intermediate representations of the dataset). Further inquiry is needed to understand how dead, affine and repeated neurons coalesce in solving the problem. In addition, our experimentation suggests that the  the distribution of dead, affine and repeated units varies according to constraint type (in the extent of our experimentation). 
\\\\
Indeed, Figure \ref{fig:moonsLayerwise} for the \SepLayer constraint type showcases a feature layer (divisions \ref{fig:moonsLayerwiseFeature1} and \ref{fig:moonsLayerwiseFeature1}) with two dead units and two repeated. Meanwhile Figure \ref{fig:moonsUnitwise} at the 24th layer (divisions \ref{fig:moonsUnitwise251} and \ref{fig:moonsUnitwise252}), depicts repeated neurons (organized pairwise) with no affine nor dead units. 

\subsection{Relation between separation constraints and main loss}\label{subsec:sepvsxent}

The use of the separation constraints enables back-propagation, and thus cross-entropy minimization. Particularly, we wish to turn our attention to Figure \ref{fig:peaks}. According to it, the training phase begins minimizing the constraint loss until a certain level is reached (approximately $0.21$ circa epoch 750), allowing the main loss to be optimized up to 80 to 100\% accuracy (recall the results in Table \ref{tab:moons}). However, using an additional loss (i.e. from the separation constraints) induces more \emph{aggresive} transient states (e.g. instabilities) during training (approximately around epochs 400, 1500 and 1750). 
\\\\
We conjecture that these instabilities can be associated to a progressive drop on the separation loss (as shown in Figure \ref{fig:peaks}) that is then followed by a peak on the cross-entropy. We argue that this is due a favorable network configuration (i.e. hyperplane positioning) that is achieved gradually by the separation constraints, in such a way that back-propagation of the gradient from the cross-entropy works favorably for the classification. In the limited scope of our experimentation, each of these spikes is followed by an increase on the accuracy. However, anecdotal evidence suggests that if $\lambda$ is too high, it can hamper the classification training in favor of hyperplane configuration, \emph{overriding} the main cross-entropy.

\begin{figure*}
  \centering
  \parbox{\textwidth}{
    \parbox{.195\textwidth}{%
      \subcaptionbox{Input layer\label{fig:moonsReLUInput}}{\includegraphics[width=\hsize]{img/toy/relu/conv2d_1-0.pdf}}
    }
    % \hskip1em
    \parbox{.195\textwidth}{%
      \subcaptionbox{4th layer\label{fig:moonsReLU41}}{\includegraphics[width=\hsize]{img/toy/relu/conv2d_4-0.pdf}}
    %   \vskip1em
      \subcaptionbox{4th layer\label{fig:moonsReLU42}}{\includegraphics[width=\hsize]{img/toy/relu/conv2d_4-2.pdf}}
    }
    % \hskip1em
    \parbox{.195\textwidth}{%
      \subcaptionbox{25th layer\label{fig:moonsReLU251}}{\includegraphics[width=\hsize]{img/toy/relu/conv2d_25-0.pdf}}
    %   \vskip1em
      \subcaptionbox{25th layer\label{fig:moonsReLU252}}{\includegraphics[width=\hsize]{img/toy/relu/conv2d_25-2.pdf}}
    }
    % \hskip1em
    \parbox{.195\textwidth}{%
      \subcaptionbox{Feature layer\label{fig:moonsReLUFeature1}}{\includegraphics[width=\hsize]{img/toy/relu/dense_1-0.pdf}}
    %   \vskip1em
      \subcaptionbox{Feature layer\label{fig:moonsReLUFeature2}}{\includegraphics[width=\hsize]{img/toy/relu/dense_1-2.pdf}}
    }
    % \hskip1em
    \parbox{.195\textwidth}{%
      \subcaptionbox{Output\label{fig:moonsReLUOutput}}{\includegraphics[clip, trim=2.35cm 1.75cm 4.5cm 0cm,width=\hsize]{img/toy/relu/output.pdf}}
    }
  }
  \caption{Data transformed across a 50x4 \ReLU classification network. Notice how the the dataset is progressively mapped to zero as it traverses the network. This renders the output layer unable to solve the problem.}
    \label{fig:moonsReLU}
\end{figure*}

\begin{figure*}
  \centering
  \parbox{\textwidth}{
    \parbox{.195\textwidth}{%
      \subcaptionbox{Input layer\label{fig:moonsReLUBNInput}}{\includegraphics[width=\hsize]{img/toy/relu-bn/conv2d_1-0.pdf}}
    }
    % \hskip1em
    \parbox{.195\textwidth}{%
      \subcaptionbox{4th layer\label{fig:moonsReLUBN41}}{\includegraphics[width=\hsize]{img/toy/relu-bn/conv2d_4-0.pdf}}
    %   \vskip1em
      \subcaptionbox{4th layer\label{fig:moonsReLUBN42}}{\includegraphics[width=\hsize]{img/toy/relu-bn/conv2d_4-2.pdf}}
    }
    % \hskip1em
    \parbox{.195\textwidth}{%
      \subcaptionbox{25th layer\label{fig:moonsReLUBN251}}{\includegraphics[width=\hsize]{img/toy/relu-bn/conv2d_25-0.pdf}}
    %   \vskip1em
      \subcaptionbox{25th layer\label{fig:moonsReLUBN252}}{\includegraphics[width=\hsize]{img/toy/relu-bn/conv2d_25-2.pdf}} 
    }
    % \hskip1em
    \parbox{.195\textwidth}{%
      \subcaptionbox{Feature layer\label{fig:moonsReLUBNFeature1}}{\includegraphics[width=\hsize]{img/toy/relu-bn/dense_1-0.pdf}}
    %   \vskip1em
      \subcaptionbox{Feature layer\label{fig:moonsReLUBNFeature2}}{\includegraphics[width=\hsize]{img/toy/relu-bn/dense_1-2.pdf}} 
    }
    % \hskip1em
    \parbox{.195\textwidth}{%
      \subcaptionbox{Output\label{fig:moonsReLUBNOutput}}{\includegraphics[clip, trim=2.35cm 1.75cm 4.5cm 0cm,width=\hsize]{img/toy/relu-bn/output.pdf}}
    }
  }
  \caption{Data transformed across a 50x4 \ReLUBN network. The dataset is collapsed in few points at the feature layer. As the gradient cannot be backpropagated across the truncation after the affine transform of $\gamma$ and $\beta$ despite the standarization, failing in the same manner than \ReLU only that with non-zero activations. This results in \emph{topological mixing} of the datasets. Therefore, the representational capability of the network is hindered to such extent that the resulting output, although non-trivial, is totally arbitrary.}
    \label{fig:moonsReLUBN}
\end{figure*}


\begin{figure*}
  \centering
     % Unitwise
  \parbox{\textwidth}{
    \parbox{.195\textwidth}{%
      \subcaptionbox{Input layer\label{fig:moonsUnitwiseInput}}{\includegraphics[width=\hsize]{img/toy/unitwise/conv2d_1-0.pdf}}
    }
    % \hskip1em
    \parbox{.195\textwidth}{%
      \subcaptionbox{4th layer\label{fig:moonsUnitwise41}}{\includegraphics[width=\hsize]{img/toy/unitwise/conv2d_4-0.pdf}}
    %   \vskip1em
      \subcaptionbox{4th layer\label{fig:moonsUnitwise42}}{\includegraphics[width=\hsize]{img/toy/unitwise/conv2d_4-2.pdf}}
    }
    % \hskip1em
    \parbox{.195\textwidth}{%
      \subcaptionbox{25th layer\label{fig:moonsUnitwise251}}{\includegraphics[width=\hsize]{img/toy/unitwise/conv2d_25-0.pdf}}
    %   \vskip1em
      \subcaptionbox{25th layer\label{fig:moonsUnitwise252}}{\includegraphics[width=\hsize]{img/toy/unitwise/conv2d_25-2.pdf}}
    }
    % \hskip1em
    \parbox{.195\textwidth}{%
      \subcaptionbox{Feature layer\label{fig:moonsUnitwiseFeature1}}{\includegraphics[width=\hsize]{img/toy/unitwise/dense_1-0.pdf}}
    %   \vskip1em
      \subcaptionbox{Feature layer\label{fig:moonsUnitwiseFeature2}}{\includegraphics[width=\hsize]{img/toy/unitwise/dense_1-2.pdf}}
    }
    % \hskip1em
    \parbox{.195\textwidth}{%
      \subcaptionbox{Output\label{fig:moonsUnitwiseOutput}}{\includegraphics[clip, trim=2.35cm 1.75cm 4.5cm 0cm,width=\hsize]{img/toy/unitwise/output.pdf}}
    }
  }
  \caption{Data transformed across a 50x4 \SepUnit network. Notice how dead and affine units have been reduced. Despite collapsing the dataset in two points at the feature layer, the classification performed in the output layer is approximately correct. We conjecture that this is due the dead point addressed with \SepPoint.}
    \label{fig:moonsUnitwise}
\end{figure*}



\begin{figure*}
  \centering
  %Pointwise
  \parbox{\textwidth}{
    \parbox{.195\textwidth}{%
      \subcaptionbox{Input layer\label{fig:moonsPointwiseInput}}{\includegraphics[width=\hsize]{img/toy/pointwise/conv2d_1-0.pdf}}
    }
    % \hskip1em
    \parbox{.195\textwidth}{%
      \subcaptionbox{4th layer\label{fig:moonsPointwise41}}{\includegraphics[width=\hsize]{img/toy/pointwise/conv2d_4-0.pdf}}
    %   \vskip1em
      \subcaptionbox{4th layer\label{fig:moonsPointwise42}}{\includegraphics[width=\hsize]{img/toy/pointwise/conv2d_4-2.pdf}} 
    }
    % \hskip1em
    \parbox{.195\textwidth}{%
      \subcaptionbox{25th layer\label{fig:moonsPointwise251}}{\includegraphics[width=\hsize]{img/toy/pointwise/conv2d_25-0.pdf}}
    %   \vskip1em
      \subcaptionbox{25th layer\label{fig:moonsPointwise252}}{\includegraphics[width=\hsize]{img/toy/pointwise/conv2d_25-2.pdf}} 
    }
    % \hskip1em
    \parbox{.195\textwidth}{%
      \subcaptionbox{Feature layer\label{fig:moonsPointwiseFeature1}}{\includegraphics[width=\hsize]{img/toy/pointwise/dense_1-0.pdf}}
    %   \vskip1em
      \subcaptionbox{Feature layer\label{fig:moonsPointwiseFeature2}}{\includegraphics[width=\hsize]{img/toy/pointwise/dense_1-2.pdf}} 
    }
    % \hskip1em
    \parbox{.195\textwidth}{%
      \subcaptionbox{Output\label{fig:moonsPointwiseOutput}}{\includegraphics[clip, trim=2.35cm 1.75cm 4.5cm 0cm,width=\hsize]{img/toy/pointwise/output.pdf}}
    }
  }
  \caption{Data transformed across a 50x4 \SepPoint network. The network displays a richer internal representation without collapsing the dataset like \SepUnit or \ReLUBN. However, plenty of dead and affine units appear since they are not penalized, causing underfitting.}
    \label{fig:moonsPointwise}
\end{figure*}


\begin{figure*}
  \centering
   \parbox{\textwidth}{
    \parbox{.195\textwidth}{%
      \subcaptionbox{Input layer\label{fig:moonsUnitpointwiseInput}}{\includegraphics[width=\hsize]{img/toy/unitpointwise/conv2d_1-0.pdf}}
    }
    % \hskip1em
    \parbox{.195\textwidth}{%
      \subcaptionbox{4th layer\label{fig:moonsUnitpointwise41}}{\includegraphics[width=\hsize]{img/toy/unitpointwise/conv2d_4-0.pdf}}
    %   \vskip1em
      \subcaptionbox{4th layer\label{fig:moonsUnitpointwise42}}{\includegraphics[width=\hsize]{img/toy/unitpointwise/conv2d_4-2.pdf}} 
    }
    % \hskip1em
    \parbox{.195\textwidth}{%
      \subcaptionbox{25th layer\label{fig:moonsUnitpointwise251}}{\includegraphics[width=\hsize]{img/toy/unitpointwise/conv2d_25-0.pdf}}
    %   \vskip1em
      \subcaptionbox{25th layer\label{fig:moonsUnitpointwise252}}{\includegraphics[width=\hsize]{img/toy/unitpointwise/conv2d_25-2.pdf}} 
    }
    % \hskip1em
    \parbox{.195\textwidth}{%
      \subcaptionbox{Feature layer\label{fig:moonsUnitpointwiseFeature1}}{\includegraphics[width=\hsize]{img/toy/unitpointwise/dense_1-0.pdf}}
    %   \vskip1em
      \subcaptionbox{Feature layer\label{fig:moonsUnitpointwiseFeature2}}{\includegraphics[width=\hsize]{img/toy/unitpointwise/dense_1-2.pdf}} 
    }
    % \hskip1em
    \parbox{.195\textwidth}{%
      \subcaptionbox{Output\label{fig:moonsUnitpointwiseOutput}}{
      \includegraphics[clip, trim=2.35cm 1.75cm 4.5cm 0cm,width=\hsize, height=\hsize]{img/toy/unitpointwise/output.pdf}
      }
    }
  }

    \caption{Data transformed across a 50x4 \SepUnitPoint network. The network displays internal representations without collapsing the dataset like \SepPoint while retaining classification power like \SepUnit.}
    \label{fig:moonsUnitPointwise}
\end{figure*}

\begin{figure*}
  \centering
  \parbox{\textwidth}{
    \parbox{.195\textwidth}{%
      \subcaptionbox{Input layer\label{fig:moonsLayerwiseInput}}{\includegraphics[width=\hsize]{img/toy/layerwise/conv2d_1-0.pdf}}
    }
    % \hskip1em
    \parbox{.195\textwidth}{%
      \subcaptionbox{4th layer\label{fig:moonsLayerwise41}}{\includegraphics[width=\hsize]{img/toy/layerwise/conv2d_4-0.pdf}}
    %   \vskip1em
      \subcaptionbox{4th layer\label{fig:moonsLayerwise42}}{\includegraphics[width=\hsize]{img/toy/layerwise/conv2d_4-2.pdf}} 
    }
    % \hskip1em
    \parbox{.195\textwidth}{%
      \subcaptionbox{25th layer\label{fig:moonsLayerwise251}}{\includegraphics[width=\hsize]{img/toy/layerwise/conv2d_25-0.pdf}}
    %   \vskip1em
      \subcaptionbox{25th layer\label{fig:moonsLayerwise252}}{\includegraphics[width=\hsize]{img/toy/layerwise/conv2d_25-2.pdf}} 
    }
    % \hskip1em
    \parbox{.195\textwidth}{%
      \subcaptionbox{Feature layer\label{fig:moonsLayerwiseFeature1}}{\includegraphics[width=\hsize]{img/toy/layerwise/dense_1-0.pdf}}
    %   \vskip1em
      \subcaptionbox{Feature layer\label{fig:moonsLayerwiseFeature2}}{\includegraphics[width=\hsize]{img/toy/layerwise/dense_1-2.pdf}} 
    }
    % \hskip1em
    \parbox{.195\textwidth}{%
      \subcaptionbox{Output\label{fig:moonsLayerwiseOutput}}{\includegraphics[clip, trim=2.35cm 1.75cm 4.5cm 0cm,width=\hsize]{img/toy/layerwise/output.pdf}}
    }
  }
    \caption{Data transformed across a 50x4 \SepLayer network. Although being a relaxation of \SepUnitPoint showing an intuitive solution with several affine and dead units. Indeed, they not only do not hinder the network but forward the solution 4th layer to the output.}
    \label{fig:moonsLayerwise}
\end{figure*}

\begin{figure*}
  \centering
    \begin{subfigure}[b]{0.3\textwidth}
        \includegraphics[width=\textwidth]{img/init/relu/conv2d_1-0.pdf}
        \caption{\ReLU input layer}
        \label{fig:reluInitInput}
    \end{subfigure}
    ~ %add desired spacing between images, e. g. ~, \quad, \qquad, \hfill etc. 
      %(or a blank line to force the subfigure onto a new line)
    \begin{subfigure}[b]{0.3\textwidth}
        \includegraphics[width=\textwidth]{img/init/relu/conv2d_50-0.pdf}
        \caption{\ReLU 50th layer}
        \label{fig:reluInit501}
    \end{subfigure}
    ~ %add desired spacing between images, e. g. ~, \quad, \qquad, \hfill etc. 
    %(or a blank line to force the subfigure onto a new line)
    \begin{subfigure}[b]{0.3\textwidth}
        \includegraphics[width=\textwidth]{img/init/relu/conv2d_50-2.pdf}
        \caption{\ReLU 50th layer}
        \label{fig:reluInit502}
    \end{subfigure}
    \\
    \begin{subfigure}[b]{0.3\textwidth}
        \includegraphics[width=\textwidth]{img/init/relu-bn/conv2d_1-0.pdf}
        \caption{\ReLUBN input layer}
        \label{fig:reluBNInitInput}
    \end{subfigure}
    ~ %add desired spacing between images, e. g. ~, \quad, \qquad, \hfill etc. 
      %(or a blank line to force the subfigure onto a new line)
    \begin{subfigure}[b]{0.3\textwidth}
        \includegraphics[width=\textwidth]{img/init/relu-bn/conv2d_50-0.pdf}
        \caption{\ReLUBN 50th layer}
        \label{fig:reluBNInit501}
    \end{subfigure}
    ~ %add desired spacing between images, e. g. ~, \quad, \qquad, \hfill etc. 
    %(or a blank line to force the subfigure onto a new line)
    \begin{subfigure}[b]{0.3\textwidth}
        \includegraphics[width=\textwidth]{img/init/relu-bn/conv2d_50-2.pdf}
        \caption{\ReLUBN 50th layer}
        \label{fig:reluBNInit502}
    \end{subfigure}
    \\
    \begin{subfigure}[b]{0.3\textwidth}
        \includegraphics[width=\textwidth]{img/init/layerwise/conv2d_1-0.pdf}
        \caption{\SepLayer input layer}
        \label{fig:layerwiseInitInput}
    \end{subfigure}
    ~ %add desired spacing between images, e. g. ~, \quad, \qquad, \hfill etc. 
      %(or a blank line to force the subfigure onto a new line)
    \begin{subfigure}[b]{0.3\textwidth}
        \includegraphics[width=\textwidth]{img/init/layerwise/conv2d_50-0.pdf}
        \caption{\SepLayer 50th layer}
        \label{fig:layerwiseInit501}
    \end{subfigure}
    ~ %add desired spacing between images, e. g. ~, \quad, \qquad, \hfill etc. 
    %(or a blank line to force the subfigure onto a new line)
    \begin{subfigure}[b]{0.3\textwidth}
        \includegraphics[width=\textwidth]{img/init/layerwise/conv2d_50-2.pdf}
        \caption{\SepLayer 50th layer}
        \label{fig:layerwiseInit502}
    \end{subfigure}
    \\
    \begin{subfigure}[b]{0.3\textwidth}
        \includegraphics[width=\textwidth]{img/init/unitpointwise/conv2d_1-0.pdf}
        \caption{\SepUnitPoint Input}
        \label{fig:unitpointInitInput}
    \end{subfigure}
    ~ %add desired spacing between images, e. g. ~, \quad, \qquad, \hfill etc. 
      %(or a blank line to force the subfigure onto a new line)
    \begin{subfigure}[b]{0.3\textwidth}
        \includegraphics[width=\textwidth]{img/init/unitpointwise/conv2d_50-0.pdf}
        \caption{\SepUnitPoint 50th layer}
        \label{fig:unitpointInit501}
    \end{subfigure}
    ~ %add desired spacing between images, e. g. ~, \quad, \qquad, \hfill etc. 
    %(or a blank line to force the subfigure onto a new line)
    \begin{subfigure}[b]{0.3\textwidth}
        \includegraphics[width=\textwidth]{img/init/unitpointwise/conv2d_50-2.pdf}
        \caption{\SepUnitPoint 50th layer}
        \label{fig:unitpointInit502}
    \end{subfigure}
    
  \caption{Data transformed across a 50x4 network with no main loss (cross-entropy) with constraints \SepLayer and \SepUnitPoint, versus \ReLU and \ReLUBN. Notice how effectively \ReLU and \ReLUBN collapse the dataset into few points whereas \SepLayer and \SepUnitPoint force the network to learn representations that preserve geometrical structure useful for back-propagation.} 
  \label{fig:init} 
\end{figure*}



\begin{figure*}[h]
   
    \begin{subfigure}[c]{1\textwidth}
        \includegraphics[width=1\textwidth]{img/convergence/peaks_acc.pdf}
        \caption{Evolution of accuracy during training.}
        \label{fig:accuracy_convergence}
    \end{subfigure}
    \\
    
    \begin{subfigure}[c]{1\textwidth}
    \centering
        \includegraphics[width=1\textwidth]{img/convergence/peaks_loss.pdf}
        \caption{Evolution of cross-entropy and constraint loss during training.}
        \label{fig:loss_convergence}
    \end{subfigure}
    \\
   
\begin{subfigure}[b]{0.09\textwidth}
    \includegraphics[clip, trim=2.35cm 1.75cm 4.5cm 0cm,width=\textwidth]{img/convergence/0.pdf}
    \caption{0}
    \label{fig:convergence_0}
\end{subfigure}
%
\begin{subfigure}[b]{0.09\textwidth}
    \includegraphics[clip, trim=2.35cm 1.75cm 4.5cm 0cm,width=\textwidth]{img/convergence/100.pdf}
    \caption{100}
    \label{fig:convergence_100}
\end{subfigure}
%
\begin{subfigure}[b]{0.09\textwidth}
    \includegraphics[clip, trim=2.35cm 1.75cm 4.5cm 0cm,width=\textwidth]{img/convergence/200.pdf}
    \caption{200}
    \label{fig:convergence_200}
\end{subfigure}
%
\begin{subfigure}[b]{0.09\textwidth}
    \includegraphics[clip, trim=2.35cm 1.75cm 4.5cm 0cm,width=\textwidth]{img/convergence/246.pdf}
    \caption{246}
    \label{fig:convergence_246}
\end{subfigure}
%
\begin{subfigure}[b]{0.09\textwidth}
    \includegraphics[clip, trim=2.35cm 1.75cm 4.5cm 0cm,width=\textwidth]{img/convergence/250.pdf}
    \caption{250}
    \label{fig:convergence_250}
\end{subfigure}
%
\begin{subfigure}[b]{0.09\textwidth}
    \includegraphics[clip, trim=2.35cm 1.75cm 4.5cm 0cm,width=\textwidth]{img/convergence/253.pdf}
    \caption{253}
    \label{fig:convergence_253}
\end{subfigure}
%
\begin{subfigure}[b]{0.09\textwidth}
    \includegraphics[clip, trim=2.35cm 1.75cm 4.5cm 0cm,width=\textwidth]{img/convergence/255.pdf}
    \caption{255}
    \label{fig:convergence_255}
\end{subfigure}
%
\begin{subfigure}[b]{0.09\textwidth}
    \includegraphics[clip, trim=2.35cm 1.75cm 4.5cm 0cm,width=\textwidth]{img/convergence/256.pdf}
    \caption{256}
    \label{fig:convergence_256}
\end{subfigure}
%
\begin{subfigure}[b]{0.09\textwidth}
    \includegraphics[clip, trim=2.35cm 1.75cm 4.5cm 0cm,width=\textwidth]{img/convergence/259.pdf}
    \caption{259}
    \label{fig:convergence_259}
\end{subfigure}
%
\begin{subfigure}[b]{0.09\textwidth}
    \includegraphics[clip, trim=2.35cm 1.75cm 4.5cm 0cm,width=\textwidth]{img/convergence/260.pdf}
    \caption{260}
    \label{fig:convergence_260}
\end{subfigure}
%
\begin{subfigure}[b]{0.09\textwidth}
    \includegraphics[clip, trim=2.35cm 1.75cm 4.5cm 0cm,width=\textwidth]{img/convergence/261.pdf}
    \caption{261}
    \label{fig:convergence_261}
\end{subfigure}
%
\begin{subfigure}[b]{0.09\textwidth}
    \includegraphics[clip, trim=2.35cm 1.75cm 4.5cm 0cm,width=\textwidth]{img/convergence/262.pdf}
    \caption{262}
    \label{fig:convergence_262}
\end{subfigure}
%
\begin{subfigure}[b]{0.09\textwidth}
    \includegraphics[clip, trim=2.35cm 1.75cm 4.5cm 0cm,width=\textwidth]{img/convergence/263.pdf}
    \caption{263}
    \label{fig:convergence_263}
\end{subfigure}
%
\begin{subfigure}[b]{0.09\textwidth}
    \includegraphics[clip, trim=2.35cm 1.75cm 4.5cm 0cm,width=\textwidth]{img/convergence/268.pdf}
    \caption{268}
    \label{fig:convergence_268}
\end{subfigure}
%
\begin{subfigure}[b]{0.09\textwidth}
    \includegraphics[clip, trim=2.35cm 1.75cm 4.5cm 0cm,width=\textwidth]{img/convergence/269.pdf}
    \caption{269}
    \label{fig:convergence_269}
\end{subfigure}
%
\begin{subfigure}[b]{0.09\textwidth}
    \includegraphics[clip, trim=2.35cm 1.75cm 4.5cm 0cm,width=\textwidth]{img/convergence/270.pdf}
    \caption{270}
    \label{fig:convergence_270}
\end{subfigure}
%
\begin{subfigure}[b]{0.09\textwidth}
    \includegraphics[clip, trim=2.35cm 1.75cm 4.5cm 0cm,width=\textwidth]{img/convergence/271.pdf}
    \caption{271}
    \label{fig:convergence_271}
\end{subfigure}
%
\begin{subfigure}[b]{0.09\textwidth}
    \includegraphics[clip, trim=2.35cm 1.75cm 4.5cm 0cm,width=\textwidth]{img/convergence/279.pdf}
    \caption{279}
    \label{fig:convergence_279}
\end{subfigure}
%
\begin{subfigure}[b]{0.09\textwidth}
    \includegraphics[clip, trim=2.35cm 1.75cm 4.5cm 0cm,width=\textwidth]{img/convergence/281.pdf}
    \caption{281}
    \label{fig:convergence_281}
\end{subfigure}
%
\begin{subfigure}[b]{0.09\textwidth}
    \includegraphics[clip, trim=2.35cm 1.75cm 4.5cm 0cm,width=\textwidth]{img/convergence/284.pdf}
    \caption{284}
    \label{fig:convergence_284}
\end{subfigure}
%
\begin{subfigure}[b]{0.09\textwidth}
    \includegraphics[clip, trim=2.35cm 1.75cm 4.5cm 0cm,width=\textwidth]{img/convergence/285.pdf}
    \caption{285}
    \label{fig:convergence_285}
\end{subfigure}
%
\begin{subfigure}[b]{0.09\textwidth}
    \includegraphics[clip, trim=2.35cm 1.75cm 4.5cm 0cm,width=\textwidth]{img/convergence/286.pdf}
    \caption{286}
    \label{fig:convergence_286}
\end{subfigure}
%
\begin{subfigure}[b]{0.09\textwidth}
    \includegraphics[clip, trim=2.35cm 1.75cm 4.5cm 0cm,width=\textwidth]{img/convergence/287.pdf}
    \caption{287}
    \label{fig:convergence_287}
\end{subfigure}
%
\begin{subfigure}[b]{0.09\textwidth}
    \includegraphics[clip, trim=2.35cm 1.75cm 4.5cm 0cm,width=\textwidth]{img/convergence/288.pdf}
    \caption{288}
    \label{fig:convergence_288}
\end{subfigure}
%
\begin{subfigure}[b]{0.09\textwidth}
    \includegraphics[clip, trim=2.35cm 1.75cm 4.5cm 0cm,width=\textwidth]{img/convergence/290.pdf}
    \caption{290}
    \label{fig:convergence_290}
\end{subfigure}
%
\begin{subfigure}[b]{0.09\textwidth}
    \includegraphics[clip, trim=2.35cm 1.75cm 4.5cm 0cm,width=\textwidth]{img/convergence/292.pdf}
    \caption{292}
    \label{fig:convergence_292}
\end{subfigure}
%
\begin{subfigure}[b]{0.09\textwidth}
    \includegraphics[clip, trim=2.35cm 1.75cm 4.5cm 0cm,width=\textwidth]{img/convergence/300.pdf}
    \caption{300}
    \label{fig:convergence_300}
\end{subfigure}
%
\begin{subfigure}[b]{0.09\textwidth}
    \includegraphics[clip, trim=2.35cm 1.75cm 4.5cm 0cm,width=\textwidth]{img/convergence/305.pdf}
    \caption{305}
    \label{fig:convergence_305}
\end{subfigure}
%
\begin{subfigure}[b]{0.09\textwidth}
    \includegraphics[clip, trim=2.35cm 1.75cm 4.5cm 0cm,width=\textwidth]{img/convergence/306.pdf}
    \caption{306}
    \label{fig:convergence_306}
\end{subfigure}
%
\begin{subfigure}[b]{0.09\textwidth}
    \includegraphics[clip, trim=2.35cm 1.75cm 4.5cm 0cm,width=\textwidth]{img/convergence/307.pdf}
    \caption{307}
    \label{fig:convergence_307}
\end{subfigure}
%
\begin{subfigure}[b]{0.09\textwidth}
    \includegraphics[clip, trim=2.35cm 1.75cm 4.5cm 0cm,width=\textwidth]{img/convergence/308.pdf}
    \caption{308}
    \label{fig:convergence_308}
\end{subfigure}
%
\begin{subfigure}[b]{0.09\textwidth}
    \includegraphics[clip, trim=2.35cm 1.75cm 4.5cm 0cm,width=\textwidth]{img/convergence/310.pdf}
    \caption{310}
    \label{fig:convergence_310}
\end{subfigure}
%
\begin{subfigure}[b]{0.09\textwidth}
    \includegraphics[clip, trim=2.35cm 1.75cm 4.5cm 0cm,width=\textwidth]{img/convergence/311.pdf}
    \caption{311}
    \label{fig:convergence_311}
\end{subfigure}
%
\begin{subfigure}[b]{0.09\textwidth}
    \includegraphics[clip, trim=2.35cm 1.75cm 4.5cm 0cm,width=\textwidth]{img/convergence/312.pdf}
    \caption{312}
    \label{fig:convergence_312}
\end{subfigure}
%
\begin{subfigure}[b]{0.09\textwidth}
    \includegraphics[clip, trim=2.35cm 1.75cm 4.5cm 0cm,width=\textwidth]{img/convergence/318.pdf}
    \caption{318}
    \label{fig:convergence_318}
\end{subfigure}
%
\begin{subfigure}[b]{0.09\textwidth}
    \includegraphics[clip, trim=2.35cm 1.75cm 4.5cm 0cm,width=\textwidth]{img/convergence/321.pdf}
    \caption{321}
    \label{fig:convergence_321}
\end{subfigure}
%
\begin{subfigure}[b]{0.09\textwidth}
    \includegraphics[clip, trim=2.35cm 1.75cm 4.5cm 0cm,width=\textwidth]{img/convergence/322.pdf}
    \caption{322}
    \label{fig:convergence_322}
\end{subfigure}
%
\begin{subfigure}[b]{0.09\textwidth}
    \includegraphics[clip, trim=2.35cm 1.75cm 4.5cm 0cm,width=\textwidth]{img/convergence/324.pdf}
    \caption{324}
    \label{fig:convergence_324}
\end{subfigure}
%
\begin{subfigure}[b]{0.09\textwidth}
    \includegraphics[clip, trim=2.35cm 1.75cm 4.5cm 0cm,width=\textwidth]{img/convergence/325.pdf}
    \caption{325}
    \label{fig:convergence_325}
\end{subfigure}
%
\begin{subfigure}[b]{0.09\textwidth}
    \includegraphics[clip, trim=2.35cm 1.75cm 4.5cm 0cm,width=\textwidth]{img/convergence/326.pdf}
    \caption{326}
    \label{fig:convergence_326}
\end{subfigure}
%
\begin{subfigure}[b]{0.09\textwidth}
    \includegraphics[clip, trim=2.35cm 1.75cm 4.5cm 0cm,width=\textwidth]{img/convergence/328.pdf}
    \caption{328}
    \label{fig:convergence_328}
\end{subfigure}
%
\begin{subfigure}[b]{0.09\textwidth}
    \includegraphics[clip, trim=2.35cm 1.75cm 4.5cm 0cm,width=\textwidth]{img/convergence/332.pdf}
    \caption{332}
    \label{fig:convergence_332}
\end{subfigure}
%
\begin{subfigure}[b]{0.09\textwidth}
    \includegraphics[clip, trim=2.35cm 1.75cm 4.5cm 0cm,width=\textwidth]{img/convergence/335.pdf}
    \caption{335}
    \label{fig:convergence_335}
\end{subfigure}
%
\begin{subfigure}[b]{0.09\textwidth}
    \includegraphics[clip, trim=2.35cm 1.75cm 4.5cm 0cm,width=\textwidth]{img/convergence/337.pdf}
    \caption{337}
    \label{fig:convergence_337}
\end{subfigure}
%
\begin{subfigure}[b]{0.09\textwidth}
    \includegraphics[clip, trim=2.35cm 1.75cm 4.5cm 0cm,width=\textwidth]{img/convergence/344.pdf}
    \caption{344}
    \label{fig:convergence_344}
\end{subfigure}
%
\begin{subfigure}[b]{0.09\textwidth}
    \includegraphics[clip, trim=2.35cm 1.75cm 4.5cm 0cm,width=\textwidth]{img/convergence/345.pdf}
    \caption{345}
    \label{fig:convergence_345}
\end{subfigure}
%
\begin{subfigure}[b]{0.09\textwidth}
    \includegraphics[clip, trim=2.35cm 1.75cm 4.5cm 0cm,width=\textwidth]{img/convergence/347.pdf}
    \caption{347}
    \label{fig:convergence_347}
\end{subfigure}
%
\begin{subfigure}[b]{0.09\textwidth}
    \includegraphics[clip, trim=2.35cm 1.75cm 4.5cm 0cm,width=\textwidth]{img/convergence/349.pdf}
    \caption{349}
    \label{fig:convergence_349}
\end{subfigure}
%
\begin{subfigure}[b]{0.09\textwidth}
    \includegraphics[clip, trim=2.35cm 1.75cm 4.5cm 0cm,width=\textwidth]{img/convergence/350.pdf}
    \caption{350}
    \label{fig:convergence_350}
\end{subfigure}
%
\begin{subfigure}[b]{0.09\textwidth}
    \includegraphics[clip, trim=2.35cm 1.75cm 4.5cm 0cm,width=\textwidth]{img/convergence/351.pdf}
    \caption{351}
    \label{fig:convergence_351}
\end{subfigure}
%
\begin{subfigure}[b]{0.09\textwidth}
    \includegraphics[clip, trim=2.35cm 1.75cm 4.5cm 0cm,width=\textwidth]{img/convergence/353.pdf}
    \caption{353}
    \label{fig:convergence_353}
\end{subfigure}
%
\begin{subfigure}[b]{0.09\textwidth}
    \includegraphics[clip, trim=2.35cm 1.75cm 4.5cm 0cm,width=\textwidth]{img/convergence/354.pdf}
    \caption{354}
    \label{fig:convergence_354}
\end{subfigure}
%
\begin{subfigure}[b]{0.09\textwidth}
    \includegraphics[clip, trim=2.35cm 1.75cm 4.5cm 0cm,width=\textwidth]{img/convergence/356.pdf}
    \caption{356}
    \label{fig:convergence_356}
\end{subfigure}
%
\begin{subfigure}[b]{0.09\textwidth}
    \includegraphics[clip, trim=2.35cm 1.75cm 4.5cm 0cm,width=\textwidth]{img/convergence/359.pdf}
    \caption{359}
    \label{fig:convergence_359}
\end{subfigure}
%
\begin{subfigure}[b]{0.09\textwidth}
    \includegraphics[clip, trim=2.35cm 1.75cm 4.5cm 0cm,width=\textwidth]{img/convergence/361.pdf}
    \caption{361}
    \label{fig:convergence_361}
\end{subfigure}
%
\begin{subfigure}[b]{0.09\textwidth}
    \includegraphics[clip, trim=2.35cm 1.75cm 4.5cm 0cm,width=\textwidth]{img/convergence/362.pdf}
    \caption{362}
    \label{fig:convergence_362}
\end{subfigure}
%
\begin{subfigure}[b]{0.09\textwidth}
    \includegraphics[clip, trim=2.35cm 1.75cm 4.5cm 0cm,width=\textwidth]{img/convergence/363.pdf}
    \caption{363}
    \label{fig:convergence_363}
\end{subfigure}
%
\begin{subfigure}[b]{0.09\textwidth}
    \includegraphics[clip, trim=2.35cm 1.75cm 4.5cm 0cm,width=\textwidth]{img/convergence/376.pdf}
    \caption{376}
    \label{fig:convergence_376}
\end{subfigure}
%
\begin{subfigure}[b]{0.09\textwidth}
    \includegraphics[clip, trim=2.35cm 1.75cm 4.5cm 0cm,width=\textwidth]{img/convergence/377.pdf}
    \caption{377}
    \label{fig:convergence_377}
\end{subfigure}
%
\begin{subfigure}[b]{0.09\textwidth}
    \includegraphics[clip, trim=2.35cm 1.75cm 4.5cm 0cm,width=\textwidth]{img/convergence/379.pdf}
    \caption{379}
    \label{fig:convergence_379}
\end{subfigure}
%
\begin{subfigure}[b]{0.09\textwidth}
    \includegraphics[clip, trim=2.35cm 1.75cm 4.5cm 0cm,width=\textwidth]{img/convergence/380.pdf}
    \caption{380}
    \label{fig:convergence_380}
\end{subfigure}
%
\begin{subfigure}[b]{0.09\textwidth}
    \includegraphics[clip, trim=2.35cm 1.75cm 4.5cm 0cm,width=\textwidth]{img/convergence/384.pdf}
    \caption{384}
    \label{fig:convergence_384}
\end{subfigure}
%
\begin{subfigure}[b]{0.09\textwidth}
    \includegraphics[clip, trim=2.35cm 1.75cm 4.5cm 0cm,width=\textwidth]{img/convergence/385.pdf}
    \caption{385}
    \label{fig:convergence_385}
\end{subfigure}
%
\begin{subfigure}[b]{0.09\textwidth}
    \includegraphics[clip, trim=2.35cm 1.75cm 4.5cm 0cm,width=\textwidth]{img/convergence/388.pdf}
    \caption{388}
    \label{fig:convergence_388}
\end{subfigure}
%
\begin{subfigure}[b]{0.09\textwidth}
    \includegraphics[clip, trim=2.35cm 1.75cm 4.5cm 0cm,width=\textwidth]{img/convergence/389.pdf}
    \caption{389}
    \label{fig:convergence_389}
\end{subfigure}
%
\begin{subfigure}[b]{0.09\textwidth}
    \includegraphics[clip, trim=2.35cm 1.75cm 4.5cm 0cm,width=\textwidth]{img/convergence/392.pdf}
    \caption{392}
    \label{fig:convergence_392}
\end{subfigure}
%
\begin{subfigure}[b]{0.09\textwidth}
    \includegraphics[clip, trim=2.35cm 1.75cm 4.5cm 0cm,width=\textwidth]{img/convergence/393.pdf}
    \caption{393}
    \label{fig:convergence_393}
\end{subfigure}
%
\begin{subfigure}[b]{0.09\textwidth}
    \includegraphics[clip, trim=2.35cm 1.75cm 4.5cm 0cm,width=\textwidth]{img/convergence/394.pdf}
    \caption{394}
    \label{fig:convergence_394}
\end{subfigure}
%
\begin{subfigure}[b]{0.09\textwidth}
    \includegraphics[clip, trim=2.35cm 1.75cm 4.5cm 0cm,width=\textwidth]{img/convergence/395.pdf}
    \caption{395}
    \label{fig:convergence_395}
\end{subfigure}
%
\begin{subfigure}[b]{0.09\textwidth}
    \includegraphics[clip, trim=2.35cm 1.75cm 4.5cm 0cm,width=\textwidth]{img/convergence/397.pdf}
    \caption{397}
    \label{fig:convergence_397}
\end{subfigure}
%
\begin{subfigure}[b]{0.09\textwidth}
    \includegraphics[clip, trim=2.35cm 1.75cm 4.5cm 0cm,width=\textwidth]{img/convergence/398.pdf}
    \caption{398}
    \label{fig:convergence_398}
\end{subfigure}
%
\begin{subfigure}[b]{0.09\textwidth}
    \includegraphics[clip, trim=2.35cm 1.75cm 4.5cm 0cm,width=\textwidth]{img/convergence/400.pdf}
    \caption{400}
    \label{fig:convergence_400}
\end{subfigure}
%
\begin{subfigure}[b]{0.09\textwidth}
    \includegraphics[clip, trim=2.35cm 1.75cm 4.5cm 0cm,width=\textwidth]{img/convergence/401.pdf}
    \caption{401}
    \label{fig:convergence_401}
\end{subfigure}
%
\begin{subfigure}[b]{0.09\textwidth}
    \includegraphics[clip, trim=2.35cm 1.75cm 4.5cm 0cm,width=\textwidth]{img/convergence/402.pdf}
    \caption{402}
    \label{fig:convergence_402}
\end{subfigure}
%
\begin{subfigure}[b]{0.09\textwidth}
    \includegraphics[clip, trim=2.35cm 1.75cm 4.5cm 0cm,width=\textwidth]{img/convergence/404.pdf}
    \caption{404}
    \label{fig:convergence_404}
\end{subfigure}
%
\begin{subfigure}[b]{0.09\textwidth}
    \includegraphics[clip, trim=2.35cm 1.75cm 4.5cm 0cm,width=\textwidth]{img/convergence/405.pdf}
    \caption{405}
    \label{fig:convergence_405}
\end{subfigure}
%
\begin{subfigure}[b]{0.09\textwidth}
    \includegraphics[clip, trim=2.35cm 1.75cm 4.5cm 0cm,width=\textwidth]{img/convergence/407.pdf}
    \caption{407}
    \label{fig:convergence_407}
\end{subfigure}
%
\begin{subfigure}[b]{0.09\textwidth}
    \includegraphics[clip, trim=2.35cm 1.75cm 4.5cm 0cm,width=\textwidth]{img/convergence/408.pdf}
    \caption{408}
    \label{fig:convergence_408}
\end{subfigure}
%
\begin{subfigure}[b]{0.09\textwidth}
    \includegraphics[clip, trim=2.35cm 1.75cm 4.5cm 0cm,width=\textwidth]{img/convergence/410.pdf}
    \caption{410}
    \label{fig:convergence_410}
\end{subfigure}
%
\begin{subfigure}[b]{0.09\textwidth}
    \includegraphics[clip, trim=2.35cm 1.75cm 4.5cm 0cm,width=\textwidth]{img/convergence/413.pdf}
    \caption{413}
    \label{fig:convergence_413}
\end{subfigure}
%
\begin{subfigure}[b]{0.09\textwidth}
    \includegraphics[clip, trim=2.35cm 1.75cm 4.5cm 0cm,width=\textwidth]{img/convergence/415.pdf}
    \caption{415}
    \label{fig:convergence_415}
\end{subfigure}
%
\begin{subfigure}[b]{0.09\textwidth}
    \includegraphics[clip, trim=2.35cm 1.75cm 4.5cm 0cm,width=\textwidth]{img/convergence/419.pdf}
    \caption{419}
    \label{fig:convergence_419}
\end{subfigure}
%
\begin{subfigure}[b]{0.09\textwidth}
    \includegraphics[clip, trim=2.35cm 1.75cm 4.5cm 0cm,width=\textwidth]{img/convergence/421.pdf}
    \caption{421}
    \label{fig:convergence_421}
\end{subfigure}
%
\begin{subfigure}[b]{0.09\textwidth}
    \includegraphics[clip, trim=2.35cm 1.75cm 4.5cm 0cm,width=\textwidth]{img/convergence/425.pdf}
    \caption{425}
    \label{fig:convergence_425}
\end{subfigure}
%
\begin{subfigure}[b]{0.09\textwidth}
    \includegraphics[clip, trim=2.35cm 1.75cm 4.5cm 0cm,width=\textwidth]{img/convergence/426.pdf}
    \caption{426}
    \label{fig:convergence_426}
\end{subfigure}
%
\begin{subfigure}[b]{0.09\textwidth}
    \includegraphics[clip, trim=2.35cm 1.75cm 4.5cm 0cm,width=\textwidth]{img/convergence/448.pdf}
    \caption{448}
    \label{fig:convergence_448}
\end{subfigure}
%
\begin{subfigure}[b]{0.09\textwidth}
    \includegraphics[clip, trim=2.35cm 1.75cm 4.5cm 0cm,width=\textwidth]{img/convergence/457.pdf}
    \caption{457}
    \label{fig:convergence_457}
\end{subfigure}
%
\begin{subfigure}[b]{0.09\textwidth}
    \includegraphics[clip, trim=2.35cm 1.75cm 4.5cm 0cm,width=\textwidth]{img/convergence/499.pdf}
    \caption{499}
    \label{fig:convergence_499}
\end{subfigure}
%
\begin{subfigure}[b]{0.09\textwidth}
    \includegraphics[clip, trim=2.35cm 1.75cm 4.5cm 0cm,width=\textwidth]{img/convergence/500.pdf}
    \caption{500}
    \label{fig:convergence_500}
\end{subfigure}
%
\begin{subfigure}[b]{0.09\textwidth}
    \includegraphics[clip, trim=2.35cm 1.75cm 4.5cm 0cm,width=\textwidth]{img/convergence/515.pdf}
    \caption{515}
    \label{fig:convergence_515}
\end{subfigure}
%
\begin{subfigure}[b]{0.09\textwidth}
    \includegraphics[clip, trim=2.35cm 1.75cm 4.5cm 0cm,width=\textwidth]{img/convergence/563.pdf}
    \caption{563}
    \label{fig:convergence_563}
\end{subfigure}
%
\begin{subfigure}[b]{0.09\textwidth}
    \includegraphics[clip, trim=2.35cm 1.75cm 4.5cm 0cm,width=\textwidth]{img/convergence/565.pdf}
    \caption{565}
    \label{fig:convergence_565}
\end{subfigure}
%
\begin{subfigure}[b]{0.09\textwidth}
    \includegraphics[clip, trim=2.35cm 1.75cm 4.5cm 0cm,width=\textwidth]{img/convergence/573.pdf}
    \caption{573}
    \label{fig:convergence_573}
\end{subfigure}
%
\begin{subfigure}[b]{0.09\textwidth}
    \includegraphics[clip, trim=2.35cm 1.75cm 4.5cm 0cm,width=\textwidth]{img/convergence/574.pdf}
    \caption{574}
    \label{fig:convergence_574}
\end{subfigure}
%
\begin{subfigure}[b]{0.09\textwidth}
    \includegraphics[clip, trim=2.35cm 1.75cm 4.5cm 0cm,width=\textwidth]{img/convergence/576.pdf}
    \caption{576}
    \label{fig:convergence_576}
\end{subfigure}
%
\begin{subfigure}[b]{0.09\textwidth}
    \includegraphics[clip, trim=2.35cm 1.75cm 4.5cm 0cm,width=\textwidth]{img/convergence/579.pdf}
    \caption{579}
    \label{fig:convergence_579}
\end{subfigure}
%
\begin{subfigure}[b]{0.09\textwidth}
    \includegraphics[clip, trim=2.35cm 1.75cm 4.5cm 0cm,width=\textwidth]{img/convergence/581.pdf}
    \caption{581}
    \label{fig:convergence_581}
\end{subfigure}
%
\begin{subfigure}[b]{0.09\textwidth}
    \includegraphics[clip, trim=2.35cm 1.75cm 4.5cm 0cm,width=\textwidth]{img/convergence/584.pdf}
    \caption{584}
    \label{fig:convergence_584}
\end{subfigure}
%
\begin{subfigure}[b]{0.09\textwidth}
    \includegraphics[clip, trim=2.35cm 1.75cm 4.5cm 0cm,width=\textwidth]{img/convergence/586.pdf}
    \caption{586}
    \label{fig:convergence_586}
\end{subfigure}
%
\begin{subfigure}[b]{0.09\textwidth}
    \includegraphics[clip, trim=2.35cm 1.75cm 4.5cm 0cm,width=\textwidth]{img/convergence/600.pdf}
    \caption{600}
    \label{fig:convergence_600}
\end{subfigure}
%
\begin{subfigure}[b]{0.09\textwidth}
    \includegraphics[clip, trim=2.35cm 1.75cm 4.5cm 0cm,width=\textwidth]{img/convergence/609.pdf}
    \caption{609}
    \label{fig:convergence_609}
\end{subfigure}
%
\begin{subfigure}[b]{0.09\textwidth}
    \includegraphics[clip, trim=2.35cm 1.75cm 4.5cm 0cm,width=\textwidth]{img/convergence/610.pdf}
    \caption{610}
    \label{fig:convergence_610}
\end{subfigure}
%
\begin{subfigure}[b]{0.09\textwidth}
    \includegraphics[clip, trim=2.35cm 1.75cm 4.5cm 0cm,width=\textwidth]{img/convergence/613.pdf}
    \caption{613}
    \label{fig:convergence_613}
\end{subfigure}
%
\begin{subfigure}[b]{0.09\textwidth}
    \includegraphics[clip, trim=2.35cm 1.75cm 4.5cm 0cm,width=\textwidth]{img/convergence/615.pdf}
    \caption{615}
    \label{fig:convergence_615}
\end{subfigure}
%
\begin{subfigure}[b]{0.09\textwidth}
    \includegraphics[clip, trim=2.35cm 1.75cm 4.5cm 0cm,width=\textwidth]{img/convergence/622.pdf}
    \caption{622}
    \label{fig:convergence_622}
\end{subfigure}
%
\begin{subfigure}[b]{0.09\textwidth}
    \includegraphics[clip, trim=2.35cm 1.75cm 4.5cm 0cm,width=\textwidth]{img/convergence/625.pdf}
    \caption{625}
    \label{fig:convergence_625}
\end{subfigure}
%
\begin{subfigure}[b]{0.09\textwidth}
    \includegraphics[clip, trim=2.35cm 1.75cm 4.5cm 0cm,width=\textwidth]{img/convergence/626.pdf}
    \caption{626}
    \label{fig:convergence_626}
\end{subfigure}
%
\begin{subfigure}[b]{0.09\textwidth}
    \includegraphics[clip, trim=2.35cm 1.75cm 4.5cm 0cm,width=\textwidth]{img/convergence/632.pdf}
    \caption{632}
    \label{fig:convergence_632}
\end{subfigure}
%
\begin{subfigure}[b]{0.09\textwidth}
    \includegraphics[clip, trim=2.35cm 1.75cm 4.5cm 0cm,width=\textwidth]{img/convergence/634.pdf}
    \caption{634}
    \label{fig:convergence_634}
\end{subfigure}
%
\begin{subfigure}[b]{0.09\textwidth}
    \includegraphics[clip, trim=2.35cm 1.75cm 4.5cm 0cm,width=\textwidth]{img/convergence/635.pdf}
    \caption{635}
    \label{fig:convergence_635}
\end{subfigure}
%
\begin{subfigure}[b]{0.09\textwidth}
    \includegraphics[clip, trim=2.35cm 1.75cm 4.5cm 0cm,width=\textwidth]{img/convergence/689.pdf}
    \caption{689}
    \label{fig:convergence_689}
\end{subfigure}
%
\begin{subfigure}[b]{0.09\textwidth}
    \includegraphics[clip, trim=2.35cm 1.75cm 4.5cm 0cm,width=\textwidth]{img/convergence/690.pdf}
    \caption{690}
    \label{fig:convergence_690}
\end{subfigure}
%
\begin{subfigure}[b]{0.09\textwidth}
    \includegraphics[clip, trim=2.35cm 1.75cm 4.5cm 0cm,width=\textwidth]{img/convergence/693.pdf}
    \caption{693}
    \label{fig:convergence_693}
\end{subfigure}
%
\begin{subfigure}[b]{0.09\textwidth}
    \includegraphics[clip, trim=2.35cm 1.75cm 4.5cm 0cm,width=\textwidth]{img/convergence/695.pdf}
    \caption{695}
    \label{fig:convergence_695}
\end{subfigure}
%
\begin{subfigure}[b]{0.09\textwidth}
    \includegraphics[clip, trim=2.35cm 1.75cm 4.5cm 0cm,width=\textwidth]{img/convergence/697.pdf}
    \caption{697}
    \label{fig:convergence_697}
\end{subfigure}
%
\begin{subfigure}[b]{0.09\textwidth}
    \includegraphics[clip, trim=2.35cm 1.75cm 4.5cm 0cm,width=\textwidth]{img/convergence/698.pdf}
    \caption{698}
    \label{fig:convergence_698}
\end{subfigure}
%
\begin{subfigure}[b]{0.09\textwidth}
    \includegraphics[clip, trim=2.35cm 1.75cm 4.5cm 0cm,width=\textwidth]{img/convergence/699.pdf}
    \caption{699}
    \label{fig:convergence_699}
\end{subfigure}
%
\begin{subfigure}[b]{0.09\textwidth}
    \includegraphics[clip, trim=2.35cm 1.75cm 4.5cm 0cm,width=\textwidth]{img/convergence/700.pdf}
    \caption{700}
    \label{fig:convergence_700}
\end{subfigure}
%
\begin{subfigure}[b]{0.09\textwidth}
    \includegraphics[clip, trim=2.35cm 1.75cm 4.5cm 0cm,width=\textwidth]{img/convergence/702.pdf}
    \caption{702}
    \label{fig:convergence_702}
\end{subfigure}
%
\begin{subfigure}[b]{0.09\textwidth}
    \includegraphics[clip, trim=2.35cm 1.75cm 4.5cm 0cm,width=\textwidth]{img/convergence/705.pdf}
    \caption{705}
    \label{fig:convergence_705}
\end{subfigure}
%
\begin{subfigure}[b]{0.09\textwidth}
    \includegraphics[clip, trim=2.35cm 1.75cm 4.5cm 0cm,width=\textwidth]{img/convergence/710.pdf}
    \caption{710}
    \label{fig:convergence_710}
\end{subfigure}
%
\begin{subfigure}[b]{0.09\textwidth}
    \includegraphics[clip, trim=2.35cm 1.75cm 4.5cm 0cm,width=\textwidth]{img/convergence/712.pdf}
    \caption{712}
    \label{fig:convergence_712}
\end{subfigure}
%
\begin{subfigure}[b]{0.09\textwidth}
    \includegraphics[clip, trim=2.35cm 1.75cm 4.5cm 0cm,width=\textwidth]{img/convergence/792.pdf}
    \caption{792}
    \label{fig:convergence_792}
\end{subfigure}
%
\begin{subfigure}[b]{0.09\textwidth}
    \includegraphics[clip, trim=2.35cm 1.75cm 4.5cm 0cm,width=\textwidth]{img/convergence/793.pdf}
    \caption{793}
    \label{fig:convergence_793}
\end{subfigure}
%
\begin{subfigure}[b]{0.09\textwidth}
    \includegraphics[clip, trim=2.35cm 1.75cm 4.5cm 0cm,width=\textwidth]{img/convergence/794.pdf}
    \caption{794}
    \label{fig:convergence_794}
\end{subfigure}
%
\begin{subfigure}[b]{0.09\textwidth}
    \includegraphics[clip, trim=2.35cm 1.75cm 4.5cm 0cm,width=\textwidth]{img/convergence/795.pdf}
    \caption{795}
    \label{fig:convergence_795}
\end{subfigure}
%
\begin{subfigure}[b]{0.09\textwidth}
    \includegraphics[clip, trim=2.35cm 1.75cm 4.5cm 0cm,width=\textwidth]{img/convergence/798.pdf}
    \caption{798}
    \label{fig:convergence_798}
\end{subfigure}
%
\begin{subfigure}[b]{0.09\textwidth}
    \includegraphics[clip, trim=2.35cm 1.75cm 4.5cm 0cm,width=\textwidth]{img/convergence/800.pdf}
    \caption{800}
    \label{fig:convergence_800}
\end{subfigure}
%
\begin{subfigure}[b]{0.09\textwidth}
    \includegraphics[clip, trim=2.35cm 1.75cm 4.5cm 0cm,width=\textwidth]{img/convergence/801.pdf}
    \caption{801}
    \label{fig:convergence_801}
\end{subfigure}
%
\begin{subfigure}[b]{0.09\textwidth}
    \includegraphics[clip, trim=2.35cm 1.75cm 4.5cm 0cm,width=\textwidth]{img/convergence/803.pdf}
    \caption{803}
    \label{fig:convergence_803}
\end{subfigure}
%
\begin{subfigure}[b]{0.09\textwidth}
    \includegraphics[clip, trim=2.35cm 1.75cm 4.5cm 0cm,width=\textwidth]{img/convergence/804.pdf}
    \caption{804}
    \label{fig:convergence_804}
\end{subfigure}
%
\begin{subfigure}[b]{0.09\textwidth}
    \includegraphics[clip, trim=2.35cm 1.75cm 4.5cm 0cm,width=\textwidth]{img/convergence/805.pdf}
    \caption{805}
    \label{fig:convergence_805}
\end{subfigure}
%
\begin{subfigure}[b]{0.09\textwidth}
    \includegraphics[clip, trim=2.35cm 1.75cm 4.5cm 0cm,width=\textwidth]{img/convergence/806.pdf}
    \caption{806}
    \label{fig:convergence_806}
\end{subfigure}
%
\begin{subfigure}[b]{0.09\textwidth}
    \includegraphics[clip, trim=2.35cm 1.75cm 4.5cm 0cm,width=\textwidth]{img/convergence/811.pdf}
    \caption{811}
    \label{fig:convergence_811}
\end{subfigure}
%
\begin{subfigure}[b]{0.09\textwidth}
    \includegraphics[clip, trim=2.35cm 1.75cm 4.5cm 0cm,width=\textwidth]{img/convergence/813.pdf}
    \caption{813}
    \label{fig:convergence_813}
\end{subfigure}
%
\begin{subfigure}[b]{0.09\textwidth}
    \includegraphics[clip, trim=2.35cm 1.75cm 4.5cm 0cm,width=\textwidth]{img/convergence/822.pdf}
    \caption{822}
    \label{fig:convergence_822}
\end{subfigure}
%
\begin{subfigure}[b]{0.09\textwidth}
    \includegraphics[clip, trim=2.35cm 1.75cm 4.5cm 0cm,width=\textwidth]{img/convergence/824.pdf}
    \caption{824}
    \label{fig:convergence_824}
\end{subfigure}
%
\begin{subfigure}[b]{0.09\textwidth}
    \includegraphics[clip, trim=2.35cm 1.75cm 4.5cm 0cm,width=\textwidth]{img/convergence/825.pdf}
    \caption{825}
    \label{fig:convergence_825}
\end{subfigure}
%
\begin{subfigure}[b]{0.09\textwidth}
    \includegraphics[clip, trim=2.35cm 1.75cm 4.5cm 0cm,width=\textwidth]{img/convergence/830.pdf}
    \caption{830}
    \label{fig:convergence_830}
\end{subfigure}
%
\begin{subfigure}[b]{0.09\textwidth}
    \includegraphics[clip, trim=2.35cm 1.75cm 4.5cm 0cm,width=\textwidth]{img/convergence/833.pdf}
    \caption{833}
    \label{fig:convergence_833}
\end{subfigure}
%
\begin{subfigure}[b]{0.09\textwidth}
    \includegraphics[clip, trim=2.35cm 1.75cm 4.5cm 0cm,width=\textwidth]{img/convergence/839.pdf}
    \caption{839}
    \label{fig:convergence_839}
\end{subfigure}
%
\begin{subfigure}[b]{0.09\textwidth}
    \includegraphics[clip, trim=2.35cm 1.75cm 4.5cm 0cm,width=\textwidth]{img/convergence/841.pdf}
    \caption{841}
    \label{fig:convergence_841}
\end{subfigure}
%
\begin{subfigure}[b]{0.09\textwidth}
    \includegraphics[clip, trim=2.35cm 1.75cm 4.5cm 0cm,width=\textwidth]{img/convergence/868.pdf}
    \caption{868}
    \label{fig:convergence_868}
\end{subfigure}
%
\begin{subfigure}[b]{0.09\textwidth}
    \includegraphics[clip, trim=2.35cm 1.75cm 4.5cm 0cm,width=\textwidth]{img/convergence/869.pdf}
    \caption{869}
    \label{fig:convergence_869}
\end{subfigure}
%
\begin{subfigure}[b]{0.09\textwidth}
    \includegraphics[clip, trim=2.35cm 1.75cm 4.5cm 0cm,width=\textwidth]{img/convergence/871.pdf}
    \caption{871}
    \label{fig:convergence_871}
\end{subfigure}
%
\begin{subfigure}[b]{0.09\textwidth}
    \includegraphics[clip, trim=2.35cm 1.75cm 4.5cm 0cm,width=\textwidth]{img/convergence/872.pdf}
    \caption{872}
    \label{fig:convergence_872}
\end{subfigure}
%
\begin{subfigure}[b]{0.09\textwidth}
    \includegraphics[clip, trim=2.35cm 1.75cm 4.5cm 0cm,width=\textwidth]{img/convergence/874.pdf}
    \caption{874}
    \label{fig:convergence_874}
\end{subfigure}
%
\begin{subfigure}[b]{0.09\textwidth}
    \includegraphics[clip, trim=2.35cm 1.75cm 4.5cm 0cm,width=\textwidth]{img/convergence/875.pdf}
    \caption{875}
    \label{fig:convergence_875}
\end{subfigure}
%
\begin{subfigure}[b]{0.09\textwidth}
    \includegraphics[clip, trim=2.35cm 1.75cm 4.5cm 0cm,width=\textwidth]{img/convergence/877.pdf}
    \caption{877}
    \label{fig:convergence_877}
\end{subfigure}
%
\begin{subfigure}[b]{0.09\textwidth}
    \includegraphics[clip, trim=2.35cm 1.75cm 4.5cm 0cm,width=\textwidth]{img/convergence/878.pdf}
    \caption{878}
    \label{fig:convergence_878}
\end{subfigure}
%
\begin{subfigure}[b]{0.09\textwidth}
    \includegraphics[clip, trim=2.35cm 1.75cm 4.5cm 0cm,width=\textwidth]{img/convergence/880.pdf}
    \caption{880}
    \label{fig:convergence_880}
\end{subfigure}
%
\begin{subfigure}[b]{0.09\textwidth}
    \includegraphics[clip, trim=2.35cm 1.75cm 4.5cm 0cm,width=\textwidth]{img/convergence/881.pdf}
    \caption{881}
    \label{fig:convergence_881}
\end{subfigure}
%
\begin{subfigure}[b]{0.09\textwidth}
    \includegraphics[clip, trim=2.35cm 1.75cm 4.5cm 0cm,width=\textwidth]{img/convergence/885.pdf}
    \caption{885}
    \label{fig:convergence_885}
\end{subfigure}
%
\begin{subfigure}[b]{0.09\textwidth}
    \includegraphics[clip, trim=2.35cm 1.75cm 4.5cm 0cm,width=\textwidth]{img/convergence/886.pdf}
    \caption{886}
    \label{fig:convergence_886}
\end{subfigure}
%
\begin{subfigure}[b]{0.09\textwidth}
    \includegraphics[clip, trim=2.35cm 1.75cm 4.5cm 0cm,width=\textwidth]{img/convergence/888.pdf}
    \caption{888}
    \label{fig:convergence_888}
\end{subfigure}
%
\begin{subfigure}[b]{0.09\textwidth}
    \includegraphics[clip, trim=2.35cm 1.75cm 4.5cm 0cm,width=\textwidth]{img/convergence/889.pdf}
    \caption{889}
    \label{fig:convergence_889}
\end{subfigure}
%
\begin{subfigure}[b]{0.09\textwidth}
    \includegraphics[clip, trim=2.35cm 1.75cm 4.5cm 0cm,width=\textwidth]{img/convergence/891.pdf}
    \caption{891}
    \label{fig:convergence_891}
\end{subfigure}
%
\begin{subfigure}[b]{0.09\textwidth}
    \includegraphics[clip, trim=2.35cm 1.75cm 4.5cm 0cm,width=\textwidth]{img/convergence/892.pdf}
    \caption{892}
    \label{fig:convergence_892}
\end{subfigure}
%
\begin{subfigure}[b]{0.09\textwidth}
    \includegraphics[clip, trim=2.35cm 1.75cm 4.5cm 0cm,width=\textwidth]{img/convergence/893.pdf}
    \caption{893}
    \label{fig:convergence_893}
\end{subfigure}
%
\begin{subfigure}[b]{0.09\textwidth}
    \includegraphics[clip, trim=2.35cm 1.75cm 4.5cm 0cm,width=\textwidth]{img/convergence/895.pdf}
    \caption{895}
    \label{fig:convergence_895}
\end{subfigure}
%
\begin{subfigure}[b]{0.09\textwidth}
    \includegraphics[clip, trim=2.35cm 1.75cm 4.5cm 0cm,width=\textwidth]{img/convergence/898.pdf}
    \caption{898}
    \label{fig:convergence_898}
\end{subfigure}
%
\begin{subfigure}[b]{0.09\textwidth}
    \includegraphics[clip, trim=2.35cm 1.75cm 4.5cm 0cm,width=\textwidth]{img/convergence/900.pdf}
    \caption{900}
    \label{fig:convergence_900}
\end{subfigure}
%
\begin{subfigure}[b]{0.09\textwidth}
    \includegraphics[clip, trim=2.35cm 1.75cm 4.5cm 0cm,width=\textwidth]{img/convergence/904.pdf}
    \caption{904}
    \label{fig:convergence_904}
\end{subfigure}
%
\begin{subfigure}[b]{0.09\textwidth}
    \includegraphics[clip, trim=2.35cm 1.75cm 4.5cm 0cm,width=\textwidth]{img/convergence/905.pdf}
    \caption{905}
    \label{fig:convergence_905}
\end{subfigure}
%
\begin{subfigure}[b]{0.09\textwidth}
    \includegraphics[clip, trim=2.35cm 1.75cm 4.5cm 0cm,width=\textwidth]{img/convergence/908.pdf}
    \caption{908}
    \label{fig:convergence_908}
\end{subfigure}
%
\begin{subfigure}[b]{0.09\textwidth}
    \includegraphics[clip, trim=2.35cm 1.75cm 4.5cm 0cm,width=\textwidth]{img/convergence/933.pdf}
    \caption{933}
    \label{fig:convergence_933}
\end{subfigure}
%
\begin{subfigure}[b]{0.09\textwidth}
    \includegraphics[clip, trim=2.35cm 1.75cm 4.5cm 0cm,width=\textwidth]{img/convergence/937.pdf}
    \caption{937}
    \label{fig:convergence_937}
\end{subfigure}
%
\begin{subfigure}[b]{0.09\textwidth}
    \includegraphics[clip, trim=2.35cm 1.75cm 4.5cm 0cm,width=\textwidth]{img/convergence/956.pdf}
    \caption{956}
    \label{fig:convergence_956}
\end{subfigure}
%
\begin{subfigure}[b]{0.09\textwidth}
    \includegraphics[clip, trim=2.35cm 1.75cm 4.5cm 0cm,width=\textwidth]{img/convergence/960.pdf}
    \caption{960}
    \label{fig:convergence_960}
\end{subfigure}
%
\begin{subfigure}[b]{0.09\textwidth}
    \includegraphics[clip, trim=2.35cm 1.75cm 4.5cm 0cm,width=\textwidth]{img/convergence/966.pdf}
    \caption{966}
    \label{fig:convergence_966}
\end{subfigure}
%
\begin{subfigure}[b]{0.09\textwidth}
    \includegraphics[clip, trim=2.35cm 1.75cm 4.5cm 0cm,width=\textwidth]{img/convergence/969.pdf}
    \caption{969}
    \label{fig:convergence_969}
\end{subfigure}
%
\begin{subfigure}[b]{0.09\textwidth}
    \includegraphics[clip, trim=2.35cm 1.75cm 4.5cm 0cm,width=\textwidth]{img/convergence/979.pdf}
    \caption{979}
    \label{fig:convergence_979}
\end{subfigure}
%
\begin{subfigure}[b]{0.09\textwidth}
    \includegraphics[clip, trim=2.35cm 1.75cm 4.5cm 0cm,width=\textwidth]{img/convergence/981.pdf}
    \caption{981}
    \label{fig:convergence_981}
\end{subfigure}
%
\begin{subfigure}[b]{0.09\textwidth}
    \includegraphics[clip, trim=2.35cm 1.75cm 4.5cm 0cm,width=\textwidth]{img/convergence/989.pdf}
    \caption{989}
    \label{fig:convergence_989}
\end{subfigure}
%
\begin{subfigure}[b]{0.09\textwidth}
    \includegraphics[clip, trim=2.35cm 1.75cm 4.5cm 0cm,width=\textwidth]{img/convergence/991.pdf}
    \caption{991}
    \label{fig:convergence_991}
\end{subfigure}
%
\begin{subfigure}[b]{0.09\textwidth}
    \includegraphics[clip, trim=2.35cm 1.75cm 4.5cm 0cm,width=\textwidth]{img/convergence/993.pdf}
    \caption{993}
    \label{fig:convergence_993}
\end{subfigure}
%
\begin{subfigure}[b]{0.09\textwidth}
    \includegraphics[clip, trim=2.35cm 1.75cm 4.5cm 0cm,width=\textwidth]{img/convergence/998.pdf}
    \caption{998}
    \label{fig:convergence_998}
\end{subfigure}
%
\begin{subfigure}[b]{0.09\textwidth}
    \includegraphics[clip, trim=2.35cm 1.75cm 4.5cm 0cm,width=\textwidth]{img/convergence/1000.pdf}
    \caption{1000}
    \label{fig:convergence_1000}
\end{subfigure}
%
\begin{subfigure}[b]{0.09\textwidth}
    \includegraphics[clip, trim=2.35cm 1.75cm 4.5cm 0cm,width=\textwidth]{img/convergence/1002.pdf}
    \caption{1002}
    \label{fig:convergence_1002}
\end{subfigure}
%
\begin{subfigure}[b]{0.09\textwidth}
    \includegraphics[clip, trim=2.35cm 1.75cm 4.5cm 0cm,width=\textwidth]{img/convergence/1008.pdf}
    \caption{1008}
    \label{fig:convergence_1008}
\end{subfigure}
%
\begin{subfigure}[b]{0.09\textwidth}
    \includegraphics[clip, trim=2.35cm 1.75cm 4.5cm 0cm,width=\textwidth]{img/convergence/1017.pdf}
    \caption{1017}
    \label{fig:convergence_1017}
\end{subfigure}
%
\begin{subfigure}[b]{0.09\textwidth}
    \includegraphics[clip, trim=2.35cm 1.75cm 4.5cm 0cm,width=\textwidth]{img/convergence/1021.pdf}
    \caption{1021}
    \label{fig:convergence_1021}
\end{subfigure}
%
\begin{subfigure}[b]{0.09\textwidth}
    \includegraphics[clip, trim=2.35cm 1.75cm 4.5cm 0cm,width=\textwidth]{img/convergence/1024.pdf}
    \caption{1024}
    \label{fig:convergence_1024}
\end{subfigure}
%
\begin{subfigure}[b]{0.09\textwidth}
    \includegraphics[clip, trim=2.35cm 1.75cm 4.5cm 0cm,width=\textwidth]{img/convergence/1026.pdf}
    \caption{1026}
    \label{fig:convergence_1026}
\end{subfigure}
%
\begin{subfigure}[b]{0.09\textwidth}
    \includegraphics[clip, trim=2.35cm 1.75cm 4.5cm 0cm,width=\textwidth]{img/convergence/1030.pdf}
    \caption{1030}
    \label{fig:convergence_1030}
\end{subfigure}
%
\begin{subfigure}[b]{0.09\textwidth}
    \includegraphics[clip, trim=2.35cm 1.75cm 4.5cm 0cm,width=\textwidth]{img/convergence/1035.pdf}
    \caption{1035}
    \label{fig:convergence_1035}
\end{subfigure}
%
\begin{subfigure}[b]{0.09\textwidth}
    \includegraphics[clip, trim=2.35cm 1.75cm 4.5cm 0cm,width=\textwidth]{img/convergence/1038.pdf}
    \caption{1038}
    \label{fig:convergence_1038}
\end{subfigure}
%
\begin{subfigure}[b]{0.09\textwidth}
    \includegraphics[clip, trim=2.35cm 1.75cm 4.5cm 0cm,width=\textwidth]{img/convergence/1100.pdf}
    \caption{1100}
    \label{fig:convergence_1100}
\end{subfigure}
%
\begin{subfigure}[b]{0.09\textwidth}
    \includegraphics[clip, trim=2.35cm 1.75cm 4.5cm 0cm,width=\textwidth]{img/convergence/1103.pdf}
    \caption{1103}
    \label{fig:convergence_1103}
\end{subfigure}
%
\begin{subfigure}[b]{0.09\textwidth}
    \includegraphics[clip, trim=2.35cm 1.75cm 4.5cm 0cm,width=\textwidth]{img/convergence/1106.pdf}
    \caption{1106}
    \label{fig:convergence_1106}
\end{subfigure}
%
\begin{subfigure}[b]{0.09\textwidth}
    \includegraphics[clip, trim=2.35cm 1.75cm 4.5cm 0cm,width=\textwidth]{img/convergence/1109.pdf}
    \caption{1109}
    \label{fig:convergence_1109}
\end{subfigure}
%
\begin{subfigure}[b]{0.09\textwidth}
    \includegraphics[clip, trim=2.35cm 1.75cm 4.5cm 0cm,width=\textwidth]{img/convergence/1124.pdf}
    \caption{1124}
    \label{fig:convergence_1124}
\end{subfigure}
%
\begin{subfigure}[b]{0.09\textwidth}
    \includegraphics[clip, trim=2.35cm 1.75cm 4.5cm 0cm,width=\textwidth]{img/convergence/1126.pdf}
    \caption{1126}
    \label{fig:convergence_1126}
\end{subfigure}
%
\begin{subfigure}[b]{0.09\textwidth}
    \includegraphics[clip, trim=2.35cm 1.75cm 4.5cm 0cm,width=\textwidth]{img/convergence/1128.pdf}
    \caption{1128}
    \label{fig:convergence_1128}
\end{subfigure}
%
\begin{subfigure}[b]{0.09\textwidth}
    \includegraphics[clip, trim=2.35cm 1.75cm 4.5cm 0cm,width=\textwidth]{img/convergence/1130.pdf}
    \caption{1130}
    \label{fig:convergence_1130}
\end{subfigure}
%
\begin{subfigure}[b]{0.09\textwidth}
    \includegraphics[clip, trim=2.35cm 1.75cm 4.5cm 0cm,width=\textwidth]{img/convergence/1134.pdf}
    \caption{1134}
    \label{fig:convergence_1134}
\end{subfigure}
%
\begin{subfigure}[b]{0.09\textwidth}
    \includegraphics[clip, trim=2.35cm 1.75cm 4.5cm 0cm,width=\textwidth]{img/convergence/1135.pdf}
    \caption{1135}
    \label{fig:convergence_1135}
\end{subfigure}
%
\begin{subfigure}[b]{0.09\textwidth}
    \includegraphics[clip, trim=2.35cm 1.75cm 4.5cm 0cm,width=\textwidth]{img/convergence/1147.pdf}
    \caption{1147}
    \label{fig:convergence_1147}
\end{subfigure}
%
\begin{subfigure}[b]{0.09\textwidth}
    \includegraphics[clip, trim=2.35cm 1.75cm 4.5cm 0cm,width=\textwidth]{img/convergence/1148.pdf}
    \caption{1148}
    \label{fig:convergence_1148}
\end{subfigure}
%
\begin{subfigure}[b]{0.09\textwidth}
    \includegraphics[clip, trim=2.35cm 1.75cm 4.5cm 0cm,width=\textwidth]{img/convergence/1150.pdf}
    \caption{1150}
    \label{fig:convergence_1150}
\end{subfigure}
%
\begin{subfigure}[b]{0.09\textwidth}
    \includegraphics[clip, trim=2.35cm 1.75cm 4.5cm 0cm,width=\textwidth]{img/convergence/1151.pdf}
    \caption{1151}
    \label{fig:convergence_1151}
\end{subfigure}
%
\begin{subfigure}[b]{0.09\textwidth}
    \includegraphics[clip, trim=2.35cm 1.75cm 4.5cm 0cm,width=\textwidth]{img/convergence/1152.pdf}
    \caption{1152}
    \label{fig:convergence_1152}
\end{subfigure}
%
\begin{subfigure}[b]{0.09\textwidth}
    \includegraphics[clip, trim=2.35cm 1.75cm 4.5cm 0cm,width=\textwidth]{img/convergence/1154.pdf}
    \caption{1154}
    \label{fig:convergence_1154}
\end{subfigure}
%
\begin{subfigure}[b]{0.09\textwidth}
    \includegraphics[clip, trim=2.35cm 1.75cm 4.5cm 0cm,width=\textwidth]{img/convergence/1157.pdf}
    \caption{1157}
    \label{fig:convergence_1157}
\end{subfigure}
%
\begin{subfigure}[b]{0.09\textwidth}
    \includegraphics[clip, trim=2.35cm 1.75cm 4.5cm 0cm,width=\textwidth]{img/convergence/1160.pdf}
    \caption{1160}
    \label{fig:convergence_1160}
\end{subfigure}
%
\begin{subfigure}[b]{0.09\textwidth}
    \includegraphics[clip, trim=2.35cm 1.75cm 4.5cm 0cm,width=\textwidth]{img/convergence/1161.pdf}
    \caption{1161}
    \label{fig:convergence_1161}
\end{subfigure}
%
\begin{subfigure}[b]{0.09\textwidth}
    \includegraphics[clip, trim=2.35cm 1.75cm 4.5cm 0cm,width=\textwidth]{img/convergence/1162.pdf}
    \caption{1162}
    \label{fig:convergence_1162}
\end{subfigure}
%
\begin{subfigure}[b]{0.09\textwidth}
    \includegraphics[clip, trim=2.35cm 1.75cm 4.5cm 0cm,width=\textwidth]{img/convergence/1167.pdf}
    \caption{1167}
    \label{fig:convergence_1167}
\end{subfigure}
%
\begin{subfigure}[b]{0.09\textwidth}
    \includegraphics[clip, trim=2.35cm 1.75cm 4.5cm 0cm,width=\textwidth]{img/convergence/1168.pdf}
    \caption{1168}
    \label{fig:convergence_1168}
\end{subfigure}
%
\begin{subfigure}[b]{0.09\textwidth}
    \includegraphics[clip, trim=2.35cm 1.75cm 4.5cm 0cm,width=\textwidth]{img/convergence/1170.pdf}
    \caption{1170}
    \label{fig:convergence_1170}
\end{subfigure}
%
\begin{subfigure}[b]{0.09\textwidth}
    \includegraphics[clip, trim=2.35cm 1.75cm 4.5cm 0cm,width=\textwidth]{img/convergence/1174.pdf}
    \caption{1174}
    \label{fig:convergence_1174}
\end{subfigure}
%
\begin{subfigure}[b]{0.09\textwidth}
    \includegraphics[clip, trim=2.35cm 1.75cm 4.5cm 0cm,width=\textwidth]{img/convergence/1175.pdf}
    \caption{1175}
    \label{fig:convergence_1175}
\end{subfigure}
%
\begin{subfigure}[b]{0.09\textwidth}
    \includegraphics[clip, trim=2.35cm 1.75cm 4.5cm 0cm,width=\textwidth]{img/convergence/1182.pdf}
    \caption{1182}
    \label{fig:convergence_1182}
\end{subfigure}
%
\begin{subfigure}[b]{0.09\textwidth}
    \includegraphics[clip, trim=2.35cm 1.75cm 4.5cm 0cm,width=\textwidth]{img/convergence/1183.pdf}
    \caption{1183}
    \label{fig:convergence_1183}
\end{subfigure}
%
\begin{subfigure}[b]{0.09\textwidth}
    \includegraphics[clip, trim=2.35cm 1.75cm 4.5cm 0cm,width=\textwidth]{img/convergence/1189.pdf}
    \caption{1189}
    \label{fig:convergence_1189}
\end{subfigure}
%
\begin{subfigure}[b]{0.09\textwidth}
    \includegraphics[clip, trim=2.35cm 1.75cm 4.5cm 0cm,width=\textwidth]{img/convergence/1190.pdf}
    \caption{1190}
    \label{fig:convergence_1190}
\end{subfigure}
%
\begin{subfigure}[b]{0.09\textwidth}
    \includegraphics[clip, trim=2.35cm 1.75cm 4.5cm 0cm,width=\textwidth]{img/convergence/1195.pdf}
    \caption{1195}
    \label{fig:convergence_1195}
\end{subfigure}
%
\begin{subfigure}[b]{0.09\textwidth}
    \includegraphics[clip, trim=2.35cm 1.75cm 4.5cm 0cm,width=\textwidth]{img/convergence/1198.pdf}
    \caption{1198}
    \label{fig:convergence_1198}
\end{subfigure}
%
\begin{subfigure}[b]{0.09\textwidth}
    \includegraphics[clip, trim=2.35cm 1.75cm 4.5cm 0cm,width=\textwidth]{img/convergence/1200.pdf}
    \caption{1200}
    \label{fig:convergence_1200}
\end{subfigure}
%
\begin{subfigure}[b]{0.09\textwidth}
    \includegraphics[clip, trim=2.35cm 1.75cm 4.5cm 0cm,width=\textwidth]{img/convergence/1204.pdf}
    \caption{1204}
    \label{fig:convergence_1204}
\end{subfigure}
%
\begin{subfigure}[b]{0.09\textwidth}
    \includegraphics[clip, trim=2.35cm 1.75cm 4.5cm 0cm,width=\textwidth]{img/convergence/1206.pdf}
    \caption{1206}
    \label{fig:convergence_1206}
\end{subfigure}
%
\begin{subfigure}[b]{0.09\textwidth}
    \includegraphics[clip, trim=2.35cm 1.75cm 4.5cm 0cm,width=\textwidth]{img/convergence/1218.pdf}
    \caption{1218}
    \label{fig:convergence_1218}
\end{subfigure}
%
\begin{subfigure}[b]{0.09\textwidth}
    \includegraphics[clip, trim=2.35cm 1.75cm 4.5cm 0cm,width=\textwidth]{img/convergence/1221.pdf}
    \caption{1221}
    \label{fig:convergence_1221}
\end{subfigure}
%
\begin{subfigure}[b]{0.09\textwidth}
    \includegraphics[clip, trim=2.35cm 1.75cm 4.5cm 0cm,width=\textwidth]{img/convergence/1300.pdf}
    \caption{1300}
    \label{fig:convergence_1300}
\end{subfigure}
%
\begin{subfigure}[b]{0.09\textwidth}
    \includegraphics[clip, trim=2.35cm 1.75cm 4.5cm 0cm,width=\textwidth]{img/convergence/1324.pdf}
    \caption{1324}
    \label{fig:convergence_1324}
\end{subfigure}
%
\begin{subfigure}[b]{0.09\textwidth}
    \includegraphics[clip, trim=2.35cm 1.75cm 4.5cm 0cm,width=\textwidth]{img/convergence/1400.pdf}
    \caption{1400}
    \label{fig:convergence_1400}
\end{subfigure}
%
\begin{subfigure}[b]{0.09\textwidth}
    \includegraphics[clip, trim=2.35cm 1.75cm 4.5cm 0cm,width=\textwidth]{img/convergence/1500.pdf}
    \caption{1500}
    \label{fig:convergence_1500}
\end{subfigure}
%
\begin{subfigure}[b]{0.09\textwidth}
    \includegraphics[clip, trim=2.35cm 1.75cm 4.5cm 0cm,width=\textwidth]{img/convergence/1600.pdf}
    \caption{1600}
    \label{fig:convergence_1600}
\end{subfigure}
%
\begin{subfigure}[b]{0.09\textwidth}
    \includegraphics[clip, trim=2.35cm 1.75cm 4.5cm 0cm,width=\textwidth]{img/convergence/1700.pdf}
    \caption{1700}
    \label{fig:convergence_1700}
\end{subfigure}
%
\begin{subfigure}[b]{0.09\textwidth}
    \includegraphics[clip, trim=2.35cm 1.75cm 4.5cm 0cm,width=\textwidth]{img/convergence/1800.pdf}
    \caption{1800}
    \label{fig:convergence_1800}
\end{subfigure}
%
\begin{subfigure}[b]{0.09\textwidth}
    \includegraphics[clip, trim=2.35cm 1.75cm 4.5cm 0cm,width=\textwidth]{img/convergence/1900.pdf}
    \caption{1900}
    \label{fig:convergence_1900}
\end{subfigure}
%
\begin{subfigure}[b]{0.09\textwidth}
    \includegraphics[clip, trim=2.35cm 1.75cm 4.5cm 0cm,width=\textwidth]{img/convergence/2000.pdf}
    \caption{2000}
    \label{fig:convergence_2000}
\end{subfigure}
%
\begin{subfigure}[b]{0.09\textwidth}
    \includegraphics[clip, trim=2.35cm 1.75cm 4.5cm 0cm,width=\textwidth]{img/convergence/2100.pdf}
    \caption{2100}
    \label{fig:convergence_2100}
\end{subfigure}
%
\begin{subfigure}[b]{0.09\textwidth}
    \includegraphics[clip, trim=2.35cm 1.75cm 4.5cm 0cm,width=\textwidth]{img/convergence/2200.pdf}
    \caption{2200}
    \label{fig:convergence_2200}
\end{subfigure}
%
\begin{subfigure}[b]{0.09\textwidth}
    \includegraphics[clip, trim=2.35cm 1.75cm 4.5cm 0cm,width=\textwidth]{img/convergence/2300.pdf}
    \caption{2300}
    \label{fig:convergence_2300}
\end{subfigure}
%
\begin{subfigure}[b]{0.09\textwidth}
    \includegraphics[clip, trim=2.35cm 1.75cm 4.5cm 0cm,width=\textwidth]{img/convergence/2400.pdf}
    \caption{2400}
    \label{fig:convergence_2400}
\end{subfigure}
%
\begin{subfigure}[b]{0.09\textwidth}
    \includegraphics[clip, trim=2.35cm 1.75cm 4.5cm 0cm,width=\textwidth]{img/convergence/2500.pdf}
    \caption{2500}
    \label{fig:convergence_2500}
\end{subfigure}
%
\begin{subfigure}[b]{0.09\textwidth}
    \includegraphics[clip, trim=2.35cm 1.75cm 4.5cm 0cm,width=\textwidth]{img/convergence/2600.pdf}
    \caption{2600}
    \label{fig:convergence_2600}
\end{subfigure}
%
\begin{subfigure}[b]{0.09\textwidth}
    \includegraphics[clip, trim=2.35cm 1.75cm 4.5cm 0cm,width=\textwidth]{img/convergence/2700.pdf}
    \caption{2700}
    \label{fig:convergence_2700}
\end{subfigure}
%
\begin{subfigure}[b]{0.09\textwidth}
    \includegraphics[clip, trim=2.35cm 1.75cm 4.5cm 0cm,width=\textwidth]{img/convergence/2800.pdf}
    \caption{2800}
    \label{fig:convergence_2800}
\end{subfigure}
%
\begin{subfigure}[b]{0.09\textwidth}
    \includegraphics[clip, trim=2.35cm 1.75cm 4.5cm 0cm,width=\textwidth]{img/convergence/2900.pdf}
    \caption{2900}
    \label{fig:convergence_2900}
\end{subfigure}
%
\begin{subfigure}[b]{0.09\textwidth}
    \includegraphics[clip, trim=2.35cm 1.75cm 4.5cm 0cm,width=\textwidth]{img/convergence/3000.pdf}
    \caption{3000}
    \label{fig:convergence_3000}
\end{subfigure}
%
\begin{subfigure}[b]{0.09\textwidth}
    \includegraphics[clip, trim=2.35cm 1.75cm 4.5cm 0cm,width=\textwidth]{img/convergence/3100.pdf}
    \caption{3100}
    \label{fig:convergence_3100}
\end{subfigure}
%
\begin{subfigure}[b]{0.09\textwidth}
    \includegraphics[clip, trim=2.35cm 1.75cm 4.5cm 0cm,width=\textwidth]{img/convergence/3200.pdf}
    \caption{3200}
    \label{fig:convergence_3200}
\end{subfigure}
%
\begin{subfigure}[b]{0.09\textwidth}
    \includegraphics[clip, trim=2.35cm 1.75cm 4.5cm 0cm,width=\textwidth]{img/convergence/3300.pdf}
    \caption{3300}
    \label{fig:convergence_3300}
\end{subfigure}
%
\begin{subfigure}[b]{0.09\textwidth}
    \includegraphics[clip, trim=2.35cm 1.75cm 4.5cm 0cm,width=\textwidth]{img/convergence/3400.pdf}
    \caption{3400}
    \label{fig:convergence_3400}
\end{subfigure}
%
\begin{subfigure}[b]{0.09\textwidth}
    \includegraphics[clip, trim=2.35cm 1.75cm 4.5cm 0cm,width=\textwidth]{img/convergence/3500.pdf}
    \caption{3500}
    \label{fig:convergence_3500}
\end{subfigure}
%
\begin{subfigure}[b]{0.09\textwidth}
    \includegraphics[clip, trim=2.35cm 1.75cm 4.5cm 0cm,width=\textwidth]{img/convergence/3600.pdf}
    \caption{3600}
    \label{fig:convergence_3600}
\end{subfigure}
%
\begin{subfigure}[b]{0.09\textwidth}
    \includegraphics[clip, trim=2.35cm 1.75cm 4.5cm 0cm,width=\textwidth]{img/convergence/3700.pdf}
    \caption{3700}
    \label{fig:convergence_3700}
\end{subfigure}
%
\begin{subfigure}[b]{0.09\textwidth}
    \includegraphics[clip, trim=2.35cm 1.75cm 4.5cm 0cm,width=\textwidth]{img/convergence/3800.pdf}
    \caption{3800}
    \label{fig:convergence_3800}
\end{subfigure}
%
\begin{subfigure}[b]{0.09\textwidth}
    \includegraphics[clip, trim=2.35cm 1.75cm 4.5cm 0cm,width=\textwidth]{img/convergence/3900.pdf}
    \caption{3900}
    \label{fig:convergence_3900}
\end{subfigure}
%
\begin{subfigure}[b]{0.09\textwidth}
    \includegraphics[clip, trim=2.35cm 1.75cm 4.5cm 0cm,width=\textwidth]{img/convergence/4000.pdf}
    \caption{4000}
    \label{fig:convergence_4000}
\end{subfigure}
%
\begin{subfigure}[b]{0.09\textwidth}
    \includegraphics[clip, trim=2.35cm 1.75cm 4.5cm 0cm,width=\textwidth]{img/convergence/4100.pdf}
    \caption{4100}
    \label{fig:convergence_4100}
\end{subfigure}
%
\begin{subfigure}[b]{0.09\textwidth}
    \includegraphics[clip, trim=2.35cm 1.75cm 4.5cm 0cm,width=\textwidth]{img/convergence/4200.pdf}
    \caption{4200}
    \label{fig:convergence_4200}
\end{subfigure}
%
\begin{subfigure}[b]{0.09\textwidth}
    \includegraphics[clip, trim=2.35cm 1.75cm 4.5cm 0cm,width=\textwidth]{img/convergence/4300.pdf}
    \caption{4300}
    \label{fig:convergence_4300}
\end{subfigure}
%
\begin{subfigure}[b]{0.09\textwidth}
    \includegraphics[clip, trim=2.35cm 1.75cm 4.5cm 0cm,width=\textwidth]{img/convergence/4400.pdf}
    \caption{4400}
    \label{fig:convergence_4400}
\end{subfigure}
%
\begin{subfigure}[b]{0.09\textwidth}
    \includegraphics[clip, trim=2.35cm 1.75cm 4.5cm 0cm,width=\textwidth]{img/convergence/4500.pdf}
    \caption{4500}
    \label{fig:convergence_4500}
\end{subfigure}
%
\begin{subfigure}[b]{0.09\textwidth}
    \includegraphics[clip, trim=2.35cm 1.75cm 4.5cm 0cm,width=\textwidth]{img/convergence/4600.pdf}
    \caption{4600}
    \label{fig:convergence_4600}
\end{subfigure}
%
\begin{subfigure}[b]{0.09\textwidth}
    \includegraphics[clip, trim=2.35cm 1.75cm 4.5cm 0cm,width=\textwidth]{img/convergence/4700.pdf}
    \caption{4700}
    \label{fig:convergence_4700}
\end{subfigure}
%
\begin{subfigure}[b]{0.09\textwidth}
    \includegraphics[clip, trim=2.35cm 1.75cm 4.5cm 0cm,width=\textwidth]{img/convergence/4800.pdf}
    \caption{4800}
    \label{fig:convergence_4800}
\end{subfigure}
%
\begin{subfigure}[b]{0.09\textwidth}
    \includegraphics[clip, trim=2.35cm 1.75cm 4.5cm 0cm,width=\textwidth]{img/convergence/4900.pdf}
    \caption{4900}
    \label{fig:convergence_4900}
\end{subfigure}
%
\begin{subfigure}[b]{0.09\textwidth}
    \includegraphics[clip, trim=2.35cm 1.75cm 4.5cm 0cm,width=\textwidth]{img/convergence/5000.pdf}
    \caption{5000}
    \label{fig:convergence_5000}
\end{subfigure}
%
\begin{subfigure}[b]{0.09\textwidth}
    \includegraphics[clip, trim=2.35cm 1.75cm 4.5cm 0cm,width=\textwidth]{img/convergence/5100.pdf}
    \caption{5100}
    \label{fig:convergence_5100}
\end{subfigure}
%
\begin{subfigure}[b]{0.09\textwidth}
    \includegraphics[clip, trim=2.35cm 1.75cm 4.5cm 0cm,width=\textwidth]{img/convergence/5200.pdf}
    \caption{5200}
    \label{fig:convergence_5200}
\end{subfigure}
%
\begin{subfigure}[b]{0.09\textwidth}
    \includegraphics[clip, trim=2.35cm 1.75cm 4.5cm 0cm,width=\textwidth]{img/convergence/5300.pdf}
    \caption{5300}
    \label{fig:convergence_5300}
\end{subfigure}
%
\begin{subfigure}[b]{0.09\textwidth}
    \includegraphics[clip, trim=2.35cm 1.75cm 4.5cm 0cm,width=\textwidth]{img/convergence/5400.pdf}
    \caption{5400}
    \label{fig:convergence_5400}
\end{subfigure}
%
\begin{subfigure}[b]{0.09\textwidth}
    \includegraphics[clip, trim=2.35cm 1.75cm 4.5cm 0cm,width=\textwidth]{img/convergence/5500.pdf}
    \caption{5500}
    \label{fig:convergence_5500}
\end{subfigure}
%
\begin{subfigure}[b]{0.09\textwidth}
    \includegraphics[clip, trim=2.35cm 1.75cm 4.5cm 0cm,width=\textwidth]{img/convergence/5600.pdf}
    \caption{5600}
    \label{fig:convergence_5600}
\end{subfigure}
%
\begin{subfigure}[b]{0.09\textwidth}
    \includegraphics[clip, trim=2.35cm 1.75cm 4.5cm 0cm,width=\textwidth]{img/convergence/5700.pdf}
    \caption{5700}
    \label{fig:convergence_5700}
\end{subfigure}
%
\begin{subfigure}[b]{0.09\textwidth}
    \includegraphics[clip, trim=2.35cm 1.75cm 4.5cm 0cm,width=\textwidth]{img/convergence/5800.pdf}
    \caption{5800}
    \label{fig:convergence_5800}
\end{subfigure}
%
\begin{subfigure}[b]{0.09\textwidth}
    \includegraphics[clip, trim=2.35cm 1.75cm 4.5cm 0cm,width=\textwidth]{img/convergence/5900.pdf}
    \caption{5900}
    \label{fig:convergence_5900}
\end{subfigure}
%
\begin{subfigure}[b]{0.09\textwidth}
    \includegraphics[clip, trim=2.35cm 1.75cm 4.5cm 0cm,width=\textwidth]{img/convergence/6000.pdf}
    \caption{6000}
    \label{fig:convergence_6000}
\end{subfigure}
%
\begin{subfigure}[b]{0.09\textwidth}
    \includegraphics[clip, trim=2.35cm 1.75cm 4.5cm 0cm,width=\textwidth]{img/convergence/6100.pdf}
    \caption{6100}
    \label{fig:convergence_6100}
\end{subfigure}
%
\begin{subfigure}[b]{0.09\textwidth}
    \includegraphics[clip, trim=2.35cm 1.75cm 4.5cm 0cm,width=\textwidth]{img/convergence/6200.pdf}
    \caption{6200}
    \label{fig:convergence_6200}
\end{subfigure}
%
\begin{subfigure}[b]{0.09\textwidth}
    \includegraphics[clip, trim=2.35cm 1.75cm 4.5cm 0cm,width=\textwidth]{img/convergence/6205.pdf}
    \caption{6205}
    \label{fig:convergence_6205}
\end{subfigure}
%
\begin{subfigure}[b]{0.09\textwidth}
    \includegraphics[clip, trim=2.35cm 1.75cm 4.5cm 0cm,width=\textwidth]{img/convergence/6206.pdf}
    \caption{6206}
    \label{fig:convergence_6206}
\end{subfigure}
%
\begin{subfigure}[b]{0.09\textwidth}
    \includegraphics[clip, trim=2.35cm 1.75cm 4.5cm 0cm,width=\textwidth]{img/convergence/6207.pdf}
    \caption{6207}
    \label{fig:convergence_6207}
\end{subfigure}
%
\begin{subfigure}[b]{0.09\textwidth}
    \includegraphics[clip, trim=2.35cm 1.75cm 4.5cm 0cm,width=\textwidth]{img/convergence/6208.pdf}
    \caption{6208}
    \label{fig:convergence_6208}
\end{subfigure}
%
\begin{subfigure}[b]{0.09\textwidth}
    \includegraphics[clip, trim=2.35cm 1.75cm 4.5cm 0cm,width=\textwidth]{img/convergence/6210.pdf}
    \caption{6210}
    \label{fig:convergence_6210}
\end{subfigure}
%
\begin{subfigure}[b]{0.09\textwidth}
    \includegraphics[clip, trim=2.35cm 1.75cm 4.5cm 0cm,width=\textwidth]{img/convergence/6219.pdf}
    \caption{6219}
    \label{fig:convergence_6219}
\end{subfigure}
%
\begin{subfigure}[b]{0.09\textwidth}
    \includegraphics[clip, trim=2.35cm 1.75cm 4.5cm 0cm,width=\textwidth]{img/convergence/6223.pdf}
    \caption{6223}
    \label{fig:convergence_6223}
\end{subfigure}
%
\begin{subfigure}[b]{0.09\textwidth}
    \includegraphics[clip, trim=2.35cm 1.75cm 4.5cm 0cm,width=\textwidth]{img/convergence/6224.pdf}
    \caption{6224}
    \label{fig:convergence_6224}
\end{subfigure}
%
\begin{subfigure}[b]{0.09\textwidth}
    \includegraphics[clip, trim=2.35cm 1.75cm 4.5cm 0cm,width=\textwidth]{img/convergence/6226.pdf}
    \caption{6226}
    \label{fig:convergence_6226}
\end{subfigure}
%
\begin{subfigure}[b]{0.09\textwidth}
    \includegraphics[clip, trim=2.35cm 1.75cm 4.5cm 0cm,width=\textwidth]{img/convergence/6227.pdf}
    \caption{6227}
    \label{fig:convergence_6227}
\end{subfigure}
%
\begin{subfigure}[b]{0.09\textwidth}
    \includegraphics[clip, trim=2.35cm 1.75cm 4.5cm 0cm,width=\textwidth]{img/convergence/6228.pdf}
    \caption{6228}
    \label{fig:convergence_6228}
\end{subfigure}
%
\begin{subfigure}[b]{0.09\textwidth}
    \includegraphics[clip, trim=2.35cm 1.75cm 4.5cm 0cm,width=\textwidth]{img/convergence/6229.pdf}
    \caption{6229}
    \label{fig:convergence_6229}
\end{subfigure}
%
\begin{subfigure}[b]{0.09\textwidth}
    \includegraphics[clip, trim=2.35cm 1.75cm 4.5cm 0cm,width=\textwidth]{img/convergence/6242.pdf}
    \caption{6242}
    \label{fig:convergence_6242}
\end{subfigure}
%
\begin{subfigure}[b]{0.09\textwidth}
    \includegraphics[clip, trim=2.35cm 1.75cm 4.5cm 0cm,width=\textwidth]{img/convergence/6300.pdf}
    \caption{6300}
    \label{fig:convergence_6300}
\end{subfigure}
%
\begin{subfigure}[b]{0.09\textwidth}
    \includegraphics[clip, trim=2.35cm 1.75cm 4.5cm 0cm,width=\textwidth]{img/convergence/6400.pdf}
    \caption{6400}
    \label{fig:convergence_6400}
\end{subfigure}
%
\begin{subfigure}[b]{0.09\textwidth}
    \includegraphics[clip, trim=2.35cm 1.75cm 4.5cm 0cm,width=\textwidth]{img/convergence/6500.pdf}
    \caption{6500}
    \label{fig:convergence_6500}
\end{subfigure}
%
\begin{subfigure}[b]{0.09\textwidth}
    \includegraphics[clip, trim=2.35cm 1.75cm 4.5cm 0cm,width=\textwidth]{img/convergence/6600.pdf}
    \caption{6600}
    \label{fig:convergence_6600}
\end{subfigure}
%
\begin{subfigure}[b]{0.09\textwidth}
    \includegraphics[clip, trim=2.35cm 1.75cm 4.5cm 0cm,width=\textwidth]{img/convergence/6700.pdf}
    \caption{6700}
    \label{fig:convergence_6700}
\end{subfigure}
%
\begin{subfigure}[b]{0.09\textwidth}
    \includegraphics[clip, trim=2.35cm 1.75cm 4.5cm 0cm,width=\textwidth]{img/convergence/6738.pdf}
    \caption{6738}
    \label{fig:convergence_6738}
\end{subfigure}
%
\begin{subfigure}[b]{0.09\textwidth}
    \includegraphics[clip, trim=2.35cm 1.75cm 4.5cm 0cm,width=\textwidth]{img/convergence/6742.pdf}
    \caption{6742}
    \label{fig:convergence_6742}
\end{subfigure}
%
\begin{subfigure}[b]{0.09\textwidth}
    \includegraphics[clip, trim=2.35cm 1.75cm 4.5cm 0cm,width=\textwidth]{img/convergence/6743.pdf}
    \caption{6743}
    \label{fig:convergence_6743}
\end{subfigure}
%
\begin{subfigure}[b]{0.09\textwidth}
    \includegraphics[clip, trim=2.35cm 1.75cm 4.5cm 0cm,width=\textwidth]{img/convergence/6745.pdf}
    \caption{6745}
    \label{fig:convergence_6745}
\end{subfigure}
%
\begin{subfigure}[b]{0.09\textwidth}
    \includegraphics[clip, trim=2.35cm 1.75cm 4.5cm 0cm,width=\textwidth]{img/convergence/6746.pdf}
    \caption{6746}
    \label{fig:convergence_6746}
\end{subfigure}
%
\begin{subfigure}[b]{0.09\textwidth}
    \includegraphics[clip, trim=2.35cm 1.75cm 4.5cm 0cm,width=\textwidth]{img/convergence/6748.pdf}
    \caption{6748}
    \label{fig:convergence_6748}
\end{subfigure}
%
\begin{subfigure}[b]{0.09\textwidth}
    \includegraphics[clip, trim=2.35cm 1.75cm 4.5cm 0cm,width=\textwidth]{img/convergence/6749.pdf}
    \caption{6749}
    \label{fig:convergence_6749}
\end{subfigure}
%
\begin{subfigure}[b]{0.09\textwidth}
    \includegraphics[clip, trim=2.35cm 1.75cm 4.5cm 0cm,width=\textwidth]{img/convergence/6750.pdf}
    \caption{6750}
    \label{fig:convergence_6750}
\end{subfigure}
%
\begin{subfigure}[b]{0.09\textwidth}
    \includegraphics[clip, trim=2.35cm 1.75cm 4.5cm 0cm,width=\textwidth]{img/convergence/6751.pdf}
    \caption{6751}
    \label{fig:convergence_6751}
\end{subfigure}
%
\begin{subfigure}[b]{0.09\textwidth}
    \includegraphics[clip, trim=2.35cm 1.75cm 4.5cm 0cm,width=\textwidth]{img/convergence/6774.pdf}
    \caption{6774}
    \label{fig:convergence_6774}
\end{subfigure}
%
\begin{subfigure}[b]{0.09\textwidth}
    \includegraphics[clip, trim=2.35cm 1.75cm 4.5cm 0cm,width=\textwidth]{img/convergence/6775.pdf}
    \caption{6775}
    \label{fig:convergence_6775}
\end{subfigure}
%
\begin{subfigure}[b]{0.09\textwidth}
    \includegraphics[clip, trim=2.35cm 1.75cm 4.5cm 0cm,width=\textwidth]{img/convergence/6777.pdf}
    \caption{6777}
    \label{fig:convergence_6777}
\end{subfigure}
%
\begin{subfigure}[b]{0.09\textwidth}
    \includegraphics[clip, trim=2.35cm 1.75cm 4.5cm 0cm,width=\textwidth]{img/convergence/6778.pdf}
    \caption{6778}
    \label{fig:convergence_6778}
\end{subfigure}
%
\begin{subfigure}[b]{0.09\textwidth}
    \includegraphics[clip, trim=2.35cm 1.75cm 4.5cm 0cm,width=\textwidth]{img/convergence/6780.pdf}
    \caption{6780}
    \label{fig:convergence_6780}
\end{subfigure}
%
\begin{subfigure}[b]{0.09\textwidth}
    \includegraphics[clip, trim=2.35cm 1.75cm 4.5cm 0cm,width=\textwidth]{img/convergence/6781.pdf}
    \caption{6781}
    \label{fig:convergence_6781}
\end{subfigure}
%
\begin{subfigure}[b]{0.09\textwidth}
    \includegraphics[clip, trim=2.35cm 1.75cm 4.5cm 0cm,width=\textwidth]{img/convergence/6782.pdf}
    \caption{6782}
    \label{fig:convergence_6782}
\end{subfigure}
%
\begin{subfigure}[b]{0.09\textwidth}
    \includegraphics[clip, trim=2.35cm 1.75cm 4.5cm 0cm,width=\textwidth]{img/convergence/6794.pdf}
    \caption{6794}
    \label{fig:convergence_6794}
\end{subfigure}
%
\begin{subfigure}[b]{0.09\textwidth}
    \includegraphics[clip, trim=2.35cm 1.75cm 4.5cm 0cm,width=\textwidth]{img/convergence/6795.pdf}
    \caption{6795}
    \label{fig:convergence_6795}
\end{subfigure}
%
\begin{subfigure}[b]{0.09\textwidth}
    \includegraphics[clip, trim=2.35cm 1.75cm 4.5cm 0cm,width=\textwidth]{img/convergence/6796.pdf}
    \caption{6796}
    \label{fig:convergence_6796}
\end{subfigure}
%
\begin{subfigure}[b]{0.09\textwidth}
    \includegraphics[clip, trim=2.35cm 1.75cm 4.5cm 0cm,width=\textwidth]{img/convergence/6797.pdf}
    \caption{6797}
    \label{fig:convergence_6797}
\end{subfigure}
%
\begin{subfigure}[b]{0.09\textwidth}
    \includegraphics[clip, trim=2.35cm 1.75cm 4.5cm 0cm,width=\textwidth]{img/convergence/6798.pdf}
    \caption{6798}
    \label{fig:convergence_6798}
\end{subfigure}
%
\begin{subfigure}[b]{0.09\textwidth}
    \includegraphics[clip, trim=2.35cm 1.75cm 4.5cm 0cm,width=\textwidth]{img/convergence/6799.pdf}
    \caption{6799}
    \label{fig:convergence_6799}
\end{subfigure}
%
\begin{subfigure}[b]{0.09\textwidth}
    \includegraphics[clip, trim=2.35cm 1.75cm 4.5cm 0cm,width=\textwidth]{img/convergence/6800.pdf}
    \caption{6800}
    \label{fig:convergence_6800}
\end{subfigure}
%
\begin{subfigure}[b]{0.09\textwidth}
    \includegraphics[clip, trim=2.35cm 1.75cm 4.5cm 0cm,width=\textwidth]{img/convergence/6811.pdf}
    \caption{6811}
    \label{fig:convergence_6811}
\end{subfigure}
%
\begin{subfigure}[b]{0.09\textwidth}
    \includegraphics[clip, trim=2.35cm 1.75cm 4.5cm 0cm,width=\textwidth]{img/convergence/6812.pdf}
    \caption{6812}
    \label{fig:convergence_6812}
\end{subfigure}
%
\begin{subfigure}[b]{0.09\textwidth}
    \includegraphics[clip, trim=2.35cm 1.75cm 4.5cm 0cm,width=\textwidth]{img/convergence/6813.pdf}
    \caption{6813}
    \label{fig:convergence_6813}
\end{subfigure}
%
\begin{subfigure}[b]{0.09\textwidth}
    \includegraphics[clip, trim=2.35cm 1.75cm 4.5cm 0cm,width=\textwidth]{img/convergence/6814.pdf}
    \caption{6814}
    \label{fig:convergence_6814}
\end{subfigure}
%
\begin{subfigure}[b]{0.09\textwidth}
    \includegraphics[clip, trim=2.35cm 1.75cm 4.5cm 0cm,width=\textwidth]{img/convergence/6842.pdf}
    \caption{6842}
    \label{fig:convergence_6842}
\end{subfigure}
%
\begin{subfigure}[b]{0.09\textwidth}
    \includegraphics[clip, trim=2.35cm 1.75cm 4.5cm 0cm,width=\textwidth]{img/convergence/6843.pdf}
    \caption{6843}
    \label{fig:convergence_6843}
\end{subfigure}
%
\begin{subfigure}[b]{0.09\textwidth}
    \includegraphics[clip, trim=2.35cm 1.75cm 4.5cm 0cm,width=\textwidth]{img/convergence/6844.pdf}
    \caption{6844}
    \label{fig:convergence_6844}
\end{subfigure}
%
\begin{subfigure}[b]{0.09\textwidth}
    \includegraphics[clip, trim=2.35cm 1.75cm 4.5cm 0cm,width=\textwidth]{img/convergence/6845.pdf}
    \caption{6845}
    \label{fig:convergence_6845}
\end{subfigure}
%
\begin{subfigure}[b]{0.09\textwidth}
    \includegraphics[clip, trim=2.35cm 1.75cm 4.5cm 0cm,width=\textwidth]{img/convergence/6846.pdf}
    \caption{6846}
    \label{fig:convergence_6846}
\end{subfigure}
%
\begin{subfigure}[b]{0.09\textwidth}
    \includegraphics[clip, trim=2.35cm 1.75cm 4.5cm 0cm,width=\textwidth]{img/convergence/6847.pdf}
    \caption{6847}
    \label{fig:convergence_6847}
\end{subfigure}
%
\begin{subfigure}[b]{0.09\textwidth}
    \includegraphics[clip, trim=2.35cm 1.75cm 4.5cm 0cm,width=\textwidth]{img/convergence/6848.pdf}
    \caption{6848}
    \label{fig:convergence_6848}
\end{subfigure}
%
\begin{subfigure}[b]{0.09\textwidth}
    \includegraphics[clip, trim=2.35cm 1.75cm 4.5cm 0cm,width=\textwidth]{img/convergence/6849.pdf}
    \caption{6849}
    \label{fig:convergence_6849}
\end{subfigure}
%
\begin{subfigure}[b]{0.09\textwidth}
    \includegraphics[clip, trim=2.35cm 1.75cm 4.5cm 0cm,width=\textwidth]{img/convergence/6853.pdf}
    \caption{6853}
    \label{fig:convergence_6853}
\end{subfigure}
%
\begin{subfigure}[b]{0.09\textwidth}
    \includegraphics[clip, trim=2.35cm 1.75cm 4.5cm 0cm,width=\textwidth]{img/convergence/6857.pdf}
    \caption{6857}
    \label{fig:convergence_6857}
\end{subfigure}
%
\begin{subfigure}[b]{0.09\textwidth}
    \includegraphics[clip, trim=2.35cm 1.75cm 4.5cm 0cm,width=\textwidth]{img/convergence/6861.pdf}
    \caption{6861}
    \label{fig:convergence_6861}
\end{subfigure}
%
\begin{subfigure}[b]{0.09\textwidth}
    \includegraphics[clip, trim=2.35cm 1.75cm 4.5cm 0cm,width=\textwidth]{img/convergence/6877.pdf}
    \caption{6877}
    \label{fig:convergence_6877}
\end{subfigure}
%
\begin{subfigure}[b]{0.09\textwidth}
    \includegraphics[clip, trim=2.35cm 1.75cm 4.5cm 0cm,width=\textwidth]{img/convergence/6900.pdf}
    \caption{6900}
    \label{fig:convergence_6900}
\end{subfigure}
%
\begin{subfigure}[b]{0.09\textwidth}
    \includegraphics[clip, trim=2.35cm 1.75cm 4.5cm 0cm,width=\textwidth]{img/convergence/7000.pdf}
    \caption{7000}
    \label{fig:convergence_7000}
\end{subfigure}
%
\begin{subfigure}[b]{0.09\textwidth}
    \includegraphics[clip, trim=2.35cm 1.75cm 4.5cm 0cm,width=\textwidth]{img/convergence/7019.pdf}
    \caption{7019}
    \label{fig:convergence_7019}
\end{subfigure}
%
\begin{subfigure}[b]{0.09\textwidth}
    \includegraphics[clip, trim=2.35cm 1.75cm 4.5cm 0cm,width=\textwidth]{img/convergence/7020.pdf}
    \caption{7020}
    \label{fig:convergence_7020}
\end{subfigure}
%
\begin{subfigure}[b]{0.09\textwidth}
    \includegraphics[clip, trim=2.35cm 1.75cm 4.5cm 0cm,width=\textwidth]{img/convergence/7021.pdf}
    \caption{7021}
    \label{fig:convergence_7021}
\end{subfigure}
%
\begin{subfigure}[b]{0.09\textwidth}
    \includegraphics[clip, trim=2.35cm 1.75cm 4.5cm 0cm,width=\textwidth]{img/convergence/7025.pdf}
    \caption{7025}
    \label{fig:convergence_7025}
\end{subfigure}
%
\begin{subfigure}[b]{0.09\textwidth}
    \includegraphics[clip, trim=2.35cm 1.75cm 4.5cm 0cm,width=\textwidth]{img/convergence/7026.pdf}
    \caption{7026}
    \label{fig:convergence_7026}
\end{subfigure}
%
\begin{subfigure}[b]{0.09\textwidth}
    \includegraphics[clip, trim=2.35cm 1.75cm 4.5cm 0cm,width=\textwidth]{img/convergence/7027.pdf}
    \caption{7027}
    \label{fig:convergence_7027}
\end{subfigure}
%
\begin{subfigure}[b]{0.09\textwidth}
    \includegraphics[clip, trim=2.35cm 1.75cm 4.5cm 0cm,width=\textwidth]{img/convergence/7029.pdf}
    \caption{7029}
    \label{fig:convergence_7029}
\end{subfigure}
%
\begin{subfigure}[b]{0.09\textwidth}
    \includegraphics[clip, trim=2.35cm 1.75cm 4.5cm 0cm,width=\textwidth]{img/convergence/7030.pdf}
    \caption{7030}
    \label{fig:convergence_7030}
\end{subfigure}
%
\begin{subfigure}[b]{0.09\textwidth}
    \includegraphics[clip, trim=2.35cm 1.75cm 4.5cm 0cm,width=\textwidth]{img/convergence/7031.pdf}
    \caption{7031}
    \label{fig:convergence_7031}
\end{subfigure}
%
\begin{subfigure}[b]{0.09\textwidth}
    \includegraphics[clip, trim=2.35cm 1.75cm 4.5cm 0cm,width=\textwidth]{img/convergence/7032.pdf}
    \caption{7032}
    \label{fig:convergence_7032}
\end{subfigure}
%
\begin{subfigure}[b]{0.09\textwidth}
    \includegraphics[clip, trim=2.35cm 1.75cm 4.5cm 0cm,width=\textwidth]{img/convergence/7033.pdf}
    \caption{7033}
    \label{fig:convergence_7033}
\end{subfigure}
%
\begin{subfigure}[b]{0.09\textwidth}
    \includegraphics[clip, trim=2.35cm 1.75cm 4.5cm 0cm,width=\textwidth]{img/convergence/7036.pdf}
    \caption{7036}
    \label{fig:convergence_7036}
\end{subfigure}
%
\begin{subfigure}[b]{0.09\textwidth}
    \includegraphics[clip, trim=2.35cm 1.75cm 4.5cm 0cm,width=\textwidth]{img/convergence/7037.pdf}
    \caption{7037}
    \label{fig:convergence_7037}
\end{subfigure}
%
\begin{subfigure}[b]{0.09\textwidth}
    \includegraphics[clip, trim=2.35cm 1.75cm 4.5cm 0cm,width=\textwidth]{img/convergence/7039.pdf}
    \caption{7039}
    \label{fig:convergence_7039}
\end{subfigure}
%
\begin{subfigure}[b]{0.09\textwidth}
    \includegraphics[clip, trim=2.35cm 1.75cm 4.5cm 0cm,width=\textwidth]{img/convergence/7041.pdf}
    \caption{7041}
    \label{fig:convergence_7041}
\end{subfigure}
%
\begin{subfigure}[b]{0.09\textwidth}
    \includegraphics[clip, trim=2.35cm 1.75cm 4.5cm 0cm,width=\textwidth]{img/convergence/7044.pdf}
    \caption{7044}
    \label{fig:convergence_7044}
\end{subfigure}
%
\begin{subfigure}[b]{0.09\textwidth}
    \includegraphics[clip, trim=2.35cm 1.75cm 4.5cm 0cm,width=\textwidth]{img/convergence/7045.pdf}
    \caption{7045}
    \label{fig:convergence_7045}
\end{subfigure}
%
\begin{subfigure}[b]{0.09\textwidth}
    \includegraphics[clip, trim=2.35cm 1.75cm 4.5cm 0cm,width=\textwidth]{img/convergence/7048.pdf}
    \caption{7048}
    \label{fig:convergence_7048}
\end{subfigure}
%
\begin{subfigure}[b]{0.09\textwidth}
    \includegraphics[clip, trim=2.35cm 1.75cm 4.5cm 0cm,width=\textwidth]{img/convergence/7050.pdf}
    \caption{7050}
    \label{fig:convergence_7050}
\end{subfigure}
%
\begin{subfigure}[b]{0.09\textwidth}
    \includegraphics[clip, trim=2.35cm 1.75cm 4.5cm 0cm,width=\textwidth]{img/convergence/7052.pdf}
    \caption{7052}
    \label{fig:convergence_7052}
\end{subfigure}
%
\begin{subfigure}[b]{0.09\textwidth}
    \includegraphics[clip, trim=2.35cm 1.75cm 4.5cm 0cm,width=\textwidth]{img/convergence/7053.pdf}
    \caption{7053}
    \label{fig:convergence_7053}
\end{subfigure}
%
\begin{subfigure}[b]{0.09\textwidth}
    \includegraphics[clip, trim=2.35cm 1.75cm 4.5cm 0cm,width=\textwidth]{img/convergence/7054.pdf}
    \caption{7054}
    \label{fig:convergence_7054}
\end{subfigure}
%
\begin{subfigure}[b]{0.09\textwidth}
    \includegraphics[clip, trim=2.35cm 1.75cm 4.5cm 0cm,width=\textwidth]{img/convergence/7055.pdf}
    \caption{7055}
    \label{fig:convergence_7055}
\end{subfigure}
%
\begin{subfigure}[b]{0.09\textwidth}
    \includegraphics[clip, trim=2.35cm 1.75cm 4.5cm 0cm,width=\textwidth]{img/convergence/7056.pdf}
    \caption{7056}
    \label{fig:convergence_7056}
\end{subfigure}
%
\begin{subfigure}[b]{0.09\textwidth}
    \includegraphics[clip, trim=2.35cm 1.75cm 4.5cm 0cm,width=\textwidth]{img/convergence/7057.pdf}
    \caption{7057}
    \label{fig:convergence_7057}
\end{subfigure}
%
\begin{subfigure}[b]{0.09\textwidth}
    \includegraphics[clip, trim=2.35cm 1.75cm 4.5cm 0cm,width=\textwidth]{img/convergence/7058.pdf}
    \caption{7058}
    \label{fig:convergence_7058}
\end{subfigure}
%
\begin{subfigure}[b]{0.09\textwidth}
    \includegraphics[clip, trim=2.35cm 1.75cm 4.5cm 0cm,width=\textwidth]{img/convergence/7063.pdf}
    \caption{7063}
    \label{fig:convergence_7063}
\end{subfigure}
%
\begin{subfigure}[b]{0.09\textwidth}
    \includegraphics[clip, trim=2.35cm 1.75cm 4.5cm 0cm,width=\textwidth]{img/convergence/7066.pdf}
    \caption{7066}
    \label{fig:convergence_7066}
\end{subfigure}
%
\begin{subfigure}[b]{0.09\textwidth}
    \includegraphics[clip, trim=2.35cm 1.75cm 4.5cm 0cm,width=\textwidth]{img/convergence/7067.pdf}
    \caption{7067}
    \label{fig:convergence_7067}
\end{subfigure}
%
\begin{subfigure}[b]{0.09\textwidth}
    \includegraphics[clip, trim=2.35cm 1.75cm 4.5cm 0cm,width=\textwidth]{img/convergence/7069.pdf}
    \caption{7069}
    \label{fig:convergence_7069}
\end{subfigure}
%
\begin{subfigure}[b]{0.09\textwidth}
    \includegraphics[clip, trim=2.35cm 1.75cm 4.5cm 0cm,width=\textwidth]{img/convergence/7070.pdf}
    \caption{7070}
    \label{fig:convergence_7070}
\end{subfigure}
%
\begin{subfigure}[b]{0.09\textwidth}
    \includegraphics[clip, trim=2.35cm 1.75cm 4.5cm 0cm,width=\textwidth]{img/convergence/7071.pdf}
    \caption{7071}
    \label{fig:convergence_7071}
\end{subfigure}
%
\begin{subfigure}[b]{0.09\textwidth}
    \includegraphics[clip, trim=2.35cm 1.75cm 4.5cm 0cm,width=\textwidth]{img/convergence/7072.pdf}
    \caption{7072}
    \label{fig:convergence_7072}
\end{subfigure}
%
\begin{subfigure}[b]{0.09\textwidth}
    \includegraphics[clip, trim=2.35cm 1.75cm 4.5cm 0cm,width=\textwidth]{img/convergence/7073.pdf}
    \caption{7073}
    \label{fig:convergence_7073}
\end{subfigure}
%
\begin{subfigure}[b]{0.09\textwidth}
    \includegraphics[clip, trim=2.35cm 1.75cm 4.5cm 0cm,width=\textwidth]{img/convergence/7074.pdf}
    \caption{7074}
    \label{fig:convergence_7074}
\end{subfigure}
%
\begin{subfigure}[b]{0.09\textwidth}
    \includegraphics[clip, trim=2.35cm 1.75cm 4.5cm 0cm,width=\textwidth]{img/convergence/7075.pdf}
    \caption{7075}
    \label{fig:convergence_7075}
\end{subfigure}
%
\begin{subfigure}[b]{0.09\textwidth}
    \includegraphics[clip, trim=2.35cm 1.75cm 4.5cm 0cm,width=\textwidth]{img/convergence/7077.pdf}
    \caption{7077}
    \label{fig:convergence_7077}
\end{subfigure}
%
\begin{subfigure}[b]{0.09\textwidth}
    \includegraphics[clip, trim=2.35cm 1.75cm 4.5cm 0cm,width=\textwidth]{img/convergence/7079.pdf}
    \caption{7079}
    \label{fig:convergence_7079}
\end{subfigure}
%
\begin{subfigure}[b]{0.09\textwidth}
    \includegraphics[clip, trim=2.35cm 1.75cm 4.5cm 0cm,width=\textwidth]{img/convergence/7080.pdf}
    \caption{7080}
    \label{fig:convergence_7080}
\end{subfigure}
%
\begin{subfigure}[b]{0.09\textwidth}
    \includegraphics[clip, trim=2.35cm 1.75cm 4.5cm 0cm,width=\textwidth]{img/convergence/7081.pdf}
    \caption{7081}
    \label{fig:convergence_7081}
\end{subfigure}
%
\begin{subfigure}[b]{0.09\textwidth}
    \includegraphics[clip, trim=2.35cm 1.75cm 4.5cm 0cm,width=\textwidth]{img/convergence/7082.pdf}
    \caption{7082}
    \label{fig:convergence_7082}
\end{subfigure}
%
\begin{subfigure}[b]{0.09\textwidth}
    \includegraphics[clip, trim=2.35cm 1.75cm 4.5cm 0cm,width=\textwidth]{img/convergence/7083.pdf}
    \caption{7083}
    \label{fig:convergence_7083}
\end{subfigure}
%
\begin{subfigure}[b]{0.09\textwidth}
    \includegraphics[clip, trim=2.35cm 1.75cm 4.5cm 0cm,width=\textwidth]{img/convergence/7084.pdf}
    \caption{7084}
    \label{fig:convergence_7084}
\end{subfigure}
%
\begin{subfigure}[b]{0.09\textwidth}
    \includegraphics[clip, trim=2.35cm 1.75cm 4.5cm 0cm,width=\textwidth]{img/convergence/7085.pdf}
    \caption{7085}
    \label{fig:convergence_7085}
\end{subfigure}
%
\begin{subfigure}[b]{0.09\textwidth}
    \includegraphics[clip, trim=2.35cm 1.75cm 4.5cm 0cm,width=\textwidth]{img/convergence/7086.pdf}
    \caption{7086}
    \label{fig:convergence_7086}
\end{subfigure}
%
\begin{subfigure}[b]{0.09\textwidth}
    \includegraphics[clip, trim=2.35cm 1.75cm 4.5cm 0cm,width=\textwidth]{img/convergence/7088.pdf}
    \caption{7088}
    \label{fig:convergence_7088}
\end{subfigure}
%
\begin{subfigure}[b]{0.09\textwidth}
    \includegraphics[clip, trim=2.35cm 1.75cm 4.5cm 0cm,width=\textwidth]{img/convergence/7089.pdf}
    \caption{7089}
    \label{fig:convergence_7089}
\end{subfigure}
%
\begin{subfigure}[b]{0.09\textwidth}
    \includegraphics[clip, trim=2.35cm 1.75cm 4.5cm 0cm,width=\textwidth]{img/convergence/7094.pdf}
    \caption{7094}
    \label{fig:convergence_7094}
\end{subfigure}
%
\begin{subfigure}[b]{0.09\textwidth}
    \includegraphics[clip, trim=2.35cm 1.75cm 4.5cm 0cm,width=\textwidth]{img/convergence/7095.pdf}
    \caption{7095}
    \label{fig:convergence_7095}
\end{subfigure}
%
\begin{subfigure}[b]{0.09\textwidth}
    \includegraphics[clip, trim=2.35cm 1.75cm 4.5cm 0cm,width=\textwidth]{img/convergence/7099.pdf}
    \caption{7099}
    \label{fig:convergence_7099}
\end{subfigure}
%
\begin{subfigure}[b]{0.09\textwidth}
    \includegraphics[clip, trim=2.35cm 1.75cm 4.5cm 0cm,width=\textwidth]{img/convergence/7100.pdf}
    \caption{7100}
    \label{fig:convergence_7100}
\end{subfigure}
%
\begin{subfigure}[b]{0.09\textwidth}
    \includegraphics[clip, trim=2.35cm 1.75cm 4.5cm 0cm,width=\textwidth]{img/convergence/7104.pdf}
    \caption{7104}
    \label{fig:convergence_7104}
\end{subfigure}
%
\begin{subfigure}[b]{0.09\textwidth}
    \includegraphics[clip, trim=2.35cm 1.75cm 4.5cm 0cm,width=\textwidth]{img/convergence/7114.pdf}
    \caption{7114}
    \label{fig:convergence_7114}
\end{subfigure}
%
\begin{subfigure}[b]{0.09\textwidth}
    \includegraphics[clip, trim=2.35cm 1.75cm 4.5cm 0cm,width=\textwidth]{img/convergence/7115.pdf}
    \caption{7115}
    \label{fig:convergence_7115}
\end{subfigure}
%
\begin{subfigure}[b]{0.09\textwidth}
    \includegraphics[clip, trim=2.35cm 1.75cm 4.5cm 0cm,width=\textwidth]{img/convergence/7116.pdf}
    \caption{7116}
    \label{fig:convergence_7116}
\end{subfigure}
%
\begin{subfigure}[b]{0.09\textwidth}
    \includegraphics[clip, trim=2.35cm 1.75cm 4.5cm 0cm,width=\textwidth]{img/convergence/7121.pdf}
    \caption{7121}
    \label{fig:convergence_7121}
\end{subfigure}
%
\begin{subfigure}[b]{0.09\textwidth}
    \includegraphics[clip, trim=2.35cm 1.75cm 4.5cm 0cm,width=\textwidth]{img/convergence/7123.pdf}
    \caption{7123}
    \label{fig:convergence_7123}
\end{subfigure}
%
\begin{subfigure}[b]{0.09\textwidth}
    \includegraphics[clip, trim=2.35cm 1.75cm 4.5cm 0cm,width=\textwidth]{img/convergence/7124.pdf}
    \caption{7124}
    \label{fig:convergence_7124}
\end{subfigure}
%
\begin{subfigure}[b]{0.09\textwidth}
    \includegraphics[clip, trim=2.35cm 1.75cm 4.5cm 0cm,width=\textwidth]{img/convergence/7127.pdf}
    \caption{7127}
    \label{fig:convergence_7127}
\end{subfigure}
%
\begin{subfigure}[b]{0.09\textwidth}
    \includegraphics[clip, trim=2.35cm 1.75cm 4.5cm 0cm,width=\textwidth]{img/convergence/7128.pdf}
    \caption{7128}
    \label{fig:convergence_7128}
\end{subfigure}
%
\begin{subfigure}[b]{0.09\textwidth}
    \includegraphics[clip, trim=2.35cm 1.75cm 4.5cm 0cm,width=\textwidth]{img/convergence/7131.pdf}
    \caption{7131}
    \label{fig:convergence_7131}
\end{subfigure}
%
\begin{subfigure}[b]{0.09\textwidth}
    \includegraphics[clip, trim=2.35cm 1.75cm 4.5cm 0cm,width=\textwidth]{img/convergence/7134.pdf}
    \caption{7134}
    \label{fig:convergence_7134}
\end{subfigure}
%
\begin{subfigure}[b]{0.09\textwidth}
    \includegraphics[clip, trim=2.35cm 1.75cm 4.5cm 0cm,width=\textwidth]{img/convergence/7135.pdf}
    \caption{7135}
    \label{fig:convergence_7135}
\end{subfigure}
%
\begin{subfigure}[b]{0.09\textwidth}
    \includegraphics[clip, trim=2.35cm 1.75cm 4.5cm 0cm,width=\textwidth]{img/convergence/7138.pdf}
    \caption{7138}
    \label{fig:convergence_7138}
\end{subfigure}
%
\begin{subfigure}[b]{0.09\textwidth}
    \includegraphics[clip, trim=2.35cm 1.75cm 4.5cm 0cm,width=\textwidth]{img/convergence/7142.pdf}
    \caption{7142}
    \label{fig:convergence_7142}
\end{subfigure}
%
\begin{subfigure}[b]{0.09\textwidth}
    \includegraphics[clip, trim=2.35cm 1.75cm 4.5cm 0cm,width=\textwidth]{img/convergence/7147.pdf}
    \caption{7147}
    \label{fig:convergence_7147}
\end{subfigure}
%
\begin{subfigure}[b]{0.09\textwidth}
    \includegraphics[clip, trim=2.35cm 1.75cm 4.5cm 0cm,width=\textwidth]{img/convergence/7150.pdf}
    \caption{7150}
    \label{fig:convergence_7150}
\end{subfigure}
%
\begin{subfigure}[b]{0.09\textwidth}
    \includegraphics[clip, trim=2.35cm 1.75cm 4.5cm 0cm,width=\textwidth]{img/convergence/7200.pdf}
    \caption{7200}
    \label{fig:convergence_7200}
\end{subfigure}
%
\begin{subfigure}[b]{0.09\textwidth}
    \includegraphics[clip, trim=2.35cm 1.75cm 4.5cm 0cm,width=\textwidth]{img/convergence/7300.pdf}
    \caption{7300}
    \label{fig:convergence_7300}
\end{subfigure}
%
\begin{subfigure}[b]{0.09\textwidth}
    \includegraphics[clip, trim=2.35cm 1.75cm 4.5cm 0cm,width=\textwidth]{img/convergence/7400.pdf}
    \caption{7400}
    \label{fig:convergence_7400}
\end{subfigure}
%
\begin{subfigure}[b]{0.09\textwidth}
    \includegraphics[clip, trim=2.35cm 1.75cm 4.5cm 0cm,width=\textwidth]{img/convergence/7500.pdf}
    \caption{7500}
    \label{fig:convergence_7500}
\end{subfigure}
%
\begin{subfigure}[b]{0.09\textwidth}
    \includegraphics[clip, trim=2.35cm 1.75cm 4.5cm 0cm,width=\textwidth]{img/convergence/7600.pdf}
    \caption{7600}
    \label{fig:convergence_7600}
\end{subfigure}
%
\begin{subfigure}[b]{0.09\textwidth}
    \includegraphics[clip, trim=2.35cm 1.75cm 4.5cm 0cm,width=\textwidth]{img/convergence/7700.pdf}
    \caption{7700}
    \label{fig:convergence_7700}
\end{subfigure}
%
\begin{subfigure}[b]{0.09\textwidth}
    \includegraphics[clip, trim=2.35cm 1.75cm 4.5cm 0cm,width=\textwidth]{img/convergence/7800.pdf}
    \caption{7800}
    \label{fig:convergence_7800}
\end{subfigure}
%
\begin{subfigure}[b]{0.09\textwidth}
    \includegraphics[clip, trim=2.35cm 1.75cm 4.5cm 0cm,width=\textwidth]{img/convergence/7900.pdf}
    \caption{7900}
    \label{fig:convergence_7900}
\end{subfigure}
%
\begin{subfigure}[b]{0.09\textwidth}
    \includegraphics[clip, trim=2.35cm 1.75cm 4.5cm 0cm,width=\textwidth]{img/convergence/8000.pdf}
    \caption{8000}
    \label{fig:convergence_8000}
\end{subfigure}
%
\begin{subfigure}[b]{0.09\textwidth}
    \includegraphics[clip, trim=2.35cm 1.75cm 4.5cm 0cm,width=\textwidth]{img/convergence/8100.pdf}
    \caption{8100}
    \label{fig:convergence_8100}
\end{subfigure}
%
\begin{subfigure}[b]{0.09\textwidth}
    \includegraphics[clip, trim=2.35cm 1.75cm 4.5cm 0cm,width=\textwidth]{img/convergence/8122.pdf}
    \caption{8122}
    \label{fig:convergence_8122}
\end{subfigure}
%
\begin{subfigure}[b]{0.09\textwidth}
    \includegraphics[clip, trim=2.35cm 1.75cm 4.5cm 0cm,width=\textwidth]{img/convergence/8123.pdf}
    \caption{8123}
    \label{fig:convergence_8123}
\end{subfigure}
%
\begin{subfigure}[b]{0.09\textwidth}
    \includegraphics[clip, trim=2.35cm 1.75cm 4.5cm 0cm,width=\textwidth]{img/convergence/8126.pdf}
    \caption{8126}
    \label{fig:convergence_8126}
\end{subfigure}
%
\begin{subfigure}[b]{0.09\textwidth}
    \includegraphics[clip, trim=2.35cm 1.75cm 4.5cm 0cm,width=\textwidth]{img/convergence/8128.pdf}
    \caption{8128}
    \label{fig:convergence_8128}
\end{subfigure}
%
\begin{subfigure}[b]{0.09\textwidth}
    \includegraphics[clip, trim=2.35cm 1.75cm 4.5cm 0cm,width=\textwidth]{img/convergence/8129.pdf}
    \caption{8129}
    \label{fig:convergence_8129}
\end{subfigure}
%
\begin{subfigure}[b]{0.09\textwidth}
    \includegraphics[clip, trim=2.35cm 1.75cm 4.5cm 0cm,width=\textwidth]{img/convergence/8130.pdf}
    \caption{8130}
    \label{fig:convergence_8130}
\end{subfigure}
%
\begin{subfigure}[b]{0.09\textwidth}
    \includegraphics[clip, trim=2.35cm 1.75cm 4.5cm 0cm,width=\textwidth]{img/convergence/8131.pdf}
    \caption{8131}
    \label{fig:convergence_8131}
\end{subfigure}
%
\begin{subfigure}[b]{0.09\textwidth}
    \includegraphics[clip, trim=2.35cm 1.75cm 4.5cm 0cm,width=\textwidth]{img/convergence/8132.pdf}
    \caption{8132}
    \label{fig:convergence_8132}
\end{subfigure}
%
\begin{subfigure}[b]{0.09\textwidth}
    \includegraphics[clip, trim=2.35cm 1.75cm 4.5cm 0cm,width=\textwidth]{img/convergence/8133.pdf}
    \caption{8133}
    \label{fig:convergence_8133}
\end{subfigure}
%
\begin{subfigure}[b]{0.09\textwidth}
    \includegraphics[clip, trim=2.35cm 1.75cm 4.5cm 0cm,width=\textwidth]{img/convergence/8134.pdf}
    \caption{8134}
    \label{fig:convergence_8134}
\end{subfigure}
%
\begin{subfigure}[b]{0.09\textwidth}
    \includegraphics[clip, trim=2.35cm 1.75cm 4.5cm 0cm,width=\textwidth]{img/convergence/8135.pdf}
    \caption{8135}
    \label{fig:convergence_8135}
\end{subfigure}
%
\begin{subfigure}[b]{0.09\textwidth}
    \includegraphics[clip, trim=2.35cm 1.75cm 4.5cm 0cm,width=\textwidth]{img/convergence/8137.pdf}
    \caption{8137}
    \label{fig:convergence_8137}
\end{subfigure}
%
\begin{subfigure}[b]{0.09\textwidth}
    \includegraphics[clip, trim=2.35cm 1.75cm 4.5cm 0cm,width=\textwidth]{img/convergence/8138.pdf}
    \caption{8138}
    \label{fig:convergence_8138}
\end{subfigure}
%
\begin{subfigure}[b]{0.09\textwidth}
    \includegraphics[clip, trim=2.35cm 1.75cm 4.5cm 0cm,width=\textwidth]{img/convergence/8139.pdf}
    \caption{8139}
    \label{fig:convergence_8139}
\end{subfigure}
%
\begin{subfigure}[b]{0.09\textwidth}
    \includegraphics[clip, trim=2.35cm 1.75cm 4.5cm 0cm,width=\textwidth]{img/convergence/8140.pdf}
    \caption{8140}
    \label{fig:convergence_8140}
\end{subfigure}
%
\begin{subfigure}[b]{0.09\textwidth}
    \includegraphics[clip, trim=2.35cm 1.75cm 4.5cm 0cm,width=\textwidth]{img/convergence/8141.pdf}
    \caption{8141}
    \label{fig:convergence_8141}
\end{subfigure}
%
\begin{subfigure}[b]{0.09\textwidth}
    \includegraphics[clip, trim=2.35cm 1.75cm 4.5cm 0cm,width=\textwidth]{img/convergence/8142.pdf}
    \caption{8142}
    \label{fig:convergence_8142}
\end{subfigure}
%
\begin{subfigure}[b]{0.09\textwidth}
    \includegraphics[clip, trim=2.35cm 1.75cm 4.5cm 0cm,width=\textwidth]{img/convergence/8144.pdf}
    \caption{8144}
    \label{fig:convergence_8144}
\end{subfigure}
%
\begin{subfigure}[b]{0.09\textwidth}
    \includegraphics[clip, trim=2.35cm 1.75cm 4.5cm 0cm,width=\textwidth]{img/convergence/8145.pdf}
    \caption{8145}
    \label{fig:convergence_8145}
\end{subfigure}
%
\begin{subfigure}[b]{0.09\textwidth}
    \includegraphics[clip, trim=2.35cm 1.75cm 4.5cm 0cm,width=\textwidth]{img/convergence/8146.pdf}
    \caption{8146}
    \label{fig:convergence_8146}
\end{subfigure}
%
\begin{subfigure}[b]{0.09\textwidth}
    \includegraphics[clip, trim=2.35cm 1.75cm 4.5cm 0cm,width=\textwidth]{img/convergence/8147.pdf}
    \caption{8147}
    \label{fig:convergence_8147}
\end{subfigure}
%
\begin{subfigure}[b]{0.09\textwidth}
    \includegraphics[clip, trim=2.35cm 1.75cm 4.5cm 0cm,width=\textwidth]{img/convergence/8148.pdf}
    \caption{8148}
    \label{fig:convergence_8148}
\end{subfigure}
%
\begin{subfigure}[b]{0.09\textwidth}
    \includegraphics[clip, trim=2.35cm 1.75cm 4.5cm 0cm,width=\textwidth]{img/convergence/8150.pdf}
    \caption{8150}
    \label{fig:convergence_8150}
\end{subfigure}
%
\begin{subfigure}[b]{0.09\textwidth}
    \includegraphics[clip, trim=2.35cm 1.75cm 4.5cm 0cm,width=\textwidth]{img/convergence/8151.pdf}
    \caption{8151}
    \label{fig:convergence_8151}
\end{subfigure}
%
\begin{subfigure}[b]{0.09\textwidth}
    \includegraphics[clip, trim=2.35cm 1.75cm 4.5cm 0cm,width=\textwidth]{img/convergence/8154.pdf}
    \caption{8154}
    \label{fig:convergence_8154}
\end{subfigure}
%
\begin{subfigure}[b]{0.09\textwidth}
    \includegraphics[clip, trim=2.35cm 1.75cm 4.5cm 0cm,width=\textwidth]{img/convergence/8159.pdf}
    \caption{8159}
    \label{fig:convergence_8159}
\end{subfigure}
%
\begin{subfigure}[b]{0.09\textwidth}
    \includegraphics[clip, trim=2.35cm 1.75cm 4.5cm 0cm,width=\textwidth]{img/convergence/8163.pdf}
    \caption{8163}
    \label{fig:convergence_8163}
\end{subfigure}
%
\begin{subfigure}[b]{0.09\textwidth}
    \includegraphics[clip, trim=2.35cm 1.75cm 4.5cm 0cm,width=\textwidth]{img/convergence/8171.pdf}
    \caption{8171}
    \label{fig:convergence_8171}
\end{subfigure}
%
\begin{subfigure}[b]{0.09\textwidth}
    \includegraphics[clip, trim=2.35cm 1.75cm 4.5cm 0cm,width=\textwidth]{img/convergence/8173.pdf}
    \caption{8173}
    \label{fig:convergence_8173}
\end{subfigure}
%
\begin{subfigure}[b]{0.09\textwidth}
    \includegraphics[clip, trim=2.35cm 1.75cm 4.5cm 0cm,width=\textwidth]{img/convergence/8176.pdf}
    \caption{8176}
    \label{fig:convergence_8176}
\end{subfigure}
%
\begin{subfigure}[b]{0.09\textwidth}
    \includegraphics[clip, trim=2.35cm 1.75cm 4.5cm 0cm,width=\textwidth]{img/convergence/8180.pdf}
    \caption{8180}
    \label{fig:convergence_8180}
\end{subfigure}
%
\begin{subfigure}[b]{0.09\textwidth}
    \includegraphics[clip, trim=2.35cm 1.75cm 4.5cm 0cm,width=\textwidth]{img/convergence/8183.pdf}
    \caption{8183}
    \label{fig:convergence_8183}
\end{subfigure}
%
\begin{subfigure}[b]{0.09\textwidth}
    \includegraphics[clip, trim=2.35cm 1.75cm 4.5cm 0cm,width=\textwidth]{img/convergence/8185.pdf}
    \caption{8185}
    \label{fig:convergence_8185}
\end{subfigure}
%
\begin{subfigure}[b]{0.09\textwidth}
    \includegraphics[clip, trim=2.35cm 1.75cm 4.5cm 0cm,width=\textwidth]{img/convergence/8188.pdf}
    \caption{8188}
    \label{fig:convergence_8188}
\end{subfigure}
%
\begin{subfigure}[b]{0.09\textwidth}
    \includegraphics[clip, trim=2.35cm 1.75cm 4.5cm 0cm,width=\textwidth]{img/convergence/8193.pdf}
    \caption{8193}
    \label{fig:convergence_8193}
\end{subfigure}
%
\begin{subfigure}[b]{0.09\textwidth}
    \includegraphics[clip, trim=2.35cm 1.75cm 4.5cm 0cm,width=\textwidth]{img/convergence/8200.pdf}
    \caption{8200}
    \label{fig:convergence_8200}
\end{subfigure}
%
\begin{subfigure}[b]{0.09\textwidth}
    \includegraphics[clip, trim=2.35cm 1.75cm 4.5cm 0cm,width=\textwidth]{img/convergence/8204.pdf}
    \caption{8204}
    \label{fig:convergence_8204}
\end{subfigure}
%
\begin{subfigure}[b]{0.09\textwidth}
    \includegraphics[clip, trim=2.35cm 1.75cm 4.5cm 0cm,width=\textwidth]{img/convergence/8205.pdf}
    \caption{8205}
    \label{fig:convergence_8205}
\end{subfigure}
%
\begin{subfigure}[b]{0.09\textwidth}
    \includegraphics[clip, trim=2.35cm 1.75cm 4.5cm 0cm,width=\textwidth]{img/convergence/8213.pdf}
    \caption{8213}
    \label{fig:convergence_8213}
\end{subfigure}
%
\begin{subfigure}[b]{0.09\textwidth}
    \includegraphics[clip, trim=2.35cm 1.75cm 4.5cm 0cm,width=\textwidth]{img/convergence/8215.pdf}
    \caption{8215}
    \label{fig:convergence_8215}
\end{subfigure}
%
\begin{subfigure}[b]{0.09\textwidth}
    \includegraphics[clip, trim=2.35cm 1.75cm 4.5cm 0cm,width=\textwidth]{img/convergence/8246.pdf}
    \caption{8246}
    \label{fig:convergence_8246}
\end{subfigure}
%
\begin{subfigure}[b]{0.09\textwidth}
    \includegraphics[clip, trim=2.35cm 1.75cm 4.5cm 0cm,width=\textwidth]{img/convergence/8249.pdf}
    \caption{8249}
    \label{fig:convergence_8249}
\end{subfigure}
%
\begin{subfigure}[b]{0.09\textwidth}
    \includegraphics[clip, trim=2.35cm 1.75cm 4.5cm 0cm,width=\textwidth]{img/convergence/8250.pdf}
    \caption{8250}
    \label{fig:convergence_8250}
\end{subfigure}
%
\begin{subfigure}[b]{0.09\textwidth}
    \includegraphics[clip, trim=2.35cm 1.75cm 4.5cm 0cm,width=\textwidth]{img/convergence/8251.pdf}
    \caption{8251}
    \label{fig:convergence_8251}
\end{subfigure}
%
\begin{subfigure}[b]{0.09\textwidth}
    \includegraphics[clip, trim=2.35cm 1.75cm 4.5cm 0cm,width=\textwidth]{img/convergence/8253.pdf}
    \caption{8253}
    \label{fig:convergence_8253}
\end{subfigure}
%
\begin{subfigure}[b]{0.09\textwidth}
    \includegraphics[clip, trim=2.35cm 1.75cm 4.5cm 0cm,width=\textwidth]{img/convergence/8256.pdf}
    \caption{8256}
    \label{fig:convergence_8256}
\end{subfigure}
%
\begin{subfigure}[b]{0.09\textwidth}
    \includegraphics[clip, trim=2.35cm 1.75cm 4.5cm 0cm,width=\textwidth]{img/convergence/8259.pdf}
    \caption{8259}
    \label{fig:convergence_8259}
\end{subfigure}
%
\begin{subfigure}[b]{0.09\textwidth}
    \includegraphics[clip, trim=2.35cm 1.75cm 4.5cm 0cm,width=\textwidth]{img/convergence/8260.pdf}
    \caption{8260}
    \label{fig:convergence_8260}
\end{subfigure}
%
\begin{subfigure}[b]{0.09\textwidth}
    \includegraphics[clip, trim=2.35cm 1.75cm 4.5cm 0cm,width=\textwidth]{img/convergence/8263.pdf}
    \caption{8263}
    \label{fig:convergence_8263}
\end{subfigure}
%
\begin{subfigure}[b]{0.09\textwidth}
    \includegraphics[clip, trim=2.35cm 1.75cm 4.5cm 0cm,width=\textwidth]{img/convergence/8264.pdf}
    \caption{8264}
    \label{fig:convergence_8264}
\end{subfigure}
%
\begin{subfigure}[b]{0.09\textwidth}
    \includegraphics[clip, trim=2.35cm 1.75cm 4.5cm 0cm,width=\textwidth]{img/convergence/8266.pdf}
    \caption{8266}
    \label{fig:convergence_8266}
\end{subfigure}
%
\begin{subfigure}[b]{0.09\textwidth}
    \includegraphics[clip, trim=2.35cm 1.75cm 4.5cm 0cm,width=\textwidth]{img/convergence/8268.pdf}
    \caption{8268}
    \label{fig:convergence_8268}
\end{subfigure}
%
\begin{subfigure}[b]{0.09\textwidth}
    \includegraphics[clip, trim=2.35cm 1.75cm 4.5cm 0cm,width=\textwidth]{img/convergence/8269.pdf}
    \caption{8269}
    \label{fig:convergence_8269}
\end{subfigure}
%
\begin{subfigure}[b]{0.09\textwidth}
    \includegraphics[clip, trim=2.35cm 1.75cm 4.5cm 0cm,width=\textwidth]{img/convergence/8278.pdf}
    \caption{8278}
    \label{fig:convergence_8278}
\end{subfigure}
%
\begin{subfigure}[b]{0.09\textwidth}
    \includegraphics[clip, trim=2.35cm 1.75cm 4.5cm 0cm,width=\textwidth]{img/convergence/8279.pdf}
    \caption{8279}
    \label{fig:convergence_8279}
\end{subfigure}
%
\begin{subfigure}[b]{0.09\textwidth}
    \includegraphics[clip, trim=2.35cm 1.75cm 4.5cm 0cm,width=\textwidth]{img/convergence/8281.pdf}
    \caption{8281}
    \label{fig:convergence_8281}
\end{subfigure}
%
\begin{subfigure}[b]{0.09\textwidth}
    \includegraphics[clip, trim=2.35cm 1.75cm 4.5cm 0cm,width=\textwidth]{img/convergence/8282.pdf}
    \caption{8282}
    \label{fig:convergence_8282}
\end{subfigure}
%
\begin{subfigure}[b]{0.09\textwidth}
    \includegraphics[clip, trim=2.35cm 1.75cm 4.5cm 0cm,width=\textwidth]{img/convergence/8283.pdf}
    \caption{8283}
    \label{fig:convergence_8283}
\end{subfigure}
%
\begin{subfigure}[b]{0.09\textwidth}
    \includegraphics[clip, trim=2.35cm 1.75cm 4.5cm 0cm,width=\textwidth]{img/convergence/8291.pdf}
    \caption{8291}
    \label{fig:convergence_8291}
\end{subfigure}
%
\begin{subfigure}[b]{0.09\textwidth}
    \includegraphics[clip, trim=2.35cm 1.75cm 4.5cm 0cm,width=\textwidth]{img/convergence/8292.pdf}
    \caption{8292}
    \label{fig:convergence_8292}
\end{subfigure}
%
\begin{subfigure}[b]{0.09\textwidth}
    \includegraphics[clip, trim=2.35cm 1.75cm 4.5cm 0cm,width=\textwidth]{img/convergence/8295.pdf}
    \caption{8295}
    \label{fig:convergence_8295}
\end{subfigure}
%
\begin{subfigure}[b]{0.09\textwidth}
    \includegraphics[clip, trim=2.35cm 1.75cm 4.5cm 0cm,width=\textwidth]{img/convergence/8300.pdf}
    \caption{8300}
    \label{fig:convergence_8300}
\end{subfigure}
%
\begin{subfigure}[b]{0.09\textwidth}
    \includegraphics[clip, trim=2.35cm 1.75cm 4.5cm 0cm,width=\textwidth]{img/convergence/8301.pdf}
    \caption{8301}
    \label{fig:convergence_8301}
\end{subfigure}
%
\begin{subfigure}[b]{0.09\textwidth}
    \includegraphics[clip, trim=2.35cm 1.75cm 4.5cm 0cm,width=\textwidth]{img/convergence/8305.pdf}
    \caption{8305}
    \label{fig:convergence_8305}
\end{subfigure}
%
\begin{subfigure}[b]{0.09\textwidth}
    \includegraphics[clip, trim=2.35cm 1.75cm 4.5cm 0cm,width=\textwidth]{img/convergence/8308.pdf}
    \caption{8308}
    \label{fig:convergence_8308}
\end{subfigure}
%
\begin{subfigure}[b]{0.09\textwidth}
    \includegraphics[clip, trim=2.35cm 1.75cm 4.5cm 0cm,width=\textwidth]{img/convergence/8334.pdf}
    \caption{8334}
    \label{fig:convergence_8334}
\end{subfigure}
%
\begin{subfigure}[b]{0.09\textwidth}
    \includegraphics[clip, trim=2.35cm 1.75cm 4.5cm 0cm,width=\textwidth]{img/convergence/8336.pdf}
    \caption{8336}
    \label{fig:convergence_8336}
\end{subfigure}
%
\begin{subfigure}[b]{0.09\textwidth}
    \includegraphics[clip, trim=2.35cm 1.75cm 4.5cm 0cm,width=\textwidth]{img/convergence/8344.pdf}
    \caption{8344}
    \label{fig:convergence_8344}
\end{subfigure}
%
\begin{subfigure}[b]{0.09\textwidth}
    \includegraphics[clip, trim=2.35cm 1.75cm 4.5cm 0cm,width=\textwidth]{img/convergence/8345.pdf}
    \caption{8345}
    \label{fig:convergence_8345}
\end{subfigure}
%
\begin{subfigure}[b]{0.09\textwidth}
    \includegraphics[clip, trim=2.35cm 1.75cm 4.5cm 0cm,width=\textwidth]{img/convergence/8346.pdf}
    \caption{8346}
    \label{fig:convergence_8346}
\end{subfigure}
%
\begin{subfigure}[b]{0.09\textwidth}
    \includegraphics[clip, trim=2.35cm 1.75cm 4.5cm 0cm,width=\textwidth]{img/convergence/8347.pdf}
    \caption{8347}
    \label{fig:convergence_8347}
\end{subfigure}
%
\begin{subfigure}[b]{0.09\textwidth}
    \includegraphics[clip, trim=2.35cm 1.75cm 4.5cm 0cm,width=\textwidth]{img/convergence/8353.pdf}
    \caption{8353}
    \label{fig:convergence_8353}
\end{subfigure}
%
\begin{subfigure}[b]{0.09\textwidth}
    \includegraphics[clip, trim=2.35cm 1.75cm 4.5cm 0cm,width=\textwidth]{img/convergence/8354.pdf}
    \caption{8354}
    \label{fig:convergence_8354}
\end{subfigure}
%
\begin{subfigure}[b]{0.09\textwidth}
    \includegraphics[clip, trim=2.35cm 1.75cm 4.5cm 0cm,width=\textwidth]{img/convergence/8355.pdf}
    \caption{8355}
    \label{fig:convergence_8355}
\end{subfigure}
%
\begin{subfigure}[b]{0.09\textwidth}
    \includegraphics[clip, trim=2.35cm 1.75cm 4.5cm 0cm,width=\textwidth]{img/convergence/8356.pdf}
    \caption{8356}
    \label{fig:convergence_8356}
\end{subfigure}
%
\begin{subfigure}[b]{0.09\textwidth}
    \includegraphics[clip, trim=2.35cm 1.75cm 4.5cm 0cm,width=\textwidth]{img/convergence/8357.pdf}
    \caption{8357}
    \label{fig:convergence_8357}
\end{subfigure}
%
\begin{subfigure}[b]{0.09\textwidth}
    \includegraphics[clip, trim=2.35cm 1.75cm 4.5cm 0cm,width=\textwidth]{img/convergence/8358.pdf}
    \caption{8358}
    \label{fig:convergence_8358}
\end{subfigure}
%
\begin{subfigure}[b]{0.09\textwidth}
    \includegraphics[clip, trim=2.35cm 1.75cm 4.5cm 0cm,width=\textwidth]{img/convergence/8359.pdf}
    \caption{8359}
    \label{fig:convergence_8359}
\end{subfigure}
%
\begin{subfigure}[b]{0.09\textwidth}
    \includegraphics[clip, trim=2.35cm 1.75cm 4.5cm 0cm,width=\textwidth]{img/convergence/8361.pdf}
    \caption{8361}
    \label{fig:convergence_8361}
\end{subfigure}
%
\begin{subfigure}[b]{0.09\textwidth}
    \includegraphics[clip, trim=2.35cm 1.75cm 4.5cm 0cm,width=\textwidth]{img/convergence/8362.pdf}
    \caption{8362}
    \label{fig:convergence_8362}
\end{subfigure}
%
\begin{subfigure}[b]{0.09\textwidth}
    \includegraphics[clip, trim=2.35cm 1.75cm 4.5cm 0cm,width=\textwidth]{img/convergence/8363.pdf}
    \caption{8363}
    \label{fig:convergence_8363}
\end{subfigure}
%
\begin{subfigure}[b]{0.09\textwidth}
    \includegraphics[clip, trim=2.35cm 1.75cm 4.5cm 0cm,width=\textwidth]{img/convergence/8364.pdf}
    \caption{8364}
    \label{fig:convergence_8364}
\end{subfigure}
%
\begin{subfigure}[b]{0.09\textwidth}
    \includegraphics[clip, trim=2.35cm 1.75cm 4.5cm 0cm,width=\textwidth]{img/convergence/8365.pdf}
    \caption{8365}
    \label{fig:convergence_8365}
\end{subfigure}
%
\begin{subfigure}[b]{0.09\textwidth}
    \includegraphics[clip, trim=2.35cm 1.75cm 4.5cm 0cm,width=\textwidth]{img/convergence/8382.pdf}
    \caption{8382}
    \label{fig:convergence_8382}
\end{subfigure}
%
\begin{subfigure}[b]{0.09\textwidth}
    \includegraphics[clip, trim=2.35cm 1.75cm 4.5cm 0cm,width=\textwidth]{img/convergence/8400.pdf}
    \caption{8400}
    \label{fig:convergence_8400}
\end{subfigure}
%
\begin{subfigure}[b]{0.09\textwidth}
    \includegraphics[clip, trim=2.35cm 1.75cm 4.5cm 0cm,width=\textwidth]{img/convergence/8500.pdf}
    \caption{8500}
    \label{fig:convergence_8500}
\end{subfigure}
%
\begin{subfigure}[b]{0.09\textwidth}
    \includegraphics[clip, trim=2.35cm 1.75cm 4.5cm 0cm,width=\textwidth]{img/convergence/8512.pdf}
    \caption{8512}
    \label{fig:convergence_8512}
\end{subfigure}
%
\begin{subfigure}[b]{0.09\textwidth}
    \includegraphics[clip, trim=2.35cm 1.75cm 4.5cm 0cm,width=\textwidth]{img/convergence/8513.pdf}
    \caption{8513}
    \label{fig:convergence_8513}
\end{subfigure}
%
\begin{subfigure}[b]{0.09\textwidth}
    \includegraphics[clip, trim=2.35cm 1.75cm 4.5cm 0cm,width=\textwidth]{img/convergence/8514.pdf}
    \caption{8514}
    \label{fig:convergence_8514}
\end{subfigure}
%
\begin{subfigure}[b]{0.09\textwidth}
    \includegraphics[clip, trim=2.35cm 1.75cm 4.5cm 0cm,width=\textwidth]{img/convergence/8515.pdf}
    \caption{8515}
    \label{fig:convergence_8515}
\end{subfigure}
%
\begin{subfigure}[b]{0.09\textwidth}
    \includegraphics[clip, trim=2.35cm 1.75cm 4.5cm 0cm,width=\textwidth]{img/convergence/8516.pdf}
    \caption{8516}
    \label{fig:convergence_8516}
\end{subfigure}
%
\begin{subfigure}[b]{0.09\textwidth}
    \includegraphics[clip, trim=2.35cm 1.75cm 4.5cm 0cm,width=\textwidth]{img/convergence/8517.pdf}
    \caption{8517}
    \label{fig:convergence_8517}
\end{subfigure}
%
\begin{subfigure}[b]{0.09\textwidth}
    \includegraphics[clip, trim=2.35cm 1.75cm 4.5cm 0cm,width=\textwidth]{img/convergence/8518.pdf}
    \caption{8518}
    \label{fig:convergence_8518}
\end{subfigure}
%
\begin{subfigure}[b]{0.09\textwidth}
    \includegraphics[clip, trim=2.35cm 1.75cm 4.5cm 0cm,width=\textwidth]{img/convergence/8521.pdf}
    \caption{8521}
    \label{fig:convergence_8521}
\end{subfigure}
%
\begin{subfigure}[b]{0.09\textwidth}
    \includegraphics[clip, trim=2.35cm 1.75cm 4.5cm 0cm,width=\textwidth]{img/convergence/8522.pdf}
    \caption{8522}
    \label{fig:convergence_8522}
\end{subfigure}
%
\begin{subfigure}[b]{0.09\textwidth}
    \includegraphics[clip, trim=2.35cm 1.75cm 4.5cm 0cm,width=\textwidth]{img/convergence/8523.pdf}
    \caption{8523}
    \label{fig:convergence_8523}
\end{subfigure}
%
\begin{subfigure}[b]{0.09\textwidth}
    \includegraphics[clip, trim=2.35cm 1.75cm 4.5cm 0cm,width=\textwidth]{img/convergence/8526.pdf}
    \caption{8526}
    \label{fig:convergence_8526}
\end{subfigure}
%
\begin{subfigure}[b]{0.09\textwidth}
    \includegraphics[clip, trim=2.35cm 1.75cm 4.5cm 0cm,width=\textwidth]{img/convergence/8527.pdf}
    \caption{8527}
    \label{fig:convergence_8527}
\end{subfigure}
%
\begin{subfigure}[b]{0.09\textwidth}
    \includegraphics[clip, trim=2.35cm 1.75cm 4.5cm 0cm,width=\textwidth]{img/convergence/8529.pdf}
    \caption{8529}
    \label{fig:convergence_8529}
\end{subfigure}
%
\begin{subfigure}[b]{0.09\textwidth}
    \includegraphics[clip, trim=2.35cm 1.75cm 4.5cm 0cm,width=\textwidth]{img/convergence/8534.pdf}
    \caption{8534}
    \label{fig:convergence_8534}
\end{subfigure}
%
\begin{subfigure}[b]{0.09\textwidth}
    \includegraphics[clip, trim=2.35cm 1.75cm 4.5cm 0cm,width=\textwidth]{img/convergence/8536.pdf}
    \caption{8536}
    \label{fig:convergence_8536}
\end{subfigure}
%
\begin{subfigure}[b]{0.09\textwidth}
    \includegraphics[clip, trim=2.35cm 1.75cm 4.5cm 0cm,width=\textwidth]{img/convergence/8540.pdf}
    \caption{8540}
    \label{fig:convergence_8540}
\end{subfigure}
%
\begin{subfigure}[b]{0.09\textwidth}
    \includegraphics[clip, trim=2.35cm 1.75cm 4.5cm 0cm,width=\textwidth]{img/convergence/8541.pdf}
    \caption{8541}
    \label{fig:convergence_8541}
\end{subfigure}
%
\begin{subfigure}[b]{0.09\textwidth}
    \includegraphics[clip, trim=2.35cm 1.75cm 4.5cm 0cm,width=\textwidth]{img/convergence/8542.pdf}
    \caption{8542}
    \label{fig:convergence_8542}
\end{subfigure}
%
\begin{subfigure}[b]{0.09\textwidth}
    \includegraphics[clip, trim=2.35cm 1.75cm 4.5cm 0cm,width=\textwidth]{img/convergence/8543.pdf}
    \caption{8543}
    \label{fig:convergence_8543}
\end{subfigure}
%
\begin{subfigure}[b]{0.09\textwidth}
    \includegraphics[clip, trim=2.35cm 1.75cm 4.5cm 0cm,width=\textwidth]{img/convergence/8544.pdf}
    \caption{8544}
    \label{fig:convergence_8544}
\end{subfigure}
%
\begin{subfigure}[b]{0.09\textwidth}
    \includegraphics[clip, trim=2.35cm 1.75cm 4.5cm 0cm,width=\textwidth]{img/convergence/8545.pdf}
    \caption{8545}
    \label{fig:convergence_8545}
\end{subfigure}
%
\begin{subfigure}[b]{0.09\textwidth}
    \includegraphics[clip, trim=2.35cm 1.75cm 4.5cm 0cm,width=\textwidth]{img/convergence/8546.pdf}
    \caption{8546}
    \label{fig:convergence_8546}
\end{subfigure}
%
\begin{subfigure}[b]{0.09\textwidth}
    \includegraphics[clip, trim=2.35cm 1.75cm 4.5cm 0cm,width=\textwidth]{img/convergence/8547.pdf}
    \caption{8547}
    \label{fig:convergence_8547}
\end{subfigure}
%
\begin{subfigure}[b]{0.09\textwidth}
    \includegraphics[clip, trim=2.35cm 1.75cm 4.5cm 0cm,width=\textwidth]{img/convergence/8548.pdf}
    \caption{8548}
    \label{fig:convergence_8548}
\end{subfigure}
%
\begin{subfigure}[b]{0.09\textwidth}
    \includegraphics[clip, trim=2.35cm 1.75cm 4.5cm 0cm,width=\textwidth]{img/convergence/8549.pdf}
    \caption{8549}
    \label{fig:convergence_8549}
\end{subfigure}
%
\begin{subfigure}[b]{0.09\textwidth}
    \includegraphics[clip, trim=2.35cm 1.75cm 4.5cm 0cm,width=\textwidth]{img/convergence/8567.pdf}
    \caption{8567}
    \label{fig:convergence_8567}
\end{subfigure}
%
\begin{subfigure}[b]{0.09\textwidth}
    \includegraphics[clip, trim=2.35cm 1.75cm 4.5cm 0cm,width=\textwidth]{img/convergence/8568.pdf}
    \caption{8568}
    \label{fig:convergence_8568}
\end{subfigure}
%
\begin{subfigure}[b]{0.09\textwidth}
    \includegraphics[clip, trim=2.35cm 1.75cm 4.5cm 0cm,width=\textwidth]{img/convergence/8573.pdf}
    \caption{8573}
    \label{fig:convergence_8573}
\end{subfigure}
%
\begin{subfigure}[b]{0.09\textwidth}
    \includegraphics[clip, trim=2.35cm 1.75cm 4.5cm 0cm,width=\textwidth]{img/convergence/8574.pdf}
    \caption{8574}
    \label{fig:convergence_8574}
\end{subfigure}
%
\begin{subfigure}[b]{0.09\textwidth}
    \includegraphics[clip, trim=2.35cm 1.75cm 4.5cm 0cm,width=\textwidth]{img/convergence/8575.pdf}
    \caption{8575}
    \label{fig:convergence_8575}
\end{subfigure}
%
\begin{subfigure}[b]{0.09\textwidth}
    \includegraphics[clip, trim=2.35cm 1.75cm 4.5cm 0cm,width=\textwidth]{img/convergence/8578.pdf}
    \caption{8578}
    \label{fig:convergence_8578}
\end{subfigure}
%
\begin{subfigure}[b]{0.09\textwidth}
    \includegraphics[clip, trim=2.35cm 1.75cm 4.5cm 0cm,width=\textwidth]{img/convergence/8588.pdf}
    \caption{8588}
    \label{fig:convergence_8588}
\end{subfigure}
%
\begin{subfigure}[b]{0.09\textwidth}
    \includegraphics[clip, trim=2.35cm 1.75cm 4.5cm 0cm,width=\textwidth]{img/convergence/8589.pdf}
    \caption{8589}
    \label{fig:convergence_8589}
\end{subfigure}
%
\begin{subfigure}[b]{0.09\textwidth}
    \includegraphics[clip, trim=2.35cm 1.75cm 4.5cm 0cm,width=\textwidth]{img/convergence/8600.pdf}
    \caption{8600}
    \label{fig:convergence_8600}
\end{subfigure}
%
\begin{subfigure}[b]{0.09\textwidth}
    \includegraphics[clip, trim=2.35cm 1.75cm 4.5cm 0cm,width=\textwidth]{img/convergence/8653.pdf}
    \caption{8653}
    \label{fig:convergence_8653}
\end{subfigure}
%
\begin{subfigure}[b]{0.09\textwidth}
    \includegraphics[clip, trim=2.35cm 1.75cm 4.5cm 0cm,width=\textwidth]{img/convergence/8654.pdf}
    \caption{8654}
    \label{fig:convergence_8654}
\end{subfigure}
%
\begin{subfigure}[b]{0.09\textwidth}
    \includegraphics[clip, trim=2.35cm 1.75cm 4.5cm 0cm,width=\textwidth]{img/convergence/8658.pdf}
    \caption{8658}
    \label{fig:convergence_8658}
\end{subfigure}
%
\begin{subfigure}[b]{0.09\textwidth}
    \includegraphics[clip, trim=2.35cm 1.75cm 4.5cm 0cm,width=\textwidth]{img/convergence/8661.pdf}
    \caption{8661}
    \label{fig:convergence_8661}
\end{subfigure}
%
\begin{subfigure}[b]{0.09\textwidth}
    \includegraphics[clip, trim=2.35cm 1.75cm 4.5cm 0cm,width=\textwidth]{img/convergence/8664.pdf}
    \caption{8664}
    \label{fig:convergence_8664}
\end{subfigure}
%
\begin{subfigure}[b]{0.09\textwidth}
    \includegraphics[clip, trim=2.35cm 1.75cm 4.5cm 0cm,width=\textwidth]{img/convergence/8665.pdf}
    \caption{8665}
    \label{fig:convergence_8665}
\end{subfigure}
%
\begin{subfigure}[b]{0.09\textwidth}
    \includegraphics[clip, trim=2.35cm 1.75cm 4.5cm 0cm,width=\textwidth]{img/convergence/8666.pdf}
    \caption{8666}
    \label{fig:convergence_8666}
\end{subfigure}
%
\begin{subfigure}[b]{0.09\textwidth}
    \includegraphics[clip, trim=2.35cm 1.75cm 4.5cm 0cm,width=\textwidth]{img/convergence/8693.pdf}
    \caption{8693}
    \label{fig:convergence_8693}
\end{subfigure}
%
\begin{subfigure}[b]{0.09\textwidth}
    \includegraphics[clip, trim=2.35cm 1.75cm 4.5cm 0cm,width=\textwidth]{img/convergence/8700.pdf}
    \caption{8700}
    \label{fig:convergence_8700}
\end{subfigure}
%
\begin{subfigure}[b]{0.09\textwidth}
    \includegraphics[clip, trim=2.35cm 1.75cm 4.5cm 0cm,width=\textwidth]{img/convergence/8800.pdf}
    \caption{8800}
    \label{fig:convergence_8800}
\end{subfigure}
%
\begin{subfigure}[b]{0.09\textwidth}
    \includegraphics[clip, trim=2.35cm 1.75cm 4.5cm 0cm,width=\textwidth]{img/convergence/8900.pdf}
    \caption{8900}
    \label{fig:convergence_8900}
\end{subfigure}
%
\begin{subfigure}[b]{0.09\textwidth}
    \includegraphics[clip, trim=2.35cm 1.75cm 4.5cm 0cm,width=\textwidth]{img/convergence/9000.pdf}
    \caption{9000}
    \label{fig:convergence_9000}
\end{subfigure}
%
\begin{subfigure}[b]{0.09\textwidth}
    \includegraphics[clip, trim=2.35cm 1.75cm 4.5cm 0cm,width=\textwidth]{img/convergence/9048.pdf}
    \caption{9048}
    \label{fig:convergence_9048}
\end{subfigure}
%
\begin{subfigure}[b]{0.09\textwidth}
    \includegraphics[clip, trim=2.35cm 1.75cm 4.5cm 0cm,width=\textwidth]{img/convergence/9049.pdf}
    \caption{9049}
    \label{fig:convergence_9049}
\end{subfigure}
%
\begin{subfigure}[b]{0.09\textwidth}
    \includegraphics[clip, trim=2.35cm 1.75cm 4.5cm 0cm,width=\textwidth]{img/convergence/9052.pdf}
    \caption{9052}
    \label{fig:convergence_9052}
\end{subfigure}
%
\begin{subfigure}[b]{0.09\textwidth}
    \includegraphics[clip, trim=2.35cm 1.75cm 4.5cm 0cm,width=\textwidth]{img/convergence/9053.pdf}
    \caption{9053}
    \label{fig:convergence_9053}
\end{subfigure}
%
\begin{subfigure}[b]{0.09\textwidth}
    \includegraphics[clip, trim=2.35cm 1.75cm 4.5cm 0cm,width=\textwidth]{img/convergence/9055.pdf}
    \caption{9055}
    \label{fig:convergence_9055}
\end{subfigure}
%
\begin{subfigure}[b]{0.09\textwidth}
    \includegraphics[clip, trim=2.35cm 1.75cm 4.5cm 0cm,width=\textwidth]{img/convergence/9057.pdf}
    \caption{9057}
    \label{fig:convergence_9057}
\end{subfigure}
%
\begin{subfigure}[b]{0.09\textwidth}
    \includegraphics[clip, trim=2.35cm 1.75cm 4.5cm 0cm,width=\textwidth]{img/convergence/9072.pdf}
    \caption{9072}
    \label{fig:convergence_9072}
\end{subfigure}
%
\begin{subfigure}[b]{0.09\textwidth}
    \includegraphics[clip, trim=2.35cm 1.75cm 4.5cm 0cm,width=\textwidth]{img/convergence/9074.pdf}
    \caption{9074}
    \label{fig:convergence_9074}
\end{subfigure}
%
\begin{subfigure}[b]{0.09\textwidth}
    \includegraphics[clip, trim=2.35cm 1.75cm 4.5cm 0cm,width=\textwidth]{img/convergence/9075.pdf}
    \caption{9075}
    \label{fig:convergence_9075}
\end{subfigure}
%
\begin{subfigure}[b]{0.09\textwidth}
    \includegraphics[clip, trim=2.35cm 1.75cm 4.5cm 0cm,width=\textwidth]{img/convergence/9077.pdf}
    \caption{9077}
    \label{fig:convergence_9077}
\end{subfigure}
%
\begin{subfigure}[b]{0.09\textwidth}
    \includegraphics[clip, trim=2.35cm 1.75cm 4.5cm 0cm,width=\textwidth]{img/convergence/9079.pdf}
    \caption{9079}
    \label{fig:convergence_9079}
\end{subfigure}
%
\begin{subfigure}[b]{0.09\textwidth}
    \includegraphics[clip, trim=2.35cm 1.75cm 4.5cm 0cm,width=\textwidth]{img/convergence/9080.pdf}
    \caption{9080}
    \label{fig:convergence_9080}
\end{subfigure}
%
\begin{subfigure}[b]{0.09\textwidth}
    \includegraphics[clip, trim=2.35cm 1.75cm 4.5cm 0cm,width=\textwidth]{img/convergence/9081.pdf}
    \caption{9081}
    \label{fig:convergence_9081}
\end{subfigure}
%
\begin{subfigure}[b]{0.09\textwidth}
    \includegraphics[clip, trim=2.35cm 1.75cm 4.5cm 0cm,width=\textwidth]{img/convergence/9082.pdf}
    \caption{9082}
    \label{fig:convergence_9082}
\end{subfigure}
%
\begin{subfigure}[b]{0.09\textwidth}
    \includegraphics[clip, trim=2.35cm 1.75cm 4.5cm 0cm,width=\textwidth]{img/convergence/9083.pdf}
    \caption{9083}
    \label{fig:convergence_9083}
\end{subfigure}
%
\begin{subfigure}[b]{0.09\textwidth}
    \includegraphics[clip, trim=2.35cm 1.75cm 4.5cm 0cm,width=\textwidth]{img/convergence/9084.pdf}
    \caption{9084}
    \label{fig:convergence_9084}
\end{subfigure}
%
\begin{subfigure}[b]{0.09\textwidth}
    \includegraphics[clip, trim=2.35cm 1.75cm 4.5cm 0cm,width=\textwidth]{img/convergence/9090.pdf}
    \caption{9090}
    \label{fig:convergence_9090}
\end{subfigure}
%
\begin{subfigure}[b]{0.09\textwidth}
    \includegraphics[clip, trim=2.35cm 1.75cm 4.5cm 0cm,width=\textwidth]{img/convergence/9091.pdf}
    \caption{9091}
    \label{fig:convergence_9091}
\end{subfigure}
%
\begin{subfigure}[b]{0.09\textwidth}
    \includegraphics[clip, trim=2.35cm 1.75cm 4.5cm 0cm,width=\textwidth]{img/convergence/9092.pdf}
    \caption{9092}
    \label{fig:convergence_9092}
\end{subfigure}
%
\begin{subfigure}[b]{0.09\textwidth}
    \includegraphics[clip, trim=2.35cm 1.75cm 4.5cm 0cm,width=\textwidth]{img/convergence/9093.pdf}
    \caption{9093}
    \label{fig:convergence_9093}
\end{subfigure}
%
\begin{subfigure}[b]{0.09\textwidth}
    \includegraphics[clip, trim=2.35cm 1.75cm 4.5cm 0cm,width=\textwidth]{img/convergence/9094.pdf}
    \caption{9094}
    \label{fig:convergence_9094}
\end{subfigure}
%
\begin{subfigure}[b]{0.09\textwidth}
    \includegraphics[clip, trim=2.35cm 1.75cm 4.5cm 0cm,width=\textwidth]{img/convergence/9095.pdf}
    \caption{9095}
    \label{fig:convergence_9095}
\end{subfigure}
%
\begin{subfigure}[b]{0.09\textwidth}
    \includegraphics[clip, trim=2.35cm 1.75cm 4.5cm 0cm,width=\textwidth]{img/convergence/9096.pdf}
    \caption{9096}
    \label{fig:convergence_9096}
\end{subfigure}
%
\begin{subfigure}[b]{0.09\textwidth}
    \includegraphics[clip, trim=2.35cm 1.75cm 4.5cm 0cm,width=\textwidth]{img/convergence/9097.pdf}
    \caption{9097}
    \label{fig:convergence_9097}
\end{subfigure}
%
\begin{subfigure}[b]{0.09\textwidth}
    \includegraphics[clip, trim=2.35cm 1.75cm 4.5cm 0cm,width=\textwidth]{img/convergence/9098.pdf}
    \caption{9098}
    \label{fig:convergence_9098}
\end{subfigure}
%
\begin{subfigure}[b]{0.09\textwidth}
    \includegraphics[clip, trim=2.35cm 1.75cm 4.5cm 0cm,width=\textwidth]{img/convergence/9100.pdf}
    \caption{9100}
    \label{fig:convergence_9100}
\end{subfigure}
%
\begin{subfigure}[b]{0.09\textwidth}
    \includegraphics[clip, trim=2.35cm 1.75cm 4.5cm 0cm,width=\textwidth]{img/convergence/9101.pdf}
    \caption{9101}
    \label{fig:convergence_9101}
\end{subfigure}
%
\begin{subfigure}[b]{0.09\textwidth}
    \includegraphics[clip, trim=2.35cm 1.75cm 4.5cm 0cm,width=\textwidth]{img/convergence/9102.pdf}
    \caption{9102}
    \label{fig:convergence_9102}
\end{subfigure}
%
\begin{subfigure}[b]{0.09\textwidth}
    \includegraphics[clip, trim=2.35cm 1.75cm 4.5cm 0cm,width=\textwidth]{img/convergence/9103.pdf}
    \caption{9103}
    \label{fig:convergence_9103}
\end{subfigure}
%
\begin{subfigure}[b]{0.09\textwidth}
    \includegraphics[clip, trim=2.35cm 1.75cm 4.5cm 0cm,width=\textwidth]{img/convergence/9104.pdf}
    \caption{9104}
    \label{fig:convergence_9104}
\end{subfigure}
%
\begin{subfigure}[b]{0.09\textwidth}
    \includegraphics[clip, trim=2.35cm 1.75cm 4.5cm 0cm,width=\textwidth]{img/convergence/9109.pdf}
    \caption{9109}
    \label{fig:convergence_9109}
\end{subfigure}
%
\begin{subfigure}[b]{0.09\textwidth}
    \includegraphics[clip, trim=2.35cm 1.75cm 4.5cm 0cm,width=\textwidth]{img/convergence/9111.pdf}
    \caption{9111}
    \label{fig:convergence_9111}
\end{subfigure}
%
\begin{subfigure}[b]{0.09\textwidth}
    \includegraphics[clip, trim=2.35cm 1.75cm 4.5cm 0cm,width=\textwidth]{img/convergence/9114.pdf}
    \caption{9114}
    \label{fig:convergence_9114}
\end{subfigure}
%
\begin{subfigure}[b]{0.09\textwidth}
    \includegraphics[clip, trim=2.35cm 1.75cm 4.5cm 0cm,width=\textwidth]{img/convergence/9115.pdf}
    \caption{9115}
    \label{fig:convergence_9115}
\end{subfigure}
%
\begin{subfigure}[b]{0.09\textwidth}
    \includegraphics[clip, trim=2.35cm 1.75cm 4.5cm 0cm,width=\textwidth]{img/convergence/9119.pdf}
    \caption{9119}
    \label{fig:convergence_9119}
\end{subfigure}
%
\begin{subfigure}[b]{0.09\textwidth}
    \includegraphics[clip, trim=2.35cm 1.75cm 4.5cm 0cm,width=\textwidth]{img/convergence/9120.pdf}
    \caption{9120}
    \label{fig:convergence_9120}
\end{subfigure}
%
\begin{subfigure}[b]{0.09\textwidth}
    \includegraphics[clip, trim=2.35cm 1.75cm 4.5cm 0cm,width=\textwidth]{img/convergence/9123.pdf}
    \caption{9123}
    \label{fig:convergence_9123}
\end{subfigure}
%
\begin{subfigure}[b]{0.09\textwidth}
    \includegraphics[clip, trim=2.35cm 1.75cm 4.5cm 0cm,width=\textwidth]{img/convergence/9125.pdf}
    \caption{9125}
    \label{fig:convergence_9125}
\end{subfigure}
%
\begin{subfigure}[b]{0.09\textwidth}
    \includegraphics[clip, trim=2.35cm 1.75cm 4.5cm 0cm,width=\textwidth]{img/convergence/9128.pdf}
    \caption{9128}
    \label{fig:convergence_9128}
\end{subfigure}
%
\begin{subfigure}[b]{0.09\textwidth}
    \includegraphics[clip, trim=2.35cm 1.75cm 4.5cm 0cm,width=\textwidth]{img/convergence/9129.pdf}
    \caption{9129}
    \label{fig:convergence_9129}
\end{subfigure}
%
\begin{subfigure}[b]{0.09\textwidth}
    \includegraphics[clip, trim=2.35cm 1.75cm 4.5cm 0cm,width=\textwidth]{img/convergence/9130.pdf}
    \caption{9130}
    \label{fig:convergence_9130}
\end{subfigure}
%
\begin{subfigure}[b]{0.09\textwidth}
    \includegraphics[clip, trim=2.35cm 1.75cm 4.5cm 0cm,width=\textwidth]{img/convergence/9133.pdf}
    \caption{9133}
    \label{fig:convergence_9133}
\end{subfigure}
%
\begin{subfigure}[b]{0.09\textwidth}
    \includegraphics[clip, trim=2.35cm 1.75cm 4.5cm 0cm,width=\textwidth]{img/convergence/9135.pdf}
    \caption{9135}
    \label{fig:convergence_9135}
\end{subfigure}
%
\begin{subfigure}[b]{0.09\textwidth}
    \includegraphics[clip, trim=2.35cm 1.75cm 4.5cm 0cm,width=\textwidth]{img/convergence/9136.pdf}
    \caption{9136}
    \label{fig:convergence_9136}
\end{subfigure}
%
\begin{subfigure}[b]{0.09\textwidth}
    \includegraphics[clip, trim=2.35cm 1.75cm 4.5cm 0cm,width=\textwidth]{img/convergence/9147.pdf}
    \caption{9147}
    \label{fig:convergence_9147}
\end{subfigure}
%
\begin{subfigure}[b]{0.09\textwidth}
    \includegraphics[clip, trim=2.35cm 1.75cm 4.5cm 0cm,width=\textwidth]{img/convergence/9148.pdf}
    \caption{9148}
    \label{fig:convergence_9148}
\end{subfigure}
%
\begin{subfigure}[b]{0.09\textwidth}
    \includegraphics[clip, trim=2.35cm 1.75cm 4.5cm 0cm,width=\textwidth]{img/convergence/9149.pdf}
    \caption{9149}
    \label{fig:convergence_9149}
\end{subfigure}
%
\begin{subfigure}[b]{0.09\textwidth}
    \includegraphics[clip, trim=2.35cm 1.75cm 4.5cm 0cm,width=\textwidth]{img/convergence/9151.pdf}
    \caption{9151}
    \label{fig:convergence_9151}
\end{subfigure}
%
\begin{subfigure}[b]{0.09\textwidth}
    \includegraphics[clip, trim=2.35cm 1.75cm 4.5cm 0cm,width=\textwidth]{img/convergence/9152.pdf}
    \caption{9152}
    \label{fig:convergence_9152}
\end{subfigure}
%
\begin{subfigure}[b]{0.09\textwidth}
    \includegraphics[clip, trim=2.35cm 1.75cm 4.5cm 0cm,width=\textwidth]{img/convergence/9153.pdf}
    \caption{9153}
    \label{fig:convergence_9153}
\end{subfigure}
%
\begin{subfigure}[b]{0.09\textwidth}
    \includegraphics[clip, trim=2.35cm 1.75cm 4.5cm 0cm,width=\textwidth]{img/convergence/9154.pdf}
    \caption{9154}
    \label{fig:convergence_9154}
\end{subfigure}
%
\begin{subfigure}[b]{0.09\textwidth}
    \includegraphics[clip, trim=2.35cm 1.75cm 4.5cm 0cm,width=\textwidth]{img/convergence/9157.pdf}
    \caption{9157}
    \label{fig:convergence_9157}
\end{subfigure}
%
\begin{subfigure}[b]{0.09\textwidth}
    \includegraphics[clip, trim=2.35cm 1.75cm 4.5cm 0cm,width=\textwidth]{img/convergence/9161.pdf}
    \caption{9161}
    \label{fig:convergence_9161}
\end{subfigure}
%
\begin{subfigure}[b]{0.09\textwidth}
    \includegraphics[clip, trim=2.35cm 1.75cm 4.5cm 0cm,width=\textwidth]{img/convergence/9162.pdf}
    \caption{9162}
    \label{fig:convergence_9162}
\end{subfigure}
%
\begin{subfigure}[b]{0.09\textwidth}
    \includegraphics[clip, trim=2.35cm 1.75cm 4.5cm 0cm,width=\textwidth]{img/convergence/9164.pdf}
    \caption{9164}
    \label{fig:convergence_9164}
\end{subfigure}
%
\begin{subfigure}[b]{0.09\textwidth}
    \includegraphics[clip, trim=2.35cm 1.75cm 4.5cm 0cm,width=\textwidth]{img/convergence/9165.pdf}
    \caption{9165}
    \label{fig:convergence_9165}
\end{subfigure}
%
\begin{subfigure}[b]{0.09\textwidth}
    \includegraphics[clip, trim=2.35cm 1.75cm 4.5cm 0cm,width=\textwidth]{img/convergence/9166.pdf}
    \caption{9166}
    \label{fig:convergence_9166}
\end{subfigure}
%
\begin{subfigure}[b]{0.09\textwidth}
    \includegraphics[clip, trim=2.35cm 1.75cm 4.5cm 0cm,width=\textwidth]{img/convergence/9167.pdf}
    \caption{9167}
    \label{fig:convergence_9167}
\end{subfigure}
%
\begin{subfigure}[b]{0.09\textwidth}
    \includegraphics[clip, trim=2.35cm 1.75cm 4.5cm 0cm,width=\textwidth]{img/convergence/9173.pdf}
    \caption{9173}
    \label{fig:convergence_9173}
\end{subfigure}
%
\begin{subfigure}[b]{0.09\textwidth}
    \includegraphics[clip, trim=2.35cm 1.75cm 4.5cm 0cm,width=\textwidth]{img/convergence/9174.pdf}
    \caption{9174}
    \label{fig:convergence_9174}
\end{subfigure}
%
\begin{subfigure}[b]{0.09\textwidth}
    \includegraphics[clip, trim=2.35cm 1.75cm 4.5cm 0cm,width=\textwidth]{img/convergence/9179.pdf}
    \caption{9179}
    \label{fig:convergence_9179}
\end{subfigure}
%
\begin{subfigure}[b]{0.09\textwidth}
    \includegraphics[clip, trim=2.35cm 1.75cm 4.5cm 0cm,width=\textwidth]{img/convergence/9188.pdf}
    \caption{9188}
    \label{fig:convergence_9188}
\end{subfigure}
%
\begin{subfigure}[b]{0.09\textwidth}
    \includegraphics[clip, trim=2.35cm 1.75cm 4.5cm 0cm,width=\textwidth]{img/convergence/9200.pdf}
    \caption{9200}
    \label{fig:convergence_9200}
\end{subfigure}
%
\begin{subfigure}[b]{0.09\textwidth}
    \includegraphics[clip, trim=2.35cm 1.75cm 4.5cm 0cm,width=\textwidth]{img/convergence/9202.pdf}
    \caption{9202}
    \label{fig:convergence_9202}
\end{subfigure}
%
\begin{subfigure}[b]{0.09\textwidth}
    \includegraphics[clip, trim=2.35cm 1.75cm 4.5cm 0cm,width=\textwidth]{img/convergence/9300.pdf}
    \caption{9300}
    \label{fig:convergence_9300}
\end{subfigure}
%
\begin{subfigure}[b]{0.09\textwidth}
    \includegraphics[clip, trim=2.35cm 1.75cm 4.5cm 0cm,width=\textwidth]{img/convergence/9389.pdf}
    \caption{9389}
    \label{fig:convergence_9389}
\end{subfigure}
%
\begin{subfigure}[b]{0.09\textwidth}
    \includegraphics[clip, trim=2.35cm 1.75cm 4.5cm 0cm,width=\textwidth]{img/convergence/9400.pdf}
    \caption{9400}
    \label{fig:convergence_9400}
\end{subfigure}
%
\begin{subfigure}[b]{0.09\textwidth}
    \includegraphics[clip, trim=2.35cm 1.75cm 4.5cm 0cm,width=\textwidth]{img/convergence/9500.pdf}
    \caption{9500}
    \label{fig:convergence_9500}
\end{subfigure}
%
\begin{subfigure}[b]{0.09\textwidth}
    \includegraphics[clip, trim=2.35cm 1.75cm 4.5cm 0cm,width=\textwidth]{img/convergence/9600.pdf}
    \caption{9600}
    \label{fig:convergence_9600}
\end{subfigure}
%
\begin{subfigure}[b]{0.09\textwidth}
    \includegraphics[clip, trim=2.35cm 1.75cm 4.5cm 0cm,width=\textwidth]{img/convergence/9700.pdf}
    \caption{9700}
    \label{fig:convergence_9700}
\end{subfigure}
%
\begin{subfigure}[b]{0.09\textwidth}
    \includegraphics[clip, trim=2.35cm 1.75cm 4.5cm 0cm,width=\textwidth]{img/convergence/9800.pdf}
    \caption{9800}
    \label{fig:convergence_9800}
\end{subfigure}
%
\begin{subfigure}[b]{0.09\textwidth}
    \includegraphics[clip, trim=2.35cm 1.75cm 4.5cm 0cm,width=\textwidth]{img/convergence/9900.pdf}
    \caption{9900}
    \label{fig:convergence_9900}
\end{subfigure}
%
\begin{subfigure}[b]{0.09\textwidth}
    \includegraphics[clip, trim=2.35cm 1.75cm 4.5cm 0cm,width=\textwidth]{img/convergence/9979.pdf}
    \caption{9979}
    \label{fig:convergence_9979}
\end{subfigure}
%
\begin{subfigure}[b]{0.09\textwidth}
    \includegraphics[clip, trim=2.35cm 1.75cm 4.5cm 0cm,width=\textwidth]{img/convergence/9982.pdf}
    \caption{9982}
    \label{fig:convergence_9982}
\end{subfigure}
%
\begin{subfigure}[b]{0.09\textwidth}
    \includegraphics[clip, trim=2.35cm 1.75cm 4.5cm 0cm,width=\textwidth]{img/convergence/9985.pdf}
    \caption{9985}
    \label{fig:convergence_9985}
\end{subfigure}
%
\begin{subfigure}[b]{0.09\textwidth}
    \includegraphics[clip, trim=2.35cm 1.75cm 4.5cm 0cm,width=\textwidth]{img/convergence/9988.pdf}
    \caption{9988}
    \label{fig:convergence_9988}
\end{subfigure}
%
\begin{subfigure}[b]{0.09\textwidth}
    \includegraphics[clip, trim=2.35cm 1.75cm 4.5cm 0cm,width=\textwidth]{img/convergence/9991.pdf}
    \caption{9991}
    \label{fig:convergence_9991}
\end{subfigure}
%
\begin{subfigure}[b]{0.09\textwidth}
    \includegraphics[clip, trim=2.35cm 1.75cm 4.5cm 0cm,width=\textwidth]{img/convergence/10000.pdf}
    \caption{10000}
    \label{fig:convergence_10000}
\end{subfigure}
%
\begin{subfigure}[b]{0.09\textwidth}
    \includegraphics[clip, trim=2.35cm 1.75cm 4.5cm 0cm,width=\textwidth]{img/convergence/10057.pdf}
    \caption{10057}
    \label{fig:convergence_10057}
\end{subfigure}
%
\begin{subfigure}[b]{0.09\textwidth}
    \includegraphics[clip, trim=2.35cm 1.75cm 4.5cm 0cm,width=\textwidth]{img/convergence/10065.pdf}
    \caption{10065}
    \label{fig:convergence_10065}
\end{subfigure}
%
\begin{subfigure}[b]{0.09\textwidth}
    \includegraphics[clip, trim=2.35cm 1.75cm 4.5cm 0cm,width=\textwidth]{img/convergence/10066.pdf}
    \caption{10066}
    \label{fig:convergence_10066}
\end{subfigure}
%
\begin{subfigure}[b]{0.09\textwidth}
    \includegraphics[clip, trim=2.35cm 1.75cm 4.5cm 0cm,width=\textwidth]{img/convergence/10100.pdf}
    \caption{10100}
    \label{fig:convergence_10100}
\end{subfigure}
%
\begin{subfigure}[b]{0.09\textwidth}
    \includegraphics[clip, trim=2.35cm 1.75cm 4.5cm 0cm,width=\textwidth]{img/convergence/10106.pdf}
    \caption{10106}
    \label{fig:convergence_10106}
\end{subfigure}
%
\begin{subfigure}[b]{0.09\textwidth}
    \includegraphics[clip, trim=2.35cm 1.75cm 4.5cm 0cm,width=\textwidth]{img/convergence/10109.pdf}
    \caption{10109}
    \label{fig:convergence_10109}
\end{subfigure}
%
\begin{subfigure}[b]{0.09\textwidth}
    \includegraphics[clip, trim=2.35cm 1.75cm 4.5cm 0cm,width=\textwidth]{img/convergence/10136.pdf}
    \caption{10136}
    \label{fig:convergence_10136}
\end{subfigure}
%
\begin{subfigure}[b]{0.09\textwidth}
    \includegraphics[clip, trim=2.35cm 1.75cm 4.5cm 0cm,width=\textwidth]{img/convergence/10137.pdf}
    \caption{10137}
    \label{fig:convergence_10137}
\end{subfigure}
%
\begin{subfigure}[b]{0.09\textwidth}
    \includegraphics[clip, trim=2.35cm 1.75cm 4.5cm 0cm,width=\textwidth]{img/convergence/10200.pdf}
    \caption{10200}
    \label{fig:convergence_10200}
\end{subfigure}
%
\begin{subfigure}[b]{0.09\textwidth}
    \includegraphics[clip, trim=2.35cm 1.75cm 4.5cm 0cm,width=\textwidth]{img/convergence/10237.pdf}
    \caption{10237}
    \label{fig:convergence_10237}
\end{subfigure}
%
\begin{subfigure}[b]{0.09\textwidth}
    \includegraphics[clip, trim=2.35cm 1.75cm 4.5cm 0cm,width=\textwidth]{img/convergence/10244.pdf}
    \caption{10244}
    \label{fig:convergence_10244}
\end{subfigure}
%
\begin{subfigure}[b]{0.09\textwidth}
    \includegraphics[clip, trim=2.35cm 1.75cm 4.5cm 0cm,width=\textwidth]{img/convergence/10291.pdf}
    \caption{10291}
    \label{fig:convergence_10291}
\end{subfigure}
%
\begin{subfigure}[b]{0.09\textwidth}
    \includegraphics[clip, trim=2.35cm 1.75cm 4.5cm 0cm,width=\textwidth]{img/convergence/10293.pdf}
    \caption{10293}
    \label{fig:convergence_10293}
\end{subfigure}
%
\begin{subfigure}[b]{0.09\textwidth}
    \includegraphics[clip, trim=2.35cm 1.75cm 4.5cm 0cm,width=\textwidth]{img/convergence/10300.pdf}
    \caption{10300}
    \label{fig:convergence_10300}
\end{subfigure}
%
\begin{subfigure}[b]{0.09\textwidth}
    \includegraphics[clip, trim=2.35cm 1.75cm 4.5cm 0cm,width=\textwidth]{img/convergence/10316.pdf}
    \caption{10316}
    \label{fig:convergence_10316}
\end{subfigure}
%
\begin{subfigure}[b]{0.09\textwidth}
    \includegraphics[clip, trim=2.35cm 1.75cm 4.5cm 0cm,width=\textwidth]{img/convergence/10317.pdf}
    \caption{10317}
    \label{fig:convergence_10317}
\end{subfigure}
%
\begin{subfigure}[b]{0.09\textwidth}
    \includegraphics[clip, trim=2.35cm 1.75cm 4.5cm 0cm,width=\textwidth]{img/convergence/10332.pdf}
    \caption{10332}
    \label{fig:convergence_10332}
\end{subfigure}
%
\begin{subfigure}[b]{0.09\textwidth}
    \includegraphics[clip, trim=2.35cm 1.75cm 4.5cm 0cm,width=\textwidth]{img/convergence/10336.pdf}
    \caption{10336}
    \label{fig:convergence_10336}
\end{subfigure}
%
\begin{subfigure}[b]{0.09\textwidth}
    \includegraphics[clip, trim=2.35cm 1.75cm 4.5cm 0cm,width=\textwidth]{img/convergence/10338.pdf}
    \caption{10338}
    \label{fig:convergence_10338}
\end{subfigure}
%
\begin{subfigure}[b]{0.09\textwidth}
    \includegraphics[clip, trim=2.35cm 1.75cm 4.5cm 0cm,width=\textwidth]{img/convergence/10341.pdf}
    \caption{10341}
    \label{fig:convergence_10341}
\end{subfigure}
%
\begin{subfigure}[b]{0.09\textwidth}
    \includegraphics[clip, trim=2.35cm 1.75cm 4.5cm 0cm,width=\textwidth]{img/convergence/10342.pdf}
    \caption{10342}
    \label{fig:convergence_10342}
\end{subfigure}
%
\begin{subfigure}[b]{0.09\textwidth}
    \includegraphics[clip, trim=2.35cm 1.75cm 4.5cm 0cm,width=\textwidth]{img/convergence/10347.pdf}
    \caption{10347}
    \label{fig:convergence_10347}
\end{subfigure}
%
\begin{subfigure}[b]{0.09\textwidth}
    \includegraphics[clip, trim=2.35cm 1.75cm 4.5cm 0cm,width=\textwidth]{img/convergence/10348.pdf}
    \caption{10348}
    \label{fig:convergence_10348}
\end{subfigure}
%
\begin{subfigure}[b]{0.09\textwidth}
    \includegraphics[clip, trim=2.35cm 1.75cm 4.5cm 0cm,width=\textwidth]{img/convergence/10400.pdf}
    \caption{10400}
    \label{fig:convergence_10400}
\end{subfigure}
%
\begin{subfigure}[b]{0.09\textwidth}
    \includegraphics[clip, trim=2.35cm 1.75cm 4.5cm 0cm,width=\textwidth]{img/convergence/10471.pdf}
    \caption{10471}
    \label{fig:convergence_10471}
\end{subfigure}
%
\begin{subfigure}[b]{0.09\textwidth}
    \includegraphics[clip, trim=2.35cm 1.75cm 4.5cm 0cm,width=\textwidth]{img/convergence/10475.pdf}
    \caption{10475}
    \label{fig:convergence_10475}
\end{subfigure}
%
\begin{subfigure}[b]{0.09\textwidth}
    \includegraphics[clip, trim=2.35cm 1.75cm 4.5cm 0cm,width=\textwidth]{img/convergence/10483.pdf}
    \caption{10483}
    \label{fig:convergence_10483}
\end{subfigure}
%
\begin{subfigure}[b]{0.09\textwidth}
    \includegraphics[clip, trim=2.35cm 1.75cm 4.5cm 0cm,width=\textwidth]{img/convergence/10500.pdf}
    \caption{10500}
    \label{fig:convergence_10500}
\end{subfigure}
%
\begin{subfigure}[b]{0.09\textwidth}
    \includegraphics[clip, trim=2.35cm 1.75cm 4.5cm 0cm,width=\textwidth]{img/convergence/10600.pdf}
    \caption{10600}
    \label{fig:convergence_10600}
\end{subfigure}
%
\begin{subfigure}[b]{0.09\textwidth}
    \includegraphics[clip, trim=2.35cm 1.75cm 4.5cm 0cm,width=\textwidth]{img/convergence/10633.pdf}
    \caption{10633}
    \label{fig:convergence_10633}
\end{subfigure}
%
\begin{subfigure}[b]{0.09\textwidth}
    \includegraphics[clip, trim=2.35cm 1.75cm 4.5cm 0cm,width=\textwidth]{img/convergence/10634.pdf}
    \caption{10634}
    \label{fig:convergence_10634}
\end{subfigure}
%
\begin{subfigure}[b]{0.09\textwidth}
    \includegraphics[clip, trim=2.35cm 1.75cm 4.5cm 0cm,width=\textwidth]{img/convergence/10640.pdf}
    \caption{10640}
    \label{fig:convergence_10640}
\end{subfigure}
%
\begin{subfigure}[b]{0.09\textwidth}
    \includegraphics[clip, trim=2.35cm 1.75cm 4.5cm 0cm,width=\textwidth]{img/convergence/10655.pdf}
    \caption{10655}
    \label{fig:convergence_10655}
\end{subfigure}
%
\begin{subfigure}[b]{0.09\textwidth}
    \includegraphics[clip, trim=2.35cm 1.75cm 4.5cm 0cm,width=\textwidth]{img/convergence/10656.pdf}
    \caption{10656}
    \label{fig:convergence_10656}
\end{subfigure}
%
\begin{subfigure}[b]{0.09\textwidth}
    \includegraphics[clip, trim=2.35cm 1.75cm 4.5cm 0cm,width=\textwidth]{img/convergence/10700.pdf}
    \caption{10700}
    \label{fig:convergence_10700}
\end{subfigure}
%
\begin{subfigure}[b]{0.09\textwidth}
    \includegraphics[clip, trim=2.35cm 1.75cm 4.5cm 0cm,width=\textwidth]{img/convergence/10800.pdf}
    \caption{10800}
    \label{fig:convergence_10800}
\end{subfigure}
%
\begin{subfigure}[b]{0.09\textwidth}
    \includegraphics[clip, trim=2.35cm 1.75cm 4.5cm 0cm,width=\textwidth]{img/convergence/10900.pdf}
    \caption{10900}
    \label{fig:convergence_10900}
\end{subfigure}
%
\begin{subfigure}[b]{0.09\textwidth}
    \includegraphics[clip, trim=2.35cm 1.75cm 4.5cm 0cm,width=\textwidth]{img/convergence/11000.pdf}
    \caption{11000}
    \label{fig:convergence_11000}
\end{subfigure}
%
\begin{subfigure}[b]{0.09\textwidth}
    \includegraphics[clip, trim=2.35cm 1.75cm 4.5cm 0cm,width=\textwidth]{img/convergence/11100.pdf}
    \caption{11100}
    \label{fig:convergence_11100}
\end{subfigure}
%
\begin{subfigure}[b]{0.09\textwidth}
    \includegraphics[clip, trim=2.35cm 1.75cm 4.5cm 0cm,width=\textwidth]{img/convergence/11200.pdf}
    \caption{11200}
    \label{fig:convergence_11200}
\end{subfigure}
%
\begin{subfigure}[b]{0.09\textwidth}
    \includegraphics[clip, trim=2.35cm 1.75cm 4.5cm 0cm,width=\textwidth]{img/convergence/11300.pdf}
    \caption{11300}
    \label{fig:convergence_11300}
\end{subfigure}
%
\begin{subfigure}[b]{0.09\textwidth}
    \includegraphics[clip, trim=2.35cm 1.75cm 4.5cm 0cm,width=\textwidth]{img/convergence/11400.pdf}
    \caption{11400}
    \label{fig:convergence_11400}
\end{subfigure}
%
\begin{subfigure}[b]{0.09\textwidth}
    \includegraphics[clip, trim=2.35cm 1.75cm 4.5cm 0cm,width=\textwidth]{img/convergence/11500.pdf}
    \caption{11500}
    \label{fig:convergence_11500}
\end{subfigure}
%
\begin{subfigure}[b]{0.09\textwidth}
    \includegraphics[clip, trim=2.35cm 1.75cm 4.5cm 0cm,width=\textwidth]{img/convergence/11600.pdf}
    \caption{11600}
    \label{fig:convergence_11600}
\end{subfigure}
%
\begin{subfigure}[b]{0.09\textwidth}
    \includegraphics[clip, trim=2.35cm 1.75cm 4.5cm 0cm,width=\textwidth]{img/convergence/11700.pdf}
    \caption{11700}
    \label{fig:convergence_11700}
\end{subfigure}
%
\begin{subfigure}[b]{0.09\textwidth}
    \includegraphics[clip, trim=2.35cm 1.75cm 4.5cm 0cm,width=\textwidth]{img/convergence/11800.pdf}
    \caption{11800}
    \label{fig:convergence_11800}
\end{subfigure}
%
\begin{subfigure}[b]{0.09\textwidth}
    \includegraphics[clip, trim=2.35cm 1.75cm 4.5cm 0cm,width=\textwidth]{img/convergence/11881.pdf}
    \caption{11881}
    \label{fig:convergence_11881}
\end{subfigure}
%
\begin{subfigure}[b]{0.09\textwidth}
    \includegraphics[clip, trim=2.35cm 1.75cm 4.5cm 0cm,width=\textwidth]{img/convergence/11883.pdf}
    \caption{11883}
    \label{fig:convergence_11883}
\end{subfigure}
%
\begin{subfigure}[b]{0.09\textwidth}
    \includegraphics[clip, trim=2.35cm 1.75cm 4.5cm 0cm,width=\textwidth]{img/convergence/11884.pdf}
    \caption{11884}
    \label{fig:convergence_11884}
\end{subfigure}
%
\begin{subfigure}[b]{0.09\textwidth}
    \includegraphics[clip, trim=2.35cm 1.75cm 4.5cm 0cm,width=\textwidth]{img/convergence/11885.pdf}
    \caption{11885}
    \label{fig:convergence_11885}
\end{subfigure}
%
\begin{subfigure}[b]{0.09\textwidth}
    \includegraphics[clip, trim=2.35cm 1.75cm 4.5cm 0cm,width=\textwidth]{img/convergence/11886.pdf}
    \caption{11886}
    \label{fig:convergence_11886}
\end{subfigure}
%
\begin{subfigure}[b]{0.09\textwidth}
    \includegraphics[clip, trim=2.35cm 1.75cm 4.5cm 0cm,width=\textwidth]{img/convergence/11887.pdf}
    \caption{11887}
    \label{fig:convergence_11887}
\end{subfigure}
%
\begin{subfigure}[b]{0.09\textwidth}
    \includegraphics[clip, trim=2.35cm 1.75cm 4.5cm 0cm,width=\textwidth]{img/convergence/11888.pdf}
    \caption{11888}
    \label{fig:convergence_11888}
\end{subfigure}
%
\begin{subfigure}[b]{0.09\textwidth}
    \includegraphics[clip, trim=2.35cm 1.75cm 4.5cm 0cm,width=\textwidth]{img/convergence/11893.pdf}
    \caption{11893}
    \label{fig:convergence_11893}
\end{subfigure}
%
\begin{subfigure}[b]{0.09\textwidth}
    \includegraphics[clip, trim=2.35cm 1.75cm 4.5cm 0cm,width=\textwidth]{img/convergence/11894.pdf}
    \caption{11894}
    \label{fig:convergence_11894}
\end{subfigure}
%
\begin{subfigure}[b]{0.09\textwidth}
    \includegraphics[clip, trim=2.35cm 1.75cm 4.5cm 0cm,width=\textwidth]{img/convergence/11895.pdf}
    \caption{11895}
    \label{fig:convergence_11895}
\end{subfigure}
%
\begin{subfigure}[b]{0.09\textwidth}
    \includegraphics[clip, trim=2.35cm 1.75cm 4.5cm 0cm,width=\textwidth]{img/convergence/11896.pdf}
    \caption{11896}
    \label{fig:convergence_11896}
\end{subfigure}
%
\begin{subfigure}[b]{0.09\textwidth}
    \includegraphics[clip, trim=2.35cm 1.75cm 4.5cm 0cm,width=\textwidth]{img/convergence/11898.pdf}
    \caption{11898}
    \label{fig:convergence_11898}
\end{subfigure}
%
\begin{subfigure}[b]{0.09\textwidth}
    \includegraphics[clip, trim=2.35cm 1.75cm 4.5cm 0cm,width=\textwidth]{img/convergence/11899.pdf}
    \caption{11899}
    \label{fig:convergence_11899}
\end{subfigure}
%
\begin{subfigure}[b]{0.09\textwidth}
    \includegraphics[clip, trim=2.35cm 1.75cm 4.5cm 0cm,width=\textwidth]{img/convergence/11900.pdf}
    \caption{11900}
    \label{fig:convergence_11900}
\end{subfigure}
%
\begin{subfigure}[b]{0.09\textwidth}
    \includegraphics[clip, trim=2.35cm 1.75cm 4.5cm 0cm,width=\textwidth]{img/convergence/11902.pdf}
    \caption{11902}
    \label{fig:convergence_11902}
\end{subfigure}
%
\begin{subfigure}[b]{0.09\textwidth}
    \includegraphics[clip, trim=2.35cm 1.75cm 4.5cm 0cm,width=\textwidth]{img/convergence/11909.pdf}
    \caption{11909}
    \label{fig:convergence_11909}
\end{subfigure}
%
\begin{subfigure}[b]{0.09\textwidth}
    \includegraphics[clip, trim=2.35cm 1.75cm 4.5cm 0cm,width=\textwidth]{img/convergence/11910.pdf}
    \caption{11910}
    \label{fig:convergence_11910}
\end{subfigure}
%
\begin{subfigure}[b]{0.09\textwidth}
    \includegraphics[clip, trim=2.35cm 1.75cm 4.5cm 0cm,width=\textwidth]{img/convergence/11911.pdf}
    \caption{11911}
    \label{fig:convergence_11911}
\end{subfigure}
%
\begin{subfigure}[b]{0.09\textwidth}
    \includegraphics[clip, trim=2.35cm 1.75cm 4.5cm 0cm,width=\textwidth]{img/convergence/11916.pdf}
    \caption{11916}
    \label{fig:convergence_11916}
\end{subfigure}
%
\begin{subfigure}[b]{0.09\textwidth}
    \includegraphics[clip, trim=2.35cm 1.75cm 4.5cm 0cm,width=\textwidth]{img/convergence/11922.pdf}
    \caption{11922}
    \label{fig:convergence_11922}
\end{subfigure}
%
\begin{subfigure}[b]{0.09\textwidth}
    \includegraphics[clip, trim=2.35cm 1.75cm 4.5cm 0cm,width=\textwidth]{img/convergence/11927.pdf}
    \caption{11927}
    \label{fig:convergence_11927}
\end{subfigure}
%
\begin{subfigure}[b]{0.09\textwidth}
    \includegraphics[clip, trim=2.35cm 1.75cm 4.5cm 0cm,width=\textwidth]{img/convergence/11933.pdf}
    \caption{11933}
    \label{fig:convergence_11933}
\end{subfigure}
%
\begin{subfigure}[b]{0.09\textwidth}
    \includegraphics[clip, trim=2.35cm 1.75cm 4.5cm 0cm,width=\textwidth]{img/convergence/11937.pdf}
    \caption{11937}
    \label{fig:convergence_11937}
\end{subfigure}
%
\begin{subfigure}[b]{0.09\textwidth}
    \includegraphics[clip, trim=2.35cm 1.75cm 4.5cm 0cm,width=\textwidth]{img/convergence/11945.pdf}
    \caption{11945}
    \label{fig:convergence_11945}
\end{subfigure}
%
\begin{subfigure}[b]{0.09\textwidth}
    \includegraphics[clip, trim=2.35cm 1.75cm 4.5cm 0cm,width=\textwidth]{img/convergence/11947.pdf}
    \caption{11947}
    \label{fig:convergence_11947}
\end{subfigure}
%
\begin{subfigure}[b]{0.09\textwidth}
    \includegraphics[clip, trim=2.35cm 1.75cm 4.5cm 0cm,width=\textwidth]{img/convergence/11972.pdf}
    \caption{11972}
    \label{fig:convergence_11972}
\end{subfigure}
%
\begin{subfigure}[b]{0.09\textwidth}
    \includegraphics[clip, trim=2.35cm 1.75cm 4.5cm 0cm,width=\textwidth]{img/convergence/11973.pdf}
    \caption{11973}
    \label{fig:convergence_11973}
\end{subfigure}
%
\begin{subfigure}[b]{0.09\textwidth}
    \includegraphics[clip, trim=2.35cm 1.75cm 4.5cm 0cm,width=\textwidth]{img/convergence/11974.pdf}
    \caption{11974}
    \label{fig:convergence_11974}
\end{subfigure}
%
\begin{subfigure}[b]{0.09\textwidth}
    \includegraphics[clip, trim=2.35cm 1.75cm 4.5cm 0cm,width=\textwidth]{img/convergence/11977.pdf}
    \caption{11977}
    \label{fig:convergence_11977}
\end{subfigure}
%
\begin{subfigure}[b]{0.09\textwidth}
    \includegraphics[clip, trim=2.35cm 1.75cm 4.5cm 0cm,width=\textwidth]{img/convergence/11978.pdf}
    \caption{11978}
    \label{fig:convergence_11978}
\end{subfigure}
%
\begin{subfigure}[b]{0.09\textwidth}
    \includegraphics[clip, trim=2.35cm 1.75cm 4.5cm 0cm,width=\textwidth]{img/convergence/11979.pdf}
    \caption{11979}
    \label{fig:convergence_11979}
\end{subfigure}
%
\begin{subfigure}[b]{0.09\textwidth}
    \includegraphics[clip, trim=2.35cm 1.75cm 4.5cm 0cm,width=\textwidth]{img/convergence/11983.pdf}
    \caption{11983}
    \label{fig:convergence_11983}
\end{subfigure}
%
\begin{subfigure}[b]{0.09\textwidth}
    \includegraphics[clip, trim=2.35cm 1.75cm 4.5cm 0cm,width=\textwidth]{img/convergence/11984.pdf}
    \caption{11984}
    \label{fig:convergence_11984}
\end{subfigure}
%
\begin{subfigure}[b]{0.09\textwidth}
    \includegraphics[clip, trim=2.35cm 1.75cm 4.5cm 0cm,width=\textwidth]{img/convergence/11988.pdf}
    \caption{11988}
    \label{fig:convergence_11988}
\end{subfigure}
%
\begin{subfigure}[b]{0.09\textwidth}
    \includegraphics[clip, trim=2.35cm 1.75cm 4.5cm 0cm,width=\textwidth]{img/convergence/11989.pdf}
    \caption{11989}
    \label{fig:convergence_11989}
\end{subfigure}
%
\begin{subfigure}[b]{0.09\textwidth}
    \includegraphics[clip, trim=2.35cm 1.75cm 4.5cm 0cm,width=\textwidth]{img/convergence/11992.pdf}
    \caption{11992}
    \label{fig:convergence_11992}
\end{subfigure}
%
\begin{subfigure}[b]{0.09\textwidth}
    \includegraphics[clip, trim=2.35cm 1.75cm 4.5cm 0cm,width=\textwidth]{img/convergence/11993.pdf}
    \caption{11993}
    \label{fig:convergence_11993}
\end{subfigure}
%
\begin{subfigure}[b]{0.09\textwidth}
    \includegraphics[clip, trim=2.35cm 1.75cm 4.5cm 0cm,width=\textwidth]{img/convergence/11994.pdf}
    \caption{11994}
    \label{fig:convergence_11994}
\end{subfigure}
%
\begin{subfigure}[b]{0.09\textwidth}
    \includegraphics[clip, trim=2.35cm 1.75cm 4.5cm 0cm,width=\textwidth]{img/convergence/11995.pdf}
    \caption{11995}
    \label{fig:convergence_11995}
\end{subfigure}
%
\begin{subfigure}[b]{0.09\textwidth}
    \includegraphics[clip, trim=2.35cm 1.75cm 4.5cm 0cm,width=\textwidth]{img/convergence/11997.pdf}
    \caption{11997}
    \label{fig:convergence_11997}
\end{subfigure}
%
\begin{subfigure}[b]{0.09\textwidth}
    \includegraphics[clip, trim=2.35cm 1.75cm 4.5cm 0cm,width=\textwidth]{img/convergence/11998.pdf}
    \caption{11998}
    \label{fig:convergence_11998}
\end{subfigure}
%
\begin{subfigure}[b]{0.09\textwidth}
    \includegraphics[clip, trim=2.35cm 1.75cm 4.5cm 0cm,width=\textwidth]{img/convergence/11999.pdf}
    \caption{11999}
    \label{fig:convergence_11999}
\end{subfigure}
%
\begin{subfigure}[b]{0.09\textwidth}
    \includegraphics[clip, trim=2.35cm 1.75cm 4.5cm 0cm,width=\textwidth]{img/convergence/12000.pdf}
    \caption{12000}
    \label{fig:convergence_12000}
\end{subfigure}
%
\begin{subfigure}[b]{0.09\textwidth}
    \includegraphics[clip, trim=2.35cm 1.75cm 4.5cm 0cm,width=\textwidth]{img/convergence/12001.pdf}
    \caption{12001}
    \label{fig:convergence_12001}
\end{subfigure}
%
\begin{subfigure}[b]{0.09\textwidth}
    \includegraphics[clip, trim=2.35cm 1.75cm 4.5cm 0cm,width=\textwidth]{img/convergence/12002.pdf}
    \caption{12002}
    \label{fig:convergence_12002}
\end{subfigure}
%
\begin{subfigure}[b]{0.09\textwidth}
    \includegraphics[clip, trim=2.35cm 1.75cm 4.5cm 0cm,width=\textwidth]{img/convergence/12024.pdf}
    \caption{12024}
    \label{fig:convergence_12024}
\end{subfigure}
%
\begin{subfigure}[b]{0.09\textwidth}
    \includegraphics[clip, trim=2.35cm 1.75cm 4.5cm 0cm,width=\textwidth]{img/convergence/12033.pdf}
    \caption{12033}
    \label{fig:convergence_12033}
\end{subfigure}
%
\begin{subfigure}[b]{0.09\textwidth}
    \includegraphics[clip, trim=2.35cm 1.75cm 4.5cm 0cm,width=\textwidth]{img/convergence/12046.pdf}
    \caption{12046}
    \label{fig:convergence_12046}
\end{subfigure}
%
\begin{subfigure}[b]{0.09\textwidth}
    \includegraphics[clip, trim=2.35cm 1.75cm 4.5cm 0cm,width=\textwidth]{img/convergence/12047.pdf}
    \caption{12047}
    \label{fig:convergence_12047}
\end{subfigure}
%
\begin{subfigure}[b]{0.09\textwidth}
    \includegraphics[clip, trim=2.35cm 1.75cm 4.5cm 0cm,width=\textwidth]{img/convergence/12050.pdf}
    \caption{12050}
    \label{fig:convergence_12050}
\end{subfigure}
%
\begin{subfigure}[b]{0.09\textwidth}
    \includegraphics[clip, trim=2.35cm 1.75cm 4.5cm 0cm,width=\textwidth]{img/convergence/12051.pdf}
    \caption{12051}
    \label{fig:convergence_12051}
\end{subfigure}
%
\begin{subfigure}[b]{0.09\textwidth}
    \includegraphics[clip, trim=2.35cm 1.75cm 4.5cm 0cm,width=\textwidth]{img/convergence/12053.pdf}
    \caption{12053}
    \label{fig:convergence_12053}
\end{subfigure}
%
\begin{subfigure}[b]{0.09\textwidth}
    \includegraphics[clip, trim=2.35cm 1.75cm 4.5cm 0cm,width=\textwidth]{img/convergence/12054.pdf}
    \caption{12054}
    \label{fig:convergence_12054}
\end{subfigure}
%
\begin{subfigure}[b]{0.09\textwidth}
    \includegraphics[clip, trim=2.35cm 1.75cm 4.5cm 0cm,width=\textwidth]{img/convergence/12055.pdf}
    \caption{12055}
    \label{fig:convergence_12055}
\end{subfigure}
%
\begin{subfigure}[b]{0.09\textwidth}
    \includegraphics[clip, trim=2.35cm 1.75cm 4.5cm 0cm,width=\textwidth]{img/convergence/12057.pdf}
    \caption{12057}
    \label{fig:convergence_12057}
\end{subfigure}
%
\begin{subfigure}[b]{0.09\textwidth}
    \includegraphics[clip, trim=2.35cm 1.75cm 4.5cm 0cm,width=\textwidth]{img/convergence/12058.pdf}
    \caption{12058}
    \label{fig:convergence_12058}
\end{subfigure}
%
\begin{subfigure}[b]{0.09\textwidth}
    \includegraphics[clip, trim=2.35cm 1.75cm 4.5cm 0cm,width=\textwidth]{img/convergence/12059.pdf}
    \caption{12059}
    \label{fig:convergence_12059}
\end{subfigure}
%
\begin{subfigure}[b]{0.09\textwidth}
    \includegraphics[clip, trim=2.35cm 1.75cm 4.5cm 0cm,width=\textwidth]{img/convergence/12060.pdf}
    \caption{12060}
    \label{fig:convergence_12060}
\end{subfigure}
%
\begin{subfigure}[b]{0.09\textwidth}
    \includegraphics[clip, trim=2.35cm 1.75cm 4.5cm 0cm,width=\textwidth]{img/convergence/12063.pdf}
    \caption{12063}
    \label{fig:convergence_12063}
\end{subfigure}
%
\begin{subfigure}[b]{0.09\textwidth}
    \includegraphics[clip, trim=2.35cm 1.75cm 4.5cm 0cm,width=\textwidth]{img/convergence/12064.pdf}
    \caption{12064}
    \label{fig:convergence_12064}
\end{subfigure}
%
\begin{subfigure}[b]{0.09\textwidth}
    \includegraphics[clip, trim=2.35cm 1.75cm 4.5cm 0cm,width=\textwidth]{img/convergence/12067.pdf}
    \caption{12067}
    \label{fig:convergence_12067}
\end{subfigure}
%
\begin{subfigure}[b]{0.09\textwidth}
    \includegraphics[clip, trim=2.35cm 1.75cm 4.5cm 0cm,width=\textwidth]{img/convergence/12068.pdf}
    \caption{12068}
    \label{fig:convergence_12068}
\end{subfigure}
%
\begin{subfigure}[b]{0.09\textwidth}
    \includegraphics[clip, trim=2.35cm 1.75cm 4.5cm 0cm,width=\textwidth]{img/convergence/12100.pdf}
    \caption{12100}
    \label{fig:convergence_12100}
\end{subfigure}
%
\begin{subfigure}[b]{0.09\textwidth}
    \includegraphics[clip, trim=2.35cm 1.75cm 4.5cm 0cm,width=\textwidth]{img/convergence/12103.pdf}
    \caption{12103}
    \label{fig:convergence_12103}
\end{subfigure}
%
\begin{subfigure}[b]{0.09\textwidth}
    \includegraphics[clip, trim=2.35cm 1.75cm 4.5cm 0cm,width=\textwidth]{img/convergence/12104.pdf}
    \caption{12104}
    \label{fig:convergence_12104}
\end{subfigure}
%
\begin{subfigure}[b]{0.09\textwidth}
    \includegraphics[clip, trim=2.35cm 1.75cm 4.5cm 0cm,width=\textwidth]{img/convergence/12105.pdf}
    \caption{12105}
    \label{fig:convergence_12105}
\end{subfigure}
%
\begin{subfigure}[b]{0.09\textwidth}
    \includegraphics[clip, trim=2.35cm 1.75cm 4.5cm 0cm,width=\textwidth]{img/convergence/12106.pdf}
    \caption{12106}
    \label{fig:convergence_12106}
\end{subfigure}
%
\begin{subfigure}[b]{0.09\textwidth}
    \includegraphics[clip, trim=2.35cm 1.75cm 4.5cm 0cm,width=\textwidth]{img/convergence/12107.pdf}
    \caption{12107}
    \label{fig:convergence_12107}
\end{subfigure}
%
\begin{subfigure}[b]{0.09\textwidth}
    \includegraphics[clip, trim=2.35cm 1.75cm 4.5cm 0cm,width=\textwidth]{img/convergence/12108.pdf}
    \caption{12108}
    \label{fig:convergence_12108}
\end{subfigure}
%
\begin{subfigure}[b]{0.09\textwidth}
    \includegraphics[clip, trim=2.35cm 1.75cm 4.5cm 0cm,width=\textwidth]{img/convergence/12109.pdf}
    \caption{12109}
    \label{fig:convergence_12109}
\end{subfigure}
%
\begin{subfigure}[b]{0.09\textwidth}
    \includegraphics[clip, trim=2.35cm 1.75cm 4.5cm 0cm,width=\textwidth]{img/convergence/12110.pdf}
    \caption{12110}
    \label{fig:convergence_12110}
\end{subfigure}
%
\begin{subfigure}[b]{0.09\textwidth}
    \includegraphics[clip, trim=2.35cm 1.75cm 4.5cm 0cm,width=\textwidth]{img/convergence/12111.pdf}
    \caption{12111}
    \label{fig:convergence_12111}
\end{subfigure}
%
\begin{subfigure}[b]{0.09\textwidth}
    \includegraphics[clip, trim=2.35cm 1.75cm 4.5cm 0cm,width=\textwidth]{img/convergence/12112.pdf}
    \caption{12112}
    \label{fig:convergence_12112}
\end{subfigure}
%
\begin{subfigure}[b]{0.09\textwidth}
    \includegraphics[clip, trim=2.35cm 1.75cm 4.5cm 0cm,width=\textwidth]{img/convergence/12113.pdf}
    \caption{12113}
    \label{fig:convergence_12113}
\end{subfigure}
%
\begin{subfigure}[b]{0.09\textwidth}
    \includegraphics[clip, trim=2.35cm 1.75cm 4.5cm 0cm,width=\textwidth]{img/convergence/12115.pdf}
    \caption{12115}
    \label{fig:convergence_12115}
\end{subfigure}
%
\begin{subfigure}[b]{0.09\textwidth}
    \includegraphics[clip, trim=2.35cm 1.75cm 4.5cm 0cm,width=\textwidth]{img/convergence/12116.pdf}
    \caption{12116}
    \label{fig:convergence_12116}
\end{subfigure}
%
\begin{subfigure}[b]{0.09\textwidth}
    \includegraphics[clip, trim=2.35cm 1.75cm 4.5cm 0cm,width=\textwidth]{img/convergence/12117.pdf}
    \caption{12117}
    \label{fig:convergence_12117}
\end{subfigure}
%
\begin{subfigure}[b]{0.09\textwidth}
    \includegraphics[clip, trim=2.35cm 1.75cm 4.5cm 0cm,width=\textwidth]{img/convergence/12118.pdf}
    \caption{12118}
    \label{fig:convergence_12118}
\end{subfigure}
%
\begin{subfigure}[b]{0.09\textwidth}
    \includegraphics[clip, trim=2.35cm 1.75cm 4.5cm 0cm,width=\textwidth]{img/convergence/12119.pdf}
    \caption{12119}
    \label{fig:convergence_12119}
\end{subfigure}
%
\begin{subfigure}[b]{0.09\textwidth}
    \includegraphics[clip, trim=2.35cm 1.75cm 4.5cm 0cm,width=\textwidth]{img/convergence/12120.pdf}
    \caption{12120}
    \label{fig:convergence_12120}
\end{subfigure}
%
\begin{subfigure}[b]{0.09\textwidth}
    \includegraphics[clip, trim=2.35cm 1.75cm 4.5cm 0cm,width=\textwidth]{img/convergence/12121.pdf}
    \caption{12121}
    \label{fig:convergence_12121}
\end{subfigure}
%
\begin{subfigure}[b]{0.09\textwidth}
    \includegraphics[clip, trim=2.35cm 1.75cm 4.5cm 0cm,width=\textwidth]{img/convergence/12122.pdf}
    \caption{12122}
    \label{fig:convergence_12122}
\end{subfigure}
%
\begin{subfigure}[b]{0.09\textwidth}
    \includegraphics[clip, trim=2.35cm 1.75cm 4.5cm 0cm,width=\textwidth]{img/convergence/12124.pdf}
    \caption{12124}
    \label{fig:convergence_12124}
\end{subfigure}
%
\begin{subfigure}[b]{0.09\textwidth}
    \includegraphics[clip, trim=2.35cm 1.75cm 4.5cm 0cm,width=\textwidth]{img/convergence/12131.pdf}
    \caption{12131}
    \label{fig:convergence_12131}
\end{subfigure}
%
\begin{subfigure}[b]{0.09\textwidth}
    \includegraphics[clip, trim=2.35cm 1.75cm 4.5cm 0cm,width=\textwidth]{img/convergence/12137.pdf}
    \caption{12137}
    \label{fig:convergence_12137}
\end{subfigure}
%
\begin{subfigure}[b]{0.09\textwidth}
    \includegraphics[clip, trim=2.35cm 1.75cm 4.5cm 0cm,width=\textwidth]{img/convergence/12143.pdf}
    \caption{12143}
    \label{fig:convergence_12143}
\end{subfigure}
%
\begin{subfigure}[b]{0.09\textwidth}
    \includegraphics[clip, trim=2.35cm 1.75cm 4.5cm 0cm,width=\textwidth]{img/convergence/12150.pdf}
    \caption{12150}
    \label{fig:convergence_12150}
\end{subfigure}
%
\begin{subfigure}[b]{0.09\textwidth}
    \includegraphics[clip, trim=2.35cm 1.75cm 4.5cm 0cm,width=\textwidth]{img/convergence/12156.pdf}
    \caption{12156}
    \label{fig:convergence_12156}
\end{subfigure}
%
\begin{subfigure}[b]{0.09\textwidth}
    \includegraphics[clip, trim=2.35cm 1.75cm 4.5cm 0cm,width=\textwidth]{img/convergence/12164.pdf}
    \caption{12164}
    \label{fig:convergence_12164}
\end{subfigure}
%
\begin{subfigure}[b]{0.09\textwidth}
    \includegraphics[clip, trim=2.35cm 1.75cm 4.5cm 0cm,width=\textwidth]{img/convergence/12168.pdf}
    \caption{12168}
    \label{fig:convergence_12168}
\end{subfigure}
%
\begin{subfigure}[b]{0.09\textwidth}
    \includegraphics[clip, trim=2.35cm 1.75cm 4.5cm 0cm,width=\textwidth]{img/convergence/12178.pdf}
    \caption{12178}
    \label{fig:convergence_12178}
\end{subfigure}
%
\begin{subfigure}[b]{0.09\textwidth}
    \includegraphics[clip, trim=2.35cm 1.75cm 4.5cm 0cm,width=\textwidth]{img/convergence/12179.pdf}
    \caption{12179}
    \label{fig:convergence_12179}
\end{subfigure}
%
\begin{subfigure}[b]{0.09\textwidth}
    \includegraphics[clip, trim=2.35cm 1.75cm 4.5cm 0cm,width=\textwidth]{img/convergence/12200.pdf}
    \caption{12200}
    \label{fig:convergence_12200}
\end{subfigure}
%
\begin{subfigure}[b]{0.09\textwidth}
    \includegraphics[clip, trim=2.35cm 1.75cm 4.5cm 0cm,width=\textwidth]{img/convergence/12217.pdf}
    \caption{12217}
    \label{fig:convergence_12217}
\end{subfigure}
%
\begin{subfigure}[b]{0.09\textwidth}
    \includegraphics[clip, trim=2.35cm 1.75cm 4.5cm 0cm,width=\textwidth]{img/convergence/12218.pdf}
    \caption{12218}
    \label{fig:convergence_12218}
\end{subfigure}
%
\begin{subfigure}[b]{0.09\textwidth}
    \includegraphics[clip, trim=2.35cm 1.75cm 4.5cm 0cm,width=\textwidth]{img/convergence/12219.pdf}
    \caption{12219}
    \label{fig:convergence_12219}
\end{subfigure}
%
\begin{subfigure}[b]{0.09\textwidth}
    \includegraphics[clip, trim=2.35cm 1.75cm 4.5cm 0cm,width=\textwidth]{img/convergence/12220.pdf}
    \caption{12220}
    \label{fig:convergence_12220}
\end{subfigure}
%
\begin{subfigure}[b]{0.09\textwidth}
    \includegraphics[clip, trim=2.35cm 1.75cm 4.5cm 0cm,width=\textwidth]{img/convergence/12221.pdf}
    \caption{12221}
    \label{fig:convergence_12221}
\end{subfigure}
%
\begin{subfigure}[b]{0.09\textwidth}
    \includegraphics[clip, trim=2.35cm 1.75cm 4.5cm 0cm,width=\textwidth]{img/convergence/12223.pdf}
    \caption{12223}
    \label{fig:convergence_12223}
\end{subfigure}
%
\begin{subfigure}[b]{0.09\textwidth}
    \includegraphics[clip, trim=2.35cm 1.75cm 4.5cm 0cm,width=\textwidth]{img/convergence/12224.pdf}
    \caption{12224}
    \label{fig:convergence_12224}
\end{subfigure}
%
\begin{subfigure}[b]{0.09\textwidth}
    \includegraphics[clip, trim=2.35cm 1.75cm 4.5cm 0cm,width=\textwidth]{img/convergence/12225.pdf}
    \caption{12225}
    \label{fig:convergence_12225}
\end{subfigure}
%
\begin{subfigure}[b]{0.09\textwidth}
    \includegraphics[clip, trim=2.35cm 1.75cm 4.5cm 0cm,width=\textwidth]{img/convergence/12231.pdf}
    \caption{12231}
    \label{fig:convergence_12231}
\end{subfigure}
%
\begin{subfigure}[b]{0.09\textwidth}
    \includegraphics[clip, trim=2.35cm 1.75cm 4.5cm 0cm,width=\textwidth]{img/convergence/12232.pdf}
    \caption{12232}
    \label{fig:convergence_12232}
\end{subfigure}
%
\begin{subfigure}[b]{0.09\textwidth}
    \includegraphics[clip, trim=2.35cm 1.75cm 4.5cm 0cm,width=\textwidth]{img/convergence/12233.pdf}
    \caption{12233}
    \label{fig:convergence_12233}
\end{subfigure}
%
\begin{subfigure}[b]{0.09\textwidth}
    \includegraphics[clip, trim=2.35cm 1.75cm 4.5cm 0cm,width=\textwidth]{img/convergence/12234.pdf}
    \caption{12234}
    \label{fig:convergence_12234}
\end{subfigure}
%
\begin{subfigure}[b]{0.09\textwidth}
    \includegraphics[clip, trim=2.35cm 1.75cm 4.5cm 0cm,width=\textwidth]{img/convergence/12235.pdf}
    \caption{12235}
    \label{fig:convergence_12235}
\end{subfigure}
%
\begin{subfigure}[b]{0.09\textwidth}
    \includegraphics[clip, trim=2.35cm 1.75cm 4.5cm 0cm,width=\textwidth]{img/convergence/12239.pdf}
    \caption{12239}
    \label{fig:convergence_12239}
\end{subfigure}
%
\begin{subfigure}[b]{0.09\textwidth}
    \includegraphics[clip, trim=2.35cm 1.75cm 4.5cm 0cm,width=\textwidth]{img/convergence/12240.pdf}
    \caption{12240}
    \label{fig:convergence_12240}
\end{subfigure}
%
\begin{subfigure}[b]{0.09\textwidth}
    \includegraphics[clip, trim=2.35cm 1.75cm 4.5cm 0cm,width=\textwidth]{img/convergence/12241.pdf}
    \caption{12241}
    \label{fig:convergence_12241}
\end{subfigure}
%
\begin{subfigure}[b]{0.09\textwidth}
    \includegraphics[clip, trim=2.35cm 1.75cm 4.5cm 0cm,width=\textwidth]{img/convergence/12242.pdf}
    \caption{12242}
    \label{fig:convergence_12242}
\end{subfigure}
%
\begin{subfigure}[b]{0.09\textwidth}
    \includegraphics[clip, trim=2.35cm 1.75cm 4.5cm 0cm,width=\textwidth]{img/convergence/12245.pdf}
    \caption{12245}
    \label{fig:convergence_12245}
\end{subfigure}
%
\begin{subfigure}[b]{0.09\textwidth}
    \includegraphics[clip, trim=2.35cm 1.75cm 4.5cm 0cm,width=\textwidth]{img/convergence/12251.pdf}
    \caption{12251}
    \label{fig:convergence_12251}
\end{subfigure}
%
\begin{subfigure}[b]{0.09\textwidth}
    \includegraphics[clip, trim=2.35cm 1.75cm 4.5cm 0cm,width=\textwidth]{img/convergence/12255.pdf}
    \caption{12255}
    \label{fig:convergence_12255}
\end{subfigure}
%
\begin{subfigure}[b]{0.09\textwidth}
    \includegraphics[clip, trim=2.35cm 1.75cm 4.5cm 0cm,width=\textwidth]{img/convergence/12260.pdf}
    \caption{12260}
    \label{fig:convergence_12260}
\end{subfigure}
%
\begin{subfigure}[b]{0.09\textwidth}
    \includegraphics[clip, trim=2.35cm 1.75cm 4.5cm 0cm,width=\textwidth]{img/convergence/12267.pdf}
    \caption{12267}
    \label{fig:convergence_12267}
\end{subfigure}
%
\begin{subfigure}[b]{0.09\textwidth}
    \includegraphics[clip, trim=2.35cm 1.75cm 4.5cm 0cm,width=\textwidth]{img/convergence/12268.pdf}
    \caption{12268}
    \label{fig:convergence_12268}
\end{subfigure}
%
\begin{subfigure}[b]{0.09\textwidth}
    \includegraphics[clip, trim=2.35cm 1.75cm 4.5cm 0cm,width=\textwidth]{img/convergence/12269.pdf}
    \caption{12269}
    \label{fig:convergence_12269}
\end{subfigure}
%
\begin{subfigure}[b]{0.09\textwidth}
    \includegraphics[clip, trim=2.35cm 1.75cm 4.5cm 0cm,width=\textwidth]{img/convergence/12270.pdf}
    \caption{12270}
    \label{fig:convergence_12270}
\end{subfigure}
%
\begin{subfigure}[b]{0.09\textwidth}
    \includegraphics[clip, trim=2.35cm 1.75cm 4.5cm 0cm,width=\textwidth]{img/convergence/12272.pdf}
    \caption{12272}
    \label{fig:convergence_12272}
\end{subfigure}
%
\begin{subfigure}[b]{0.09\textwidth}
    \includegraphics[clip, trim=2.35cm 1.75cm 4.5cm 0cm,width=\textwidth]{img/convergence/12276.pdf}
    \caption{12276}
    \label{fig:convergence_12276}
\end{subfigure}
%
\begin{subfigure}[b]{0.09\textwidth}
    \includegraphics[clip, trim=2.35cm 1.75cm 4.5cm 0cm,width=\textwidth]{img/convergence/12277.pdf}
    \caption{12277}
    \label{fig:convergence_12277}
\end{subfigure}
%
\begin{subfigure}[b]{0.09\textwidth}
    \includegraphics[clip, trim=2.35cm 1.75cm 4.5cm 0cm,width=\textwidth]{img/convergence/12280.pdf}
    \caption{12280}
    \label{fig:convergence_12280}
\end{subfigure}
%
\begin{subfigure}[b]{0.09\textwidth}
    \includegraphics[clip, trim=2.35cm 1.75cm 4.5cm 0cm,width=\textwidth]{img/convergence/12281.pdf}
    \caption{12281}
    \label{fig:convergence_12281}
\end{subfigure}
%
\begin{subfigure}[b]{0.09\textwidth}
    \includegraphics[clip, trim=2.35cm 1.75cm 4.5cm 0cm,width=\textwidth]{img/convergence/12282.pdf}
    \caption{12282}
    \label{fig:convergence_12282}
\end{subfigure}
%
\begin{subfigure}[b]{0.09\textwidth}
    \includegraphics[clip, trim=2.35cm 1.75cm 4.5cm 0cm,width=\textwidth]{img/convergence/12283.pdf}
    \caption{12283}
    \label{fig:convergence_12283}
\end{subfigure}
%
\begin{subfigure}[b]{0.09\textwidth}
    \includegraphics[clip, trim=2.35cm 1.75cm 4.5cm 0cm,width=\textwidth]{img/convergence/12284.pdf}
    \caption{12284}
    \label{fig:convergence_12284}
\end{subfigure}
%
\begin{subfigure}[b]{0.09\textwidth}
    \includegraphics[clip, trim=2.35cm 1.75cm 4.5cm 0cm,width=\textwidth]{img/convergence/12285.pdf}
    \caption{12285}
    \label{fig:convergence_12285}
\end{subfigure}
%
\begin{subfigure}[b]{0.09\textwidth}
    \includegraphics[clip, trim=2.35cm 1.75cm 4.5cm 0cm,width=\textwidth]{img/convergence/12286.pdf}
    \caption{12286}
    \label{fig:convergence_12286}
\end{subfigure}
%
\begin{subfigure}[b]{0.09\textwidth}
    \includegraphics[clip, trim=2.35cm 1.75cm 4.5cm 0cm,width=\textwidth]{img/convergence/12289.pdf}
    \caption{12289}
    \label{fig:convergence_12289}
\end{subfigure}
%
\begin{subfigure}[b]{0.09\textwidth}
    \includegraphics[clip, trim=2.35cm 1.75cm 4.5cm 0cm,width=\textwidth]{img/convergence/12294.pdf}
    \caption{12294}
    \label{fig:convergence_12294}
\end{subfigure}
%
\begin{subfigure}[b]{0.09\textwidth}
    \includegraphics[clip, trim=2.35cm 1.75cm 4.5cm 0cm,width=\textwidth]{img/convergence/12299.pdf}
    \caption{12299}
    \label{fig:convergence_12299}
\end{subfigure}
%
\begin{subfigure}[b]{0.09\textwidth}
    \includegraphics[clip, trim=2.35cm 1.75cm 4.5cm 0cm,width=\textwidth]{img/convergence/12300.pdf}
    \caption{12300}
    \label{fig:convergence_12300}
\end{subfigure}
%
\begin{subfigure}[b]{0.09\textwidth}
    \includegraphics[clip, trim=2.35cm 1.75cm 4.5cm 0cm,width=\textwidth]{img/convergence/12305.pdf}
    \caption{12305}
    \label{fig:convergence_12305}
\end{subfigure}
%
\begin{subfigure}[b]{0.09\textwidth}
    \includegraphics[clip, trim=2.35cm 1.75cm 4.5cm 0cm,width=\textwidth]{img/convergence/12310.pdf}
    \caption{12310}
    \label{fig:convergence_12310}
\end{subfigure}
%
\begin{subfigure}[b]{0.09\textwidth}
    \includegraphics[clip, trim=2.35cm 1.75cm 4.5cm 0cm,width=\textwidth]{img/convergence/12317.pdf}
    \caption{12317}
    \label{fig:convergence_12317}
\end{subfigure}
%
\begin{subfigure}[b]{0.09\textwidth}
    \includegraphics[clip, trim=2.35cm 1.75cm 4.5cm 0cm,width=\textwidth]{img/convergence/12322.pdf}
    \caption{12322}
    \label{fig:convergence_12322}
\end{subfigure}
%
\begin{subfigure}[b]{0.09\textwidth}
    \includegraphics[clip, trim=2.35cm 1.75cm 4.5cm 0cm,width=\textwidth]{img/convergence/12323.pdf}
    \caption{12323}
    \label{fig:convergence_12323}
\end{subfigure}
%
\begin{subfigure}[b]{0.09\textwidth}
    \includegraphics[clip, trim=2.35cm 1.75cm 4.5cm 0cm,width=\textwidth]{img/convergence/12328.pdf}
    \caption{12328}
    \label{fig:convergence_12328}
\end{subfigure}
%
\begin{subfigure}[b]{0.09\textwidth}
    \includegraphics[clip, trim=2.35cm 1.75cm 4.5cm 0cm,width=\textwidth]{img/convergence/12329.pdf}
    \caption{12329}
    \label{fig:convergence_12329}
\end{subfigure}
%
\begin{subfigure}[b]{0.09\textwidth}
    \includegraphics[clip, trim=2.35cm 1.75cm 4.5cm 0cm,width=\textwidth]{img/convergence/12335.pdf}
    \caption{12335}
    \label{fig:convergence_12335}
\end{subfigure}
%
\begin{subfigure}[b]{0.09\textwidth}
    \includegraphics[clip, trim=2.35cm 1.75cm 4.5cm 0cm,width=\textwidth]{img/convergence/12336.pdf}
    \caption{12336}
    \label{fig:convergence_12336}
\end{subfigure}
%
\begin{subfigure}[b]{0.09\textwidth}
    \includegraphics[clip, trim=2.35cm 1.75cm 4.5cm 0cm,width=\textwidth]{img/convergence/12337.pdf}
    \caption{12337}
    \label{fig:convergence_12337}
\end{subfigure}
%
\begin{subfigure}[b]{0.09\textwidth}
    \includegraphics[clip, trim=2.35cm 1.75cm 4.5cm 0cm,width=\textwidth]{img/convergence/12338.pdf}
    \caption{12338}
    \label{fig:convergence_12338}
\end{subfigure}
%
\begin{subfigure}[b]{0.09\textwidth}
    \includegraphics[clip, trim=2.35cm 1.75cm 4.5cm 0cm,width=\textwidth]{img/convergence/12339.pdf}
    \caption{12339}
    \label{fig:convergence_12339}
\end{subfigure}
%
\begin{subfigure}[b]{0.09\textwidth}
    \includegraphics[clip, trim=2.35cm 1.75cm 4.5cm 0cm,width=\textwidth]{img/convergence/12344.pdf}
    \caption{12344}
    \label{fig:convergence_12344}
\end{subfigure}
%
\begin{subfigure}[b]{0.09\textwidth}
    \includegraphics[clip, trim=2.35cm 1.75cm 4.5cm 0cm,width=\textwidth]{img/convergence/12345.pdf}
    \caption{12345}
    \label{fig:convergence_12345}
\end{subfigure}
%
\begin{subfigure}[b]{0.09\textwidth}
    \includegraphics[clip, trim=2.35cm 1.75cm 4.5cm 0cm,width=\textwidth]{img/convergence/12346.pdf}
    \caption{12346}
    \label{fig:convergence_12346}
\end{subfigure}
%
\begin{subfigure}[b]{0.09\textwidth}
    \includegraphics[clip, trim=2.35cm 1.75cm 4.5cm 0cm,width=\textwidth]{img/convergence/12347.pdf}
    \caption{12347}
    \label{fig:convergence_12347}
\end{subfigure}
%
\begin{subfigure}[b]{0.09\textwidth}
    \includegraphics[clip, trim=2.35cm 1.75cm 4.5cm 0cm,width=\textwidth]{img/convergence/12348.pdf}
    \caption{12348}
    \label{fig:convergence_12348}
\end{subfigure}
%
\begin{subfigure}[b]{0.09\textwidth}
    \includegraphics[clip, trim=2.35cm 1.75cm 4.5cm 0cm,width=\textwidth]{img/convergence/12354.pdf}
    \caption{12354}
    \label{fig:convergence_12354}
\end{subfigure}
%
\begin{subfigure}[b]{0.09\textwidth}
    \includegraphics[clip, trim=2.35cm 1.75cm 4.5cm 0cm,width=\textwidth]{img/convergence/12400.pdf}
    \caption{12400}
    \label{fig:convergence_12400}
\end{subfigure}
%
\begin{subfigure}[b]{0.09\textwidth}
    \includegraphics[clip, trim=2.35cm 1.75cm 4.5cm 0cm,width=\textwidth]{img/convergence/12402.pdf}
    \caption{12402}
    \label{fig:convergence_12402}
\end{subfigure}
%
\begin{subfigure}[b]{0.09\textwidth}
    \includegraphics[clip, trim=2.35cm 1.75cm 4.5cm 0cm,width=\textwidth]{img/convergence/12404.pdf}
    \caption{12404}
    \label{fig:convergence_12404}
\end{subfigure}
%
\begin{subfigure}[b]{0.09\textwidth}
    \includegraphics[clip, trim=2.35cm 1.75cm 4.5cm 0cm,width=\textwidth]{img/convergence/12405.pdf}
    \caption{12405}
    \label{fig:convergence_12405}
\end{subfigure}
%
\begin{subfigure}[b]{0.09\textwidth}
    \includegraphics[clip, trim=2.35cm 1.75cm 4.5cm 0cm,width=\textwidth]{img/convergence/12408.pdf}
    \caption{12408}
    \label{fig:convergence_12408}
\end{subfigure}
%
\begin{subfigure}[b]{0.09\textwidth}
    \includegraphics[clip, trim=2.35cm 1.75cm 4.5cm 0cm,width=\textwidth]{img/convergence/12409.pdf}
    \caption{12409}
    \label{fig:convergence_12409}
\end{subfigure}
%
\begin{subfigure}[b]{0.09\textwidth}
    \includegraphics[clip, trim=2.35cm 1.75cm 4.5cm 0cm,width=\textwidth]{img/convergence/12410.pdf}
    \caption{12410}
    \label{fig:convergence_12410}
\end{subfigure}
%
\begin{subfigure}[b]{0.09\textwidth}
    \includegraphics[clip, trim=2.35cm 1.75cm 4.5cm 0cm,width=\textwidth]{img/convergence/12411.pdf}
    \caption{12411}
    \label{fig:convergence_12411}
\end{subfigure}
%
\begin{subfigure}[b]{0.09\textwidth}
    \includegraphics[clip, trim=2.35cm 1.75cm 4.5cm 0cm,width=\textwidth]{img/convergence/12413.pdf}
    \caption{12413}
    \label{fig:convergence_12413}
\end{subfigure}
%
\begin{subfigure}[b]{0.09\textwidth}
    \includegraphics[clip, trim=2.35cm 1.75cm 4.5cm 0cm,width=\textwidth]{img/convergence/12415.pdf}
    \caption{12415}
    \label{fig:convergence_12415}
\end{subfigure}
%
\begin{subfigure}[b]{0.09\textwidth}
    \includegraphics[clip, trim=2.35cm 1.75cm 4.5cm 0cm,width=\textwidth]{img/convergence/12416.pdf}
    \caption{12416}
    \label{fig:convergence_12416}
\end{subfigure}
%
\begin{subfigure}[b]{0.09\textwidth}
    \includegraphics[clip, trim=2.35cm 1.75cm 4.5cm 0cm,width=\textwidth]{img/convergence/12417.pdf}
    \caption{12417}
    \label{fig:convergence_12417}
\end{subfigure}
%
\begin{subfigure}[b]{0.09\textwidth}
    \includegraphics[clip, trim=2.35cm 1.75cm 4.5cm 0cm,width=\textwidth]{img/convergence/12418.pdf}
    \caption{12418}
    \label{fig:convergence_12418}
\end{subfigure}
%
\begin{subfigure}[b]{0.09\textwidth}
    \includegraphics[clip, trim=2.35cm 1.75cm 4.5cm 0cm,width=\textwidth]{img/convergence/12419.pdf}
    \caption{12419}
    \label{fig:convergence_12419}
\end{subfigure}
%
\begin{subfigure}[b]{0.09\textwidth}
    \includegraphics[clip, trim=2.35cm 1.75cm 4.5cm 0cm,width=\textwidth]{img/convergence/12420.pdf}
    \caption{12420}
    \label{fig:convergence_12420}
\end{subfigure}
%
\begin{subfigure}[b]{0.09\textwidth}
    \includegraphics[clip, trim=2.35cm 1.75cm 4.5cm 0cm,width=\textwidth]{img/convergence/12424.pdf}
    \caption{12424}
    \label{fig:convergence_12424}
\end{subfigure}
%
\begin{subfigure}[b]{0.09\textwidth}
    \includegraphics[clip, trim=2.35cm 1.75cm 4.5cm 0cm,width=\textwidth]{img/convergence/12429.pdf}
    \caption{12429}
    \label{fig:convergence_12429}
\end{subfigure}
%
\begin{subfigure}[b]{0.09\textwidth}
    \includegraphics[clip, trim=2.35cm 1.75cm 4.5cm 0cm,width=\textwidth]{img/convergence/12433.pdf}
    \caption{12433}
    \label{fig:convergence_12433}
\end{subfigure}
%
\begin{subfigure}[b]{0.09\textwidth}
    \includegraphics[clip, trim=2.35cm 1.75cm 4.5cm 0cm,width=\textwidth]{img/convergence/12438.pdf}
    \caption{12438}
    \label{fig:convergence_12438}
\end{subfigure}
%
\begin{subfigure}[b]{0.09\textwidth}
    \includegraphics[clip, trim=2.35cm 1.75cm 4.5cm 0cm,width=\textwidth]{img/convergence/12443.pdf}
    \caption{12443}
    \label{fig:convergence_12443}
\end{subfigure}
%
\begin{subfigure}[b]{0.09\textwidth}
    \includegraphics[clip, trim=2.35cm 1.75cm 4.5cm 0cm,width=\textwidth]{img/convergence/12449.pdf}
    \caption{12449}
    \label{fig:convergence_12449}
\end{subfigure}
%
\begin{subfigure}[b]{0.09\textwidth}
    \includegraphics[clip, trim=2.35cm 1.75cm 4.5cm 0cm,width=\textwidth]{img/convergence/12453.pdf}
    \caption{12453}
    \label{fig:convergence_12453}
\end{subfigure}
%
\begin{subfigure}[b]{0.09\textwidth}
    \includegraphics[clip, trim=2.35cm 1.75cm 4.5cm 0cm,width=\textwidth]{img/convergence/12459.pdf}
    \caption{12459}
    \label{fig:convergence_12459}
\end{subfigure}
%
\begin{subfigure}[b]{0.09\textwidth}
    \includegraphics[clip, trim=2.35cm 1.75cm 4.5cm 0cm,width=\textwidth]{img/convergence/12464.pdf}
    \caption{12464}
    \label{fig:convergence_12464}
\end{subfigure}
%
\begin{subfigure}[b]{0.09\textwidth}
    \includegraphics[clip, trim=2.35cm 1.75cm 4.5cm 0cm,width=\textwidth]{img/convergence/12471.pdf}
    \caption{12471}
    \label{fig:convergence_12471}
\end{subfigure}
%
\begin{subfigure}[b]{0.09\textwidth}
    \includegraphics[clip, trim=2.35cm 1.75cm 4.5cm 0cm,width=\textwidth]{img/convergence/12474.pdf}
    \caption{12474}
    \label{fig:convergence_12474}
\end{subfigure}
%
\begin{subfigure}[b]{0.09\textwidth}
    \includegraphics[clip, trim=2.35cm 1.75cm 4.5cm 0cm,width=\textwidth]{img/convergence/12500.pdf}
    \caption{12500}
    \label{fig:convergence_12500}
\end{subfigure}
%
\begin{subfigure}[b]{0.09\textwidth}
    \includegraphics[clip, trim=2.35cm 1.75cm 4.5cm 0cm,width=\textwidth]{img/convergence/12584.pdf}
    \caption{12584}
    \label{fig:convergence_12584}
\end{subfigure}
%
\begin{subfigure}[b]{0.09\textwidth}
    \includegraphics[clip, trim=2.35cm 1.75cm 4.5cm 0cm,width=\textwidth]{img/convergence/12585.pdf}
    \caption{12585}
    \label{fig:convergence_12585}
\end{subfigure}
%
\begin{subfigure}[b]{0.09\textwidth}
    \includegraphics[clip, trim=2.35cm 1.75cm 4.5cm 0cm,width=\textwidth]{img/convergence/12590.pdf}
    \caption{12590}
    \label{fig:convergence_12590}
\end{subfigure}
%
\begin{subfigure}[b]{0.09\textwidth}
    \includegraphics[clip, trim=2.35cm 1.75cm 4.5cm 0cm,width=\textwidth]{img/convergence/12600.pdf}
    \caption{12600}
    \label{fig:convergence_12600}
\end{subfigure}
%
\begin{subfigure}[b]{0.09\textwidth}
    \includegraphics[clip, trim=2.35cm 1.75cm 4.5cm 0cm,width=\textwidth]{img/convergence/12602.pdf}
    \caption{12602}
    \label{fig:convergence_12602}
\end{subfigure}
%
\begin{subfigure}[b]{0.09\textwidth}
    \includegraphics[clip, trim=2.35cm 1.75cm 4.5cm 0cm,width=\textwidth]{img/convergence/12608.pdf}
    \caption{12608}
    \label{fig:convergence_12608}
\end{subfigure}
%
\begin{subfigure}[b]{0.09\textwidth}
    \includegraphics[clip, trim=2.35cm 1.75cm 4.5cm 0cm,width=\textwidth]{img/convergence/12609.pdf}
    \caption{12609}
    \label{fig:convergence_12609}
\end{subfigure}
%
\begin{subfigure}[b]{0.09\textwidth}
    \includegraphics[clip, trim=2.35cm 1.75cm 4.5cm 0cm,width=\textwidth]{img/convergence/12631.pdf}
    \caption{12631}
    \label{fig:convergence_12631}
\end{subfigure}
%
\begin{subfigure}[b]{0.09\textwidth}
    \includegraphics[clip, trim=2.35cm 1.75cm 4.5cm 0cm,width=\textwidth]{img/convergence/12700.pdf}
    \caption{12700}
    \label{fig:convergence_12700}
\end{subfigure}
%
\begin{subfigure}[b]{0.09\textwidth}
    \includegraphics[clip, trim=2.35cm 1.75cm 4.5cm 0cm,width=\textwidth]{img/convergence/12800.pdf}
    \caption{12800}
    \label{fig:convergence_12800}
\end{subfigure}
%
\begin{subfigure}[b]{0.09\textwidth}
    \includegraphics[clip, trim=2.35cm 1.75cm 4.5cm 0cm,width=\textwidth]{img/convergence/12900.pdf}
    \caption{12900}
    \label{fig:convergence_12900}
\end{subfigure}
%
\begin{subfigure}[b]{0.09\textwidth}
    \includegraphics[clip, trim=2.35cm 1.75cm 4.5cm 0cm,width=\textwidth]{img/convergence/13000.pdf}
    \caption{13000}
    \label{fig:convergence_13000}
\end{subfigure}
%
\begin{subfigure}[b]{0.09\textwidth}
    \includegraphics[clip, trim=2.35cm 1.75cm 4.5cm 0cm,width=\textwidth]{img/convergence/13100.pdf}
    \caption{13100}
    \label{fig:convergence_13100}
\end{subfigure}
%
\begin{subfigure}[b]{0.09\textwidth}
    \includegraphics[clip, trim=2.35cm 1.75cm 4.5cm 0cm,width=\textwidth]{img/convergence/13200.pdf}
    \caption{13200}
    \label{fig:convergence_13200}
\end{subfigure}
%
\begin{subfigure}[b]{0.09\textwidth}
    \includegraphics[clip, trim=2.35cm 1.75cm 4.5cm 0cm,width=\textwidth]{img/convergence/13300.pdf}
    \caption{13300}
    \label{fig:convergence_13300}
\end{subfigure}
%
\begin{subfigure}[b]{0.09\textwidth}
    \includegraphics[clip, trim=2.35cm 1.75cm 4.5cm 0cm,width=\textwidth]{img/convergence/13400.pdf}
    \caption{13400}
    \label{fig:convergence_13400}
\end{subfigure}
%
\begin{subfigure}[b]{0.09\textwidth}
    \includegraphics[clip, trim=2.35cm 1.75cm 4.5cm 0cm,width=\textwidth]{img/convergence/13500.pdf}
    \caption{13500}
    \label{fig:convergence_13500}
\end{subfigure}
%
\begin{subfigure}[b]{0.09\textwidth}
    \includegraphics[clip, trim=2.35cm 1.75cm 4.5cm 0cm,width=\textwidth]{img/convergence/13600.pdf}
    \caption{13600}
    \label{fig:convergence_13600}
\end{subfigure}
%
\begin{subfigure}[b]{0.09\textwidth}
    \includegraphics[clip, trim=2.35cm 1.75cm 4.5cm 0cm,width=\textwidth]{img/convergence/13602.pdf}
    \caption{13602}
    \label{fig:convergence_13602}
\end{subfigure}
%
\begin{subfigure}[b]{0.09\textwidth}
    \includegraphics[clip, trim=2.35cm 1.75cm 4.5cm 0cm,width=\textwidth]{img/convergence/13700.pdf}
    \caption{13700}
    \label{fig:convergence_13700}
\end{subfigure}
%
\begin{subfigure}[b]{0.09\textwidth}
    \includegraphics[clip, trim=2.35cm 1.75cm 4.5cm 0cm,width=\textwidth]{img/convergence/13800.pdf}
    \caption{13800}
    \label{fig:convergence_13800}
\end{subfigure}
%
\begin{subfigure}[b]{0.09\textwidth}
    \includegraphics[clip, trim=2.35cm 1.75cm 4.5cm 0cm,width=\textwidth]{img/convergence/13900.pdf}
    \caption{13900}
    \label{fig:convergence_13900}
\end{subfigure}
%
\begin{subfigure}[b]{0.09\textwidth}
    \includegraphics[clip, trim=2.35cm 1.75cm 4.5cm 0cm,width=\textwidth]{img/convergence/13909.pdf}
    \caption{13909}
    \label{fig:convergence_13909}
\end{subfigure}
%
\begin{subfigure}[b]{0.09\textwidth}
    \includegraphics[clip, trim=2.35cm 1.75cm 4.5cm 0cm,width=\textwidth]{img/convergence/13921.pdf}
    \caption{13921}
    \label{fig:convergence_13921}
\end{subfigure}
%
\begin{subfigure}[b]{0.09\textwidth}
    \includegraphics[clip, trim=2.35cm 1.75cm 4.5cm 0cm,width=\textwidth]{img/convergence/14000.pdf}
    \caption{14000}
    \label{fig:convergence_14000}
\end{subfigure}
%
\begin{subfigure}[b]{0.09\textwidth}
    \includegraphics[clip, trim=2.35cm 1.75cm 4.5cm 0cm,width=\textwidth]{img/convergence/14021.pdf}
    \caption{14021}
    \label{fig:convergence_14021}
\end{subfigure}
%
\begin{subfigure}[b]{0.09\textwidth}
    \includegraphics[clip, trim=2.35cm 1.75cm 4.5cm 0cm,width=\textwidth]{img/convergence/14022.pdf}
    \caption{14022}
    \label{fig:convergence_14022}
\end{subfigure}
%
\begin{subfigure}[b]{0.09\textwidth}
    \includegraphics[clip, trim=2.35cm 1.75cm 4.5cm 0cm,width=\textwidth]{img/convergence/14023.pdf}
    \caption{14023}
    \label{fig:convergence_14023}
\end{subfigure}
%
\begin{subfigure}[b]{0.09\textwidth}
    \includegraphics[clip, trim=2.35cm 1.75cm 4.5cm 0cm,width=\textwidth]{img/convergence/14024.pdf}
    \caption{14024}
    \label{fig:convergence_14024}
\end{subfigure}
%
\begin{subfigure}[b]{0.09\textwidth}
    \includegraphics[clip, trim=2.35cm 1.75cm 4.5cm 0cm,width=\textwidth]{img/convergence/14025.pdf}
    \caption{14025}
    \label{fig:convergence_14025}
\end{subfigure}
%
\begin{subfigure}[b]{0.09\textwidth}
    \includegraphics[clip, trim=2.35cm 1.75cm 4.5cm 0cm,width=\textwidth]{img/convergence/14028.pdf}
    \caption{14028}
    \label{fig:convergence_14028}
\end{subfigure}
%
\begin{subfigure}[b]{0.09\textwidth}
    \includegraphics[clip, trim=2.35cm 1.75cm 4.5cm 0cm,width=\textwidth]{img/convergence/14029.pdf}
    \caption{14029}
    \label{fig:convergence_14029}
\end{subfigure}
%
\begin{subfigure}[b]{0.09\textwidth}
    \includegraphics[clip, trim=2.35cm 1.75cm 4.5cm 0cm,width=\textwidth]{img/convergence/14030.pdf}
    \caption{14030}
    \label{fig:convergence_14030}
\end{subfigure}
%
\begin{subfigure}[b]{0.09\textwidth}
    \includegraphics[clip, trim=2.35cm 1.75cm 4.5cm 0cm,width=\textwidth]{img/convergence/14034.pdf}
    \caption{14034}
    \label{fig:convergence_14034}
\end{subfigure}
%
\begin{subfigure}[b]{0.09\textwidth}
    \includegraphics[clip, trim=2.35cm 1.75cm 4.5cm 0cm,width=\textwidth]{img/convergence/14035.pdf}
    \caption{14035}
    \label{fig:convergence_14035}
\end{subfigure}
%
\begin{subfigure}[b]{0.09\textwidth}
    \includegraphics[clip, trim=2.35cm 1.75cm 4.5cm 0cm,width=\textwidth]{img/convergence/14036.pdf}
    \caption{14036}
    \label{fig:convergence_14036}
\end{subfigure}
%
\begin{subfigure}[b]{0.09\textwidth}
    \includegraphics[clip, trim=2.35cm 1.75cm 4.5cm 0cm,width=\textwidth]{img/convergence/14037.pdf}
    \caption{14037}
    \label{fig:convergence_14037}
\end{subfigure}
%
\begin{subfigure}[b]{0.09\textwidth}
    \includegraphics[clip, trim=2.35cm 1.75cm 4.5cm 0cm,width=\textwidth]{img/convergence/14038.pdf}
    \caption{14038}
    \label{fig:convergence_14038}
\end{subfigure}
%
\begin{subfigure}[b]{0.09\textwidth}
    \includegraphics[clip, trim=2.35cm 1.75cm 4.5cm 0cm,width=\textwidth]{img/convergence/14039.pdf}
    \caption{14039}
    \label{fig:convergence_14039}
\end{subfigure}
%
\begin{subfigure}[b]{0.09\textwidth}
    \includegraphics[clip, trim=2.35cm 1.75cm 4.5cm 0cm,width=\textwidth]{img/convergence/14040.pdf}
    \caption{14040}
    \label{fig:convergence_14040}
\end{subfigure}
%
\begin{subfigure}[b]{0.09\textwidth}
    \includegraphics[clip, trim=2.35cm 1.75cm 4.5cm 0cm,width=\textwidth]{img/convergence/14041.pdf}
    \caption{14041}
    \label{fig:convergence_14041}
\end{subfigure}
%
\begin{subfigure}[b]{0.09\textwidth}
    \includegraphics[clip, trim=2.35cm 1.75cm 4.5cm 0cm,width=\textwidth]{img/convergence/14042.pdf}
    \caption{14042}
    \label{fig:convergence_14042}
\end{subfigure}
%
\begin{subfigure}[b]{0.09\textwidth}
    \includegraphics[clip, trim=2.35cm 1.75cm 4.5cm 0cm,width=\textwidth]{img/convergence/14043.pdf}
    \caption{14043}
    \label{fig:convergence_14043}
\end{subfigure}
%
\begin{subfigure}[b]{0.09\textwidth}
    \includegraphics[clip, trim=2.35cm 1.75cm 4.5cm 0cm,width=\textwidth]{img/convergence/14044.pdf}
    \caption{14044}
    \label{fig:convergence_14044}
\end{subfigure}
%
\begin{subfigure}[b]{0.09\textwidth}
    \includegraphics[clip, trim=2.35cm 1.75cm 4.5cm 0cm,width=\textwidth]{img/convergence/14045.pdf}
    \caption{14045}
    \label{fig:convergence_14045}
\end{subfigure}
%
\begin{subfigure}[b]{0.09\textwidth}
    \includegraphics[clip, trim=2.35cm 1.75cm 4.5cm 0cm,width=\textwidth]{img/convergence/14046.pdf}
    \caption{14046}
    \label{fig:convergence_14046}
\end{subfigure}
%
\begin{subfigure}[b]{0.09\textwidth}
    \includegraphics[clip, trim=2.35cm 1.75cm 4.5cm 0cm,width=\textwidth]{img/convergence/14048.pdf}
    \caption{14048}
    \label{fig:convergence_14048}
\end{subfigure}
%
\begin{subfigure}[b]{0.09\textwidth}
    \includegraphics[clip, trim=2.35cm 1.75cm 4.5cm 0cm,width=\textwidth]{img/convergence/14050.pdf}
    \caption{14050}
    \label{fig:convergence_14050}
\end{subfigure}
%
\begin{subfigure}[b]{0.09\textwidth}
    \includegraphics[clip, trim=2.35cm 1.75cm 4.5cm 0cm,width=\textwidth]{img/convergence/14055.pdf}
    \caption{14055}
    \label{fig:convergence_14055}
\end{subfigure}
%
\begin{subfigure}[b]{0.09\textwidth}
    \includegraphics[clip, trim=2.35cm 1.75cm 4.5cm 0cm,width=\textwidth]{img/convergence/14056.pdf}
    \caption{14056}
    \label{fig:convergence_14056}
\end{subfigure}
%
\begin{subfigure}[b]{0.09\textwidth}
    \includegraphics[clip, trim=2.35cm 1.75cm 4.5cm 0cm,width=\textwidth]{img/convergence/14059.pdf}
    \caption{14059}
    \label{fig:convergence_14059}
\end{subfigure}
%
\begin{subfigure}[b]{0.09\textwidth}
    \includegraphics[clip, trim=2.35cm 1.75cm 4.5cm 0cm,width=\textwidth]{img/convergence/14067.pdf}
    \caption{14067}
    \label{fig:convergence_14067}
\end{subfigure}
%
\begin{subfigure}[b]{0.09\textwidth}
    \includegraphics[clip, trim=2.35cm 1.75cm 4.5cm 0cm,width=\textwidth]{img/convergence/14068.pdf}
    \caption{14068}
    \label{fig:convergence_14068}
\end{subfigure}
%
\begin{subfigure}[b]{0.09\textwidth}
    \includegraphics[clip, trim=2.35cm 1.75cm 4.5cm 0cm,width=\textwidth]{img/convergence/14070.pdf}
    \caption{14070}
    \label{fig:convergence_14070}
\end{subfigure}
%
\begin{subfigure}[b]{0.09\textwidth}
    \includegraphics[clip, trim=2.35cm 1.75cm 4.5cm 0cm,width=\textwidth]{img/convergence/14071.pdf}
    \caption{14071}
    \label{fig:convergence_14071}
\end{subfigure}
%
\begin{subfigure}[b]{0.09\textwidth}
    \includegraphics[clip, trim=2.35cm 1.75cm 4.5cm 0cm,width=\textwidth]{img/convergence/14072.pdf}
    \caption{14072}
    \label{fig:convergence_14072}
\end{subfigure}
%
\begin{subfigure}[b]{0.09\textwidth}
    \includegraphics[clip, trim=2.35cm 1.75cm 4.5cm 0cm,width=\textwidth]{img/convergence/14073.pdf}
    \caption{14073}
    \label{fig:convergence_14073}
\end{subfigure}
%
\begin{subfigure}[b]{0.09\textwidth}
    \includegraphics[clip, trim=2.35cm 1.75cm 4.5cm 0cm,width=\textwidth]{img/convergence/14074.pdf}
    \caption{14074}
    \label{fig:convergence_14074}
\end{subfigure}
%
\begin{subfigure}[b]{0.09\textwidth}
    \includegraphics[clip, trim=2.35cm 1.75cm 4.5cm 0cm,width=\textwidth]{img/convergence/14075.pdf}
    \caption{14075}
    \label{fig:convergence_14075}
\end{subfigure}
%
\begin{subfigure}[b]{0.09\textwidth}
    \includegraphics[clip, trim=2.35cm 1.75cm 4.5cm 0cm,width=\textwidth]{img/convergence/14085.pdf}
    \caption{14085}
    \label{fig:convergence_14085}
\end{subfigure}
%
\begin{subfigure}[b]{0.09\textwidth}
    \includegraphics[clip, trim=2.35cm 1.75cm 4.5cm 0cm,width=\textwidth]{img/convergence/14100.pdf}
    \caption{14100}
    \label{fig:convergence_14100}
\end{subfigure}
%
\begin{subfigure}[b]{0.09\textwidth}
    \includegraphics[clip, trim=2.35cm 1.75cm 4.5cm 0cm,width=\textwidth]{img/convergence/14131.pdf}
    \caption{14131}
    \label{fig:convergence_14131}
\end{subfigure}
%
\begin{subfigure}[b]{0.09\textwidth}
    \includegraphics[clip, trim=2.35cm 1.75cm 4.5cm 0cm,width=\textwidth]{img/convergence/14168.pdf}
    \caption{14168}
    \label{fig:convergence_14168}
\end{subfigure}
%
\begin{subfigure}[b]{0.09\textwidth}
    \includegraphics[clip, trim=2.35cm 1.75cm 4.5cm 0cm,width=\textwidth]{img/convergence/14173.pdf}
    \caption{14173}
    \label{fig:convergence_14173}
\end{subfigure}
%
\begin{subfigure}[b]{0.09\textwidth}
    \includegraphics[clip, trim=2.35cm 1.75cm 4.5cm 0cm,width=\textwidth]{img/convergence/14178.pdf}
    \caption{14178}
    \label{fig:convergence_14178}
\end{subfigure}
%
\begin{subfigure}[b]{0.09\textwidth}
    \includegraphics[clip, trim=2.35cm 1.75cm 4.5cm 0cm,width=\textwidth]{img/convergence/14183.pdf}
    \caption{14183}
    \label{fig:convergence_14183}
\end{subfigure}
%
\begin{subfigure}[b]{0.09\textwidth}
    \includegraphics[clip, trim=2.35cm 1.75cm 4.5cm 0cm,width=\textwidth]{img/convergence/14189.pdf}
    \caption{14189}
    \label{fig:convergence_14189}
\end{subfigure}
%
\begin{subfigure}[b]{0.09\textwidth}
    \includegraphics[clip, trim=2.35cm 1.75cm 4.5cm 0cm,width=\textwidth]{img/convergence/14194.pdf}
    \caption{14194}
    \label{fig:convergence_14194}
\end{subfigure}
%
\begin{subfigure}[b]{0.09\textwidth}
    \includegraphics[clip, trim=2.35cm 1.75cm 4.5cm 0cm,width=\textwidth]{img/convergence/14198.pdf}
    \caption{14198}
    \label{fig:convergence_14198}
\end{subfigure}
%
\begin{subfigure}[b]{0.09\textwidth}
    \includegraphics[clip, trim=2.35cm 1.75cm 4.5cm 0cm,width=\textwidth]{img/convergence/14199.pdf}
    \caption{14199}
    \label{fig:convergence_14199}
\end{subfigure}
%
\begin{subfigure}[b]{0.09\textwidth}
    \includegraphics[clip, trim=2.35cm 1.75cm 4.5cm 0cm,width=\textwidth]{img/convergence/14200.pdf}
    \caption{14200}
    \label{fig:convergence_14200}
\end{subfigure}
%
\begin{subfigure}[b]{0.09\textwidth}
    \includegraphics[clip, trim=2.35cm 1.75cm 4.5cm 0cm,width=\textwidth]{img/convergence/14209.pdf}
    \caption{14209}
    \label{fig:convergence_14209}
\end{subfigure}
%
\begin{subfigure}[b]{0.09\textwidth}
    \includegraphics[clip, trim=2.35cm 1.75cm 4.5cm 0cm,width=\textwidth]{img/convergence/14210.pdf}
    \caption{14210}
    \label{fig:convergence_14210}
\end{subfigure}
%
\begin{subfigure}[b]{0.09\textwidth}
    \includegraphics[clip, trim=2.35cm 1.75cm 4.5cm 0cm,width=\textwidth]{img/convergence/14216.pdf}
    \caption{14216}
    \label{fig:convergence_14216}
\end{subfigure}
%
\begin{subfigure}[b]{0.09\textwidth}
    \includegraphics[clip, trim=2.35cm 1.75cm 4.5cm 0cm,width=\textwidth]{img/convergence/14218.pdf}
    \caption{14218}
    \label{fig:convergence_14218}
\end{subfigure}
%
\begin{subfigure}[b]{0.09\textwidth}
    \includegraphics[clip, trim=2.35cm 1.75cm 4.5cm 0cm,width=\textwidth]{img/convergence/14219.pdf}
    \caption{14219}
    \label{fig:convergence_14219}
\end{subfigure}
%
\begin{subfigure}[b]{0.09\textwidth}
    \includegraphics[clip, trim=2.35cm 1.75cm 4.5cm 0cm,width=\textwidth]{img/convergence/14220.pdf}
    \caption{14220}
    \label{fig:convergence_14220}
\end{subfigure}
%
\begin{subfigure}[b]{0.09\textwidth}
    \includegraphics[clip, trim=2.35cm 1.75cm 4.5cm 0cm,width=\textwidth]{img/convergence/14221.pdf}
    \caption{14221}
    \label{fig:convergence_14221}
\end{subfigure}
%
\begin{subfigure}[b]{0.09\textwidth}
    \includegraphics[clip, trim=2.35cm 1.75cm 4.5cm 0cm,width=\textwidth]{img/convergence/14223.pdf}
    \caption{14223}
    \label{fig:convergence_14223}
\end{subfigure}
%
\begin{subfigure}[b]{0.09\textwidth}
    \includegraphics[clip, trim=2.35cm 1.75cm 4.5cm 0cm,width=\textwidth]{img/convergence/14233.pdf}
    \caption{14233}
    \label{fig:convergence_14233}
\end{subfigure}
%
\begin{subfigure}[b]{0.09\textwidth}
    \includegraphics[clip, trim=2.35cm 1.75cm 4.5cm 0cm,width=\textwidth]{img/convergence/14236.pdf}
    \caption{14236}
    \label{fig:convergence_14236}
\end{subfigure}
%
\begin{subfigure}[b]{0.09\textwidth}
    \includegraphics[clip, trim=2.35cm 1.75cm 4.5cm 0cm,width=\textwidth]{img/convergence/14245.pdf}
    \caption{14245}
    \label{fig:convergence_14245}
\end{subfigure}
%
\begin{subfigure}[b]{0.09\textwidth}
    \includegraphics[clip, trim=2.35cm 1.75cm 4.5cm 0cm,width=\textwidth]{img/convergence/14248.pdf}
    \caption{14248}
    \label{fig:convergence_14248}
\end{subfigure}
%
\begin{subfigure}[b]{0.09\textwidth}
    \includegraphics[clip, trim=2.35cm 1.75cm 4.5cm 0cm,width=\textwidth]{img/convergence/14249.pdf}
    \caption{14249}
    \label{fig:convergence_14249}
\end{subfigure}
%
\begin{subfigure}[b]{0.09\textwidth}
    \includegraphics[clip, trim=2.35cm 1.75cm 4.5cm 0cm,width=\textwidth]{img/convergence/14250.pdf}
    \caption{14250}
    \label{fig:convergence_14250}
\end{subfigure}
%
\begin{subfigure}[b]{0.09\textwidth}
    \includegraphics[clip, trim=2.35cm 1.75cm 4.5cm 0cm,width=\textwidth]{img/convergence/14252.pdf}
    \caption{14252}
    \label{fig:convergence_14252}
\end{subfigure}
%
\begin{subfigure}[b]{0.09\textwidth}
    \includegraphics[clip, trim=2.35cm 1.75cm 4.5cm 0cm,width=\textwidth]{img/convergence/14253.pdf}
    \caption{14253}
    \label{fig:convergence_14253}
\end{subfigure}
%
\begin{subfigure}[b]{0.09\textwidth}
    \includegraphics[clip, trim=2.35cm 1.75cm 4.5cm 0cm,width=\textwidth]{img/convergence/14256.pdf}
    \caption{14256}
    \label{fig:convergence_14256}
\end{subfigure}
%
\begin{subfigure}[b]{0.09\textwidth}
    \includegraphics[clip, trim=2.35cm 1.75cm 4.5cm 0cm,width=\textwidth]{img/convergence/14263.pdf}
    \caption{14263}
    \label{fig:convergence_14263}
\end{subfigure}
%
\begin{subfigure}[b]{0.09\textwidth}
    \includegraphics[clip, trim=2.35cm 1.75cm 4.5cm 0cm,width=\textwidth]{img/convergence/14291.pdf}
    \caption{14291}
    \label{fig:convergence_14291}
\end{subfigure}
%
\begin{subfigure}[b]{0.09\textwidth}
    \includegraphics[clip, trim=2.35cm 1.75cm 4.5cm 0cm,width=\textwidth]{img/convergence/14292.pdf}
    \caption{14292}
    \label{fig:convergence_14292}
\end{subfigure}
%
\begin{subfigure}[b]{0.09\textwidth}
    \includegraphics[clip, trim=2.35cm 1.75cm 4.5cm 0cm,width=\textwidth]{img/convergence/14293.pdf}
    \caption{14293}
    \label{fig:convergence_14293}
\end{subfigure}
%
\begin{subfigure}[b]{0.09\textwidth}
    \includegraphics[clip, trim=2.35cm 1.75cm 4.5cm 0cm,width=\textwidth]{img/convergence/14294.pdf}
    \caption{14294}
    \label{fig:convergence_14294}
\end{subfigure}
%
\begin{subfigure}[b]{0.09\textwidth}
    \includegraphics[clip, trim=2.35cm 1.75cm 4.5cm 0cm,width=\textwidth]{img/convergence/14298.pdf}
    \caption{14298}
    \label{fig:convergence_14298}
\end{subfigure}
%
\begin{subfigure}[b]{0.09\textwidth}
    \includegraphics[clip, trim=2.35cm 1.75cm 4.5cm 0cm,width=\textwidth]{img/convergence/14300.pdf}
    \caption{14300}
    \label{fig:convergence_14300}
\end{subfigure}
%
\begin{subfigure}[b]{0.09\textwidth}
    \includegraphics[clip, trim=2.35cm 1.75cm 4.5cm 0cm,width=\textwidth]{img/convergence/14302.pdf}
    \caption{14302}
    \label{fig:convergence_14302}
\end{subfigure}
%
\begin{subfigure}[b]{0.09\textwidth}
    \includegraphics[clip, trim=2.35cm 1.75cm 4.5cm 0cm,width=\textwidth]{img/convergence/14304.pdf}
    \caption{14304}
    \label{fig:convergence_14304}
\end{subfigure}
%
\begin{subfigure}[b]{0.09\textwidth}
    \includegraphics[clip, trim=2.35cm 1.75cm 4.5cm 0cm,width=\textwidth]{img/convergence/14314.pdf}
    \caption{14314}
    \label{fig:convergence_14314}
\end{subfigure}
%
\begin{subfigure}[b]{0.09\textwidth}
    \includegraphics[clip, trim=2.35cm 1.75cm 4.5cm 0cm,width=\textwidth]{img/convergence/14319.pdf}
    \caption{14319}
    \label{fig:convergence_14319}
\end{subfigure}
%
\begin{subfigure}[b]{0.09\textwidth}
    \includegraphics[clip, trim=2.35cm 1.75cm 4.5cm 0cm,width=\textwidth]{img/convergence/14328.pdf}
    \caption{14328}
    \label{fig:convergence_14328}
\end{subfigure}
%
\begin{subfigure}[b]{0.09\textwidth}
    \includegraphics[clip, trim=2.35cm 1.75cm 4.5cm 0cm,width=\textwidth]{img/convergence/14340.pdf}
    \caption{14340}
    \label{fig:convergence_14340}
\end{subfigure}
%
\begin{subfigure}[b]{0.09\textwidth}
    \includegraphics[clip, trim=2.35cm 1.75cm 4.5cm 0cm,width=\textwidth]{img/convergence/14381.pdf}
    \caption{14381}
    \label{fig:convergence_14381}
\end{subfigure}
%
\begin{subfigure}[b]{0.09\textwidth}
    \includegraphics[clip, trim=2.35cm 1.75cm 4.5cm 0cm,width=\textwidth]{img/convergence/14382.pdf}
    \caption{14382}
    \label{fig:convergence_14382}
\end{subfigure}
%
\begin{subfigure}[b]{0.09\textwidth}
    \includegraphics[clip, trim=2.35cm 1.75cm 4.5cm 0cm,width=\textwidth]{img/convergence/14393.pdf}
    \caption{14393}
    \label{fig:convergence_14393}
\end{subfigure}
%
\begin{subfigure}[b]{0.09\textwidth}
    \includegraphics[clip, trim=2.35cm 1.75cm 4.5cm 0cm,width=\textwidth]{img/convergence/14398.pdf}
    \caption{14398}
    \label{fig:convergence_14398}
\end{subfigure}
%
\begin{subfigure}[b]{0.09\textwidth}
    \includegraphics[clip, trim=2.35cm 1.75cm 4.5cm 0cm,width=\textwidth]{img/convergence/14400.pdf}
    \caption{14400}
    \label{fig:convergence_14400}
\end{subfigure}
%
\begin{subfigure}[b]{0.09\textwidth}
    \includegraphics[clip, trim=2.35cm 1.75cm 4.5cm 0cm,width=\textwidth]{img/convergence/14415.pdf}
    \caption{14415}
    \label{fig:convergence_14415}
\end{subfigure}
%
\begin{subfigure}[b]{0.09\textwidth}
    \includegraphics[clip, trim=2.35cm 1.75cm 4.5cm 0cm,width=\textwidth]{img/convergence/14427.pdf}
    \caption{14427}
    \label{fig:convergence_14427}
\end{subfigure}
%
\begin{subfigure}[b]{0.09\textwidth}
    \includegraphics[clip, trim=2.35cm 1.75cm 4.5cm 0cm,width=\textwidth]{img/convergence/14433.pdf}
    \caption{14433}
    \label{fig:convergence_14433}
\end{subfigure}
%
\begin{subfigure}[b]{0.09\textwidth}
    \includegraphics[clip, trim=2.35cm 1.75cm 4.5cm 0cm,width=\textwidth]{img/convergence/14438.pdf}
    \caption{14438}
    \label{fig:convergence_14438}
\end{subfigure}
%
\begin{subfigure}[b]{0.09\textwidth}
    \includegraphics[clip, trim=2.35cm 1.75cm 4.5cm 0cm,width=\textwidth]{img/convergence/14440.pdf}
    \caption{14440}
    \label{fig:convergence_14440}
\end{subfigure}
%
\begin{subfigure}[b]{0.09\textwidth}
    \includegraphics[clip, trim=2.35cm 1.75cm 4.5cm 0cm,width=\textwidth]{img/convergence/14455.pdf}
    \caption{14455}
    \label{fig:convergence_14455}
\end{subfigure}
%
\begin{subfigure}[b]{0.09\textwidth}
    \includegraphics[clip, trim=2.35cm 1.75cm 4.5cm 0cm,width=\textwidth]{img/convergence/14457.pdf}
    \caption{14457}
    \label{fig:convergence_14457}
\end{subfigure}
%
\begin{subfigure}[b]{0.09\textwidth}
    \includegraphics[clip, trim=2.35cm 1.75cm 4.5cm 0cm,width=\textwidth]{img/convergence/14461.pdf}
    \caption{14461}
    \label{fig:convergence_14461}
\end{subfigure}
%
\begin{subfigure}[b]{0.09\textwidth}
    \includegraphics[clip, trim=2.35cm 1.75cm 4.5cm 0cm,width=\textwidth]{img/convergence/14462.pdf}
    \caption{14462}
    \label{fig:convergence_14462}
\end{subfigure}
%
\begin{subfigure}[b]{0.09\textwidth}
    \includegraphics[clip, trim=2.35cm 1.75cm 4.5cm 0cm,width=\textwidth]{img/convergence/14463.pdf}
    \caption{14463}
    \label{fig:convergence_14463}
\end{subfigure}
%
\begin{subfigure}[b]{0.09\textwidth}
    \includegraphics[clip, trim=2.35cm 1.75cm 4.5cm 0cm,width=\textwidth]{img/convergence/14464.pdf}
    \caption{14464}
    \label{fig:convergence_14464}
\end{subfigure}
%
\begin{subfigure}[b]{0.09\textwidth}
    \includegraphics[clip, trim=2.35cm 1.75cm 4.5cm 0cm,width=\textwidth]{img/convergence/14466.pdf}
    \caption{14466}
    \label{fig:convergence_14466}
\end{subfigure}
%
\begin{subfigure}[b]{0.09\textwidth}
    \includegraphics[clip, trim=2.35cm 1.75cm 4.5cm 0cm,width=\textwidth]{img/convergence/14471.pdf}
    \caption{14471}
    \label{fig:convergence_14471}
\end{subfigure}
%
\begin{subfigure}[b]{0.09\textwidth}
    \includegraphics[clip, trim=2.35cm 1.75cm 4.5cm 0cm,width=\textwidth]{img/convergence/14473.pdf}
    \caption{14473}
    \label{fig:convergence_14473}
\end{subfigure}
%
\begin{subfigure}[b]{0.09\textwidth}
    \includegraphics[clip, trim=2.35cm 1.75cm 4.5cm 0cm,width=\textwidth]{img/convergence/14487.pdf}
    \caption{14487}
    \label{fig:convergence_14487}
\end{subfigure}
%
\begin{subfigure}[b]{0.09\textwidth}
    \includegraphics[clip, trim=2.35cm 1.75cm 4.5cm 0cm,width=\textwidth]{img/convergence/14492.pdf}
    \caption{14492}
    \label{fig:convergence_14492}
\end{subfigure}
%
\begin{subfigure}[b]{0.09\textwidth}
    \includegraphics[clip, trim=2.35cm 1.75cm 4.5cm 0cm,width=\textwidth]{img/convergence/14499.pdf}
    \caption{14499}
    \label{fig:convergence_14499}
\end{subfigure}
%
\begin{subfigure}[b]{0.09\textwidth}
    \includegraphics[clip, trim=2.35cm 1.75cm 4.5cm 0cm,width=\textwidth]{img/convergence/14500.pdf}
    \caption{14500}
    \label{fig:convergence_14500}
\end{subfigure}
%
\begin{subfigure}[b]{0.09\textwidth}
    \includegraphics[clip, trim=2.35cm 1.75cm 4.5cm 0cm,width=\textwidth]{img/convergence/14600.pdf}
    \caption{14600}
    \label{fig:convergence_14600}
\end{subfigure}
%
\begin{subfigure}[b]{0.09\textwidth}
    \includegraphics[clip, trim=2.35cm 1.75cm 4.5cm 0cm,width=\textwidth]{img/convergence/14700.pdf}
    \caption{14700}
    \label{fig:convergence_14700}
\end{subfigure}
%
\begin{subfigure}[b]{0.09\textwidth}
    \includegraphics[clip, trim=2.35cm 1.75cm 4.5cm 0cm,width=\textwidth]{img/convergence/14800.pdf}
    \caption{14800}
    \label{fig:convergence_14800}
\end{subfigure}
%
\begin{subfigure}[b]{0.09\textwidth}
    \includegraphics[clip, trim=2.35cm 1.75cm 4.5cm 0cm,width=\textwidth]{img/convergence/14900.pdf}
    \caption{14900}
    \label{fig:convergence_14900}
\end{subfigure}
%
\begin{subfigure}[b]{0.09\textwidth}
    \includegraphics[clip, trim=2.35cm 1.75cm 4.5cm 0cm,width=\textwidth]{img/convergence/15000.pdf}
    \caption{15000}
    \label{fig:convergence_15000}
\end{subfigure}
%
\begin{subfigure}[b]{0.09\textwidth}
    \includegraphics[clip, trim=2.35cm 1.75cm 4.5cm 0cm,width=\textwidth]{img/convergence/15043.pdf}
    \caption{15043}
    \label{fig:convergence_15043}
\end{subfigure}
%
\begin{subfigure}[b]{0.09\textwidth}
    \includegraphics[clip, trim=2.35cm 1.75cm 4.5cm 0cm,width=\textwidth]{img/convergence/15078.pdf}
    \caption{15078}
    \label{fig:convergence_15078}
\end{subfigure}
%
\begin{subfigure}[b]{0.09\textwidth}
    \includegraphics[clip, trim=2.35cm 1.75cm 4.5cm 0cm,width=\textwidth]{img/convergence/15079.pdf}
    \caption{15079}
    \label{fig:convergence_15079}
\end{subfigure}
%
\begin{subfigure}[b]{0.09\textwidth}
    \includegraphics[clip, trim=2.35cm 1.75cm 4.5cm 0cm,width=\textwidth]{img/convergence/15080.pdf}
    \caption{15080}
    \label{fig:convergence_15080}
\end{subfigure}
%
\begin{subfigure}[b]{0.09\textwidth}
    \includegraphics[clip, trim=2.35cm 1.75cm 4.5cm 0cm,width=\textwidth]{img/convergence/15081.pdf}
    \caption{15081}
    \label{fig:convergence_15081}
\end{subfigure}
%
\begin{subfigure}[b]{0.09\textwidth}
    \includegraphics[clip, trim=2.35cm 1.75cm 4.5cm 0cm,width=\textwidth]{img/convergence/15085.pdf}
    \caption{15085}
    \label{fig:convergence_15085}
\end{subfigure}
%
\begin{subfigure}[b]{0.09\textwidth}
    \includegraphics[clip, trim=2.35cm 1.75cm 4.5cm 0cm,width=\textwidth]{img/convergence/15087.pdf}
    \caption{15087}
    \label{fig:convergence_15087}
\end{subfigure}
%
\begin{subfigure}[b]{0.09\textwidth}
    \includegraphics[clip, trim=2.35cm 1.75cm 4.5cm 0cm,width=\textwidth]{img/convergence/15100.pdf}
    \caption{15100}
    \label{fig:convergence_15100}
\end{subfigure}
%
\begin{subfigure}[b]{0.09\textwidth}
    \includegraphics[clip, trim=2.35cm 1.75cm 4.5cm 0cm,width=\textwidth]{img/convergence/15125.pdf}
    \caption{15125}
    \label{fig:convergence_15125}
\end{subfigure}
%
\begin{subfigure}[b]{0.09\textwidth}
    \includegraphics[clip, trim=2.35cm 1.75cm 4.5cm 0cm,width=\textwidth]{img/convergence/15200.pdf}
    \caption{15200}
    \label{fig:convergence_15200}
\end{subfigure}
%
\begin{subfigure}[b]{0.09\textwidth}
    \includegraphics[clip, trim=2.35cm 1.75cm 4.5cm 0cm,width=\textwidth]{img/convergence/15300.pdf}
    \caption{15300}
    \label{fig:convergence_15300}
\end{subfigure}
%
\begin{subfigure}[b]{0.09\textwidth}
    \includegraphics[clip, trim=2.35cm 1.75cm 4.5cm 0cm,width=\textwidth]{img/convergence/15400.pdf}
    \caption{15400}
    \label{fig:convergence_15400}
\end{subfigure}
%
\begin{subfigure}[b]{0.09\textwidth}
    \includegraphics[clip, trim=2.35cm 1.75cm 4.5cm 0cm,width=\textwidth]{img/convergence/15425.pdf}
    \caption{15425}
    \label{fig:convergence_15425}
\end{subfigure}
%
\begin{subfigure}[b]{0.09\textwidth}
    \includegraphics[clip, trim=2.35cm 1.75cm 4.5cm 0cm,width=\textwidth]{img/convergence/15427.pdf}
    \caption{15427}
    \label{fig:convergence_15427}
\end{subfigure}
%
\begin{subfigure}[b]{0.09\textwidth}
    \includegraphics[clip, trim=2.35cm 1.75cm 4.5cm 0cm,width=\textwidth]{img/convergence/15500.pdf}
    \caption{15500}
    \label{fig:convergence_15500}
\end{subfigure}
%
\begin{subfigure}[b]{0.09\textwidth}
    \includegraphics[clip, trim=2.35cm 1.75cm 4.5cm 0cm,width=\textwidth]{img/convergence/15513.pdf}
    \caption{15513}
    \label{fig:convergence_15513}
\end{subfigure}
%
\begin{subfigure}[b]{0.09\textwidth}
    \includegraphics[clip, trim=2.35cm 1.75cm 4.5cm 0cm,width=\textwidth]{img/convergence/15514.pdf}
    \caption{15514}
    \label{fig:convergence_15514}
\end{subfigure}
%
\begin{subfigure}[b]{0.09\textwidth}
    \includegraphics[clip, trim=2.35cm 1.75cm 4.5cm 0cm,width=\textwidth]{img/convergence/15527.pdf}
    \caption{15527}
    \label{fig:convergence_15527}
\end{subfigure}
%
\begin{subfigure}[b]{0.09\textwidth}
    \includegraphics[clip, trim=2.35cm 1.75cm 4.5cm 0cm,width=\textwidth]{img/convergence/15600.pdf}
    \caption{15600}
    \label{fig:convergence_15600}
\end{subfigure}
%
\begin{subfigure}[b]{0.09\textwidth}
    \includegraphics[clip, trim=2.35cm 1.75cm 4.5cm 0cm,width=\textwidth]{img/convergence/15602.pdf}
    \caption{15602}
    \label{fig:convergence_15602}
\end{subfigure}
%
\begin{subfigure}[b]{0.09\textwidth}
    \includegraphics[clip, trim=2.35cm 1.75cm 4.5cm 0cm,width=\textwidth]{img/convergence/15639.pdf}
    \caption{15639}
    \label{fig:convergence_15639}
\end{subfigure}
%
\begin{subfigure}[b]{0.09\textwidth}
    \includegraphics[clip, trim=2.35cm 1.75cm 4.5cm 0cm,width=\textwidth]{img/convergence/15700.pdf}
    \caption{15700}
    \label{fig:convergence_15700}
\end{subfigure}
%
\begin{subfigure}[b]{0.09\textwidth}
    \includegraphics[clip, trim=2.35cm 1.75cm 4.5cm 0cm,width=\textwidth]{img/convergence/15759.pdf}
    \caption{15759}
    \label{fig:convergence_15759}
\end{subfigure}
%
\begin{subfigure}[b]{0.09\textwidth}
    \includegraphics[clip, trim=2.35cm 1.75cm 4.5cm 0cm,width=\textwidth]{img/convergence/15781.pdf}
    \caption{15781}
    \label{fig:convergence_15781}
\end{subfigure}
%
\begin{subfigure}[b]{0.09\textwidth}
    \includegraphics[clip, trim=2.35cm 1.75cm 4.5cm 0cm,width=\textwidth]{img/convergence/15783.pdf}
    \caption{15783}
    \label{fig:convergence_15783}
\end{subfigure}
%
\begin{subfigure}[b]{0.09\textwidth}
    \includegraphics[clip, trim=2.35cm 1.75cm 4.5cm 0cm,width=\textwidth]{img/convergence/15786.pdf}
    \caption{15786}
    \label{fig:convergence_15786}
\end{subfigure}
%
\begin{subfigure}[b]{0.09\textwidth}
    \includegraphics[clip, trim=2.35cm 1.75cm 4.5cm 0cm,width=\textwidth]{img/convergence/15798.pdf}
    \caption{15798}
    \label{fig:convergence_15798}
\end{subfigure}
%
\begin{subfigure}[b]{0.09\textwidth}
    \includegraphics[clip, trim=2.35cm 1.75cm 4.5cm 0cm,width=\textwidth]{img/convergence/15800.pdf}
    \caption{15800}
    \label{fig:convergence_15800}
\end{subfigure}
%
\begin{subfigure}[b]{0.09\textwidth}
    \includegraphics[clip, trim=2.35cm 1.75cm 4.5cm 0cm,width=\textwidth]{img/convergence/15851.pdf}
    \caption{15851}
    \label{fig:convergence_15851}
\end{subfigure}
%
\begin{subfigure}[b]{0.09\textwidth}
    \includegraphics[clip, trim=2.35cm 1.75cm 4.5cm 0cm,width=\textwidth]{img/convergence/15863.pdf}
    \caption{15863}
    \label{fig:convergence_15863}
\end{subfigure}
%
\begin{subfigure}[b]{0.09\textwidth}
    \includegraphics[clip, trim=2.35cm 1.75cm 4.5cm 0cm,width=\textwidth]{img/convergence/15900.pdf}
    \caption{15900}
    \label{fig:convergence_15900}
\end{subfigure}
%
\begin{subfigure}[b]{0.09\textwidth}
    \includegraphics[clip, trim=2.35cm 1.75cm 4.5cm 0cm,width=\textwidth]{img/convergence/15928.pdf}
    \caption{15928}
    \label{fig:convergence_15928}
\end{subfigure}
%
\begin{subfigure}[b]{0.09\textwidth}
    \includegraphics[clip, trim=2.35cm 1.75cm 4.5cm 0cm,width=\textwidth]{img/convergence/15953.pdf}
    \caption{15953}
    \label{fig:convergence_15953}
\end{subfigure}
%
\begin{subfigure}[b]{0.09\textwidth}
    \includegraphics[clip, trim=2.35cm 1.75cm 4.5cm 0cm,width=\textwidth]{img/convergence/16000.pdf}
    \caption{16000}
    \label{fig:convergence_16000}
\end{subfigure}
%
\begin{subfigure}[b]{0.09\textwidth}
    \includegraphics[clip, trim=2.35cm 1.75cm 4.5cm 0cm,width=\textwidth]{img/convergence/16048.pdf}
    \caption{16048}
    \label{fig:convergence_16048}
\end{subfigure}
%
\begin{subfigure}[b]{0.09\textwidth}
    \includegraphics[clip, trim=2.35cm 1.75cm 4.5cm 0cm,width=\textwidth]{img/convergence/16078.pdf}
    \caption{16078}
    \label{fig:convergence_16078}
\end{subfigure}
%
\begin{subfigure}[b]{0.09\textwidth}
    \includegraphics[clip, trim=2.35cm 1.75cm 4.5cm 0cm,width=\textwidth]{img/convergence/16100.pdf}
    \caption{16100}
    \label{fig:convergence_16100}
\end{subfigure}
%
\begin{subfigure}[b]{0.09\textwidth}
    \includegraphics[clip, trim=2.35cm 1.75cm 4.5cm 0cm,width=\textwidth]{img/convergence/16200.pdf}
    \caption{16200}
    \label{fig:convergence_16200}
\end{subfigure}
%
\begin{subfigure}[b]{0.09\textwidth}
    \includegraphics[clip, trim=2.35cm 1.75cm 4.5cm 0cm,width=\textwidth]{img/convergence/16300.pdf}
    \caption{16300}
    \label{fig:convergence_16300}
\end{subfigure}
%
\begin{subfigure}[b]{0.09\textwidth}
    \includegraphics[clip, trim=2.35cm 1.75cm 4.5cm 0cm,width=\textwidth]{img/convergence/16389.pdf}
    \caption{16389}
    \label{fig:convergence_16389}
\end{subfigure}
%
\begin{subfigure}[b]{0.09\textwidth}
    \includegraphics[clip, trim=2.35cm 1.75cm 4.5cm 0cm,width=\textwidth]{img/convergence/16393.pdf}
    \caption{16393}
    \label{fig:convergence_16393}
\end{subfigure}
%
\begin{subfigure}[b]{0.09\textwidth}
    \includegraphics[clip, trim=2.35cm 1.75cm 4.5cm 0cm,width=\textwidth]{img/convergence/16400.pdf}
    \caption{16400}
    \label{fig:convergence_16400}
\end{subfigure}
%
\begin{subfigure}[b]{0.09\textwidth}
    \includegraphics[clip, trim=2.35cm 1.75cm 4.5cm 0cm,width=\textwidth]{img/convergence/16417.pdf}
    \caption{16417}
    \label{fig:convergence_16417}
\end{subfigure}
%
\begin{subfigure}[b]{0.09\textwidth}
    \includegraphics[clip, trim=2.35cm 1.75cm 4.5cm 0cm,width=\textwidth]{img/convergence/16424.pdf}
    \caption{16424}
    \label{fig:convergence_16424}
\end{subfigure}
%
\begin{subfigure}[b]{0.09\textwidth}
    \includegraphics[clip, trim=2.35cm 1.75cm 4.5cm 0cm,width=\textwidth]{img/convergence/16432.pdf}
    \caption{16432}
    \label{fig:convergence_16432}
\end{subfigure}
%
\begin{subfigure}[b]{0.09\textwidth}
    \includegraphics[clip, trim=2.35cm 1.75cm 4.5cm 0cm,width=\textwidth]{img/convergence/16433.pdf}
    \caption{16433}
    \label{fig:convergence_16433}
\end{subfigure}
%
\begin{subfigure}[b]{0.09\textwidth}
    \includegraphics[clip, trim=2.35cm 1.75cm 4.5cm 0cm,width=\textwidth]{img/convergence/16434.pdf}
    \caption{16434}
    \label{fig:convergence_16434}
\end{subfigure}
%
\begin{subfigure}[b]{0.09\textwidth}
    \includegraphics[clip, trim=2.35cm 1.75cm 4.5cm 0cm,width=\textwidth]{img/convergence/16435.pdf}
    \caption{16435}
    \label{fig:convergence_16435}
\end{subfigure}
%
\begin{subfigure}[b]{0.09\textwidth}
    \includegraphics[clip, trim=2.35cm 1.75cm 4.5cm 0cm,width=\textwidth]{img/convergence/16436.pdf}
    \caption{16436}
    \label{fig:convergence_16436}
\end{subfigure}
%
\begin{subfigure}[b]{0.09\textwidth}
    \includegraphics[clip, trim=2.35cm 1.75cm 4.5cm 0cm,width=\textwidth]{img/convergence/16437.pdf}
    \caption{16437}
    \label{fig:convergence_16437}
\end{subfigure}
%
\begin{subfigure}[b]{0.09\textwidth}
    \includegraphics[clip, trim=2.35cm 1.75cm 4.5cm 0cm,width=\textwidth]{img/convergence/16438.pdf}
    \caption{16438}
    \label{fig:convergence_16438}
\end{subfigure}
%
\begin{subfigure}[b]{0.09\textwidth}
    \includegraphics[clip, trim=2.35cm 1.75cm 4.5cm 0cm,width=\textwidth]{img/convergence/16439.pdf}
    \caption{16439}
    \label{fig:convergence_16439}
\end{subfigure}
%
\begin{subfigure}[b]{0.09\textwidth}
    \includegraphics[clip, trim=2.35cm 1.75cm 4.5cm 0cm,width=\textwidth]{img/convergence/16442.pdf}
    \caption{16442}
    \label{fig:convergence_16442}
\end{subfigure}
%
\begin{subfigure}[b]{0.09\textwidth}
    \includegraphics[clip, trim=2.35cm 1.75cm 4.5cm 0cm,width=\textwidth]{img/convergence/16444.pdf}
    \caption{16444}
    \label{fig:convergence_16444}
\end{subfigure}
%
\begin{subfigure}[b]{0.09\textwidth}
    \includegraphics[clip, trim=2.35cm 1.75cm 4.5cm 0cm,width=\textwidth]{img/convergence/16445.pdf}
    \caption{16445}
    \label{fig:convergence_16445}
\end{subfigure}
%
\begin{subfigure}[b]{0.09\textwidth}
    \includegraphics[clip, trim=2.35cm 1.75cm 4.5cm 0cm,width=\textwidth]{img/convergence/16450.pdf}
    \caption{16450}
    \label{fig:convergence_16450}
\end{subfigure}
%
\begin{subfigure}[b]{0.09\textwidth}
    \includegraphics[clip, trim=2.35cm 1.75cm 4.5cm 0cm,width=\textwidth]{img/convergence/16464.pdf}
    \caption{16464}
    \label{fig:convergence_16464}
\end{subfigure}
%
\begin{subfigure}[b]{0.09\textwidth}
    \includegraphics[clip, trim=2.35cm 1.75cm 4.5cm 0cm,width=\textwidth]{img/convergence/16465.pdf}
    \caption{16465}
    \label{fig:convergence_16465}
\end{subfigure}
%
\begin{subfigure}[b]{0.09\textwidth}
    \includegraphics[clip, trim=2.35cm 1.75cm 4.5cm 0cm,width=\textwidth]{img/convergence/16466.pdf}
    \caption{16466}
    \label{fig:convergence_16466}
\end{subfigure}
%
\begin{subfigure}[b]{0.09\textwidth}
    \includegraphics[clip, trim=2.35cm 1.75cm 4.5cm 0cm,width=\textwidth]{img/convergence/16467.pdf}
    \caption{16467}
    \label{fig:convergence_16467}
\end{subfigure}
%
\begin{subfigure}[b]{0.09\textwidth}
    \includegraphics[clip, trim=2.35cm 1.75cm 4.5cm 0cm,width=\textwidth]{img/convergence/16478.pdf}
    \caption{16478}
    \label{fig:convergence_16478}
\end{subfigure}
%
\begin{subfigure}[b]{0.09\textwidth}
    \includegraphics[clip, trim=2.35cm 1.75cm 4.5cm 0cm,width=\textwidth]{img/convergence/16481.pdf}
    \caption{16481}
    \label{fig:convergence_16481}
\end{subfigure}
%
\begin{subfigure}[b]{0.09\textwidth}
    \includegraphics[clip, trim=2.35cm 1.75cm 4.5cm 0cm,width=\textwidth]{img/convergence/16482.pdf}
    \caption{16482}
    \label{fig:convergence_16482}
\end{subfigure}
%
\begin{subfigure}[b]{0.09\textwidth}
    \includegraphics[clip, trim=2.35cm 1.75cm 4.5cm 0cm,width=\textwidth]{img/convergence/16483.pdf}
    \caption{16483}
    \label{fig:convergence_16483}
\end{subfigure}
%
\begin{subfigure}[b]{0.09\textwidth}
    \includegraphics[clip, trim=2.35cm 1.75cm 4.5cm 0cm,width=\textwidth]{img/convergence/16484.pdf}
    \caption{16484}
    \label{fig:convergence_16484}
\end{subfigure}
%
\begin{subfigure}[b]{0.09\textwidth}
    \includegraphics[clip, trim=2.35cm 1.75cm 4.5cm 0cm,width=\textwidth]{img/convergence/16488.pdf}
    \caption{16488}
    \label{fig:convergence_16488}
\end{subfigure}
%
\begin{subfigure}[b]{0.09\textwidth}
    \includegraphics[clip, trim=2.35cm 1.75cm 4.5cm 0cm,width=\textwidth]{img/convergence/16494.pdf}
    \caption{16494}
    \label{fig:convergence_16494}
\end{subfigure}
%
\begin{subfigure}[b]{0.09\textwidth}
    \includegraphics[clip, trim=2.35cm 1.75cm 4.5cm 0cm,width=\textwidth]{img/convergence/16495.pdf}
    \caption{16495}
    \label{fig:convergence_16495}
\end{subfigure}
%
\begin{subfigure}[b]{0.09\textwidth}
    \includegraphics[clip, trim=2.35cm 1.75cm 4.5cm 0cm,width=\textwidth]{img/convergence/16500.pdf}
    \caption{16500}
    \label{fig:convergence_16500}
\end{subfigure}
%
\begin{subfigure}[b]{0.09\textwidth}
    \includegraphics[clip, trim=2.35cm 1.75cm 4.5cm 0cm,width=\textwidth]{img/convergence/16600.pdf}
    \caption{16600}
    \label{fig:convergence_16600}
\end{subfigure}
%
\begin{subfigure}[b]{0.09\textwidth}
    \includegraphics[clip, trim=2.35cm 1.75cm 4.5cm 0cm,width=\textwidth]{img/convergence/16630.pdf}
    \caption{16630}
    \label{fig:convergence_16630}
\end{subfigure}
%
\begin{subfigure}[b]{0.09\textwidth}
    \includegraphics[clip, trim=2.35cm 1.75cm 4.5cm 0cm,width=\textwidth]{img/convergence/16632.pdf}
    \caption{16632}
    \label{fig:convergence_16632}
\end{subfigure}
%
\begin{subfigure}[b]{0.09\textwidth}
    \includegraphics[clip, trim=2.35cm 1.75cm 4.5cm 0cm,width=\textwidth]{img/convergence/16633.pdf}
    \caption{16633}
    \label{fig:convergence_16633}
\end{subfigure}
%
\begin{subfigure}[b]{0.09\textwidth}
    \includegraphics[clip, trim=2.35cm 1.75cm 4.5cm 0cm,width=\textwidth]{img/convergence/16634.pdf}
    \caption{16634}
    \label{fig:convergence_16634}
\end{subfigure}
%
\begin{subfigure}[b]{0.09\textwidth}
    \includegraphics[clip, trim=2.35cm 1.75cm 4.5cm 0cm,width=\textwidth]{img/convergence/16635.pdf}
    \caption{16635}
    \label{fig:convergence_16635}
\end{subfigure}
%
\begin{subfigure}[b]{0.09\textwidth}
    \includegraphics[clip, trim=2.35cm 1.75cm 4.5cm 0cm,width=\textwidth]{img/convergence/16636.pdf}
    \caption{16636}
    \label{fig:convergence_16636}
\end{subfigure}
%
\begin{subfigure}[b]{0.09\textwidth}
    \includegraphics[clip, trim=2.35cm 1.75cm 4.5cm 0cm,width=\textwidth]{img/convergence/16637.pdf}
    \caption{16637}
    \label{fig:convergence_16637}
\end{subfigure}
%
\begin{subfigure}[b]{0.09\textwidth}
    \includegraphics[clip, trim=2.35cm 1.75cm 4.5cm 0cm,width=\textwidth]{img/convergence/16638.pdf}
    \caption{16638}
    \label{fig:convergence_16638}
\end{subfigure}
%
\begin{subfigure}[b]{0.09\textwidth}
    \includegraphics[clip, trim=2.35cm 1.75cm 4.5cm 0cm,width=\textwidth]{img/convergence/16639.pdf}
    \caption{16639}
    \label{fig:convergence_16639}
\end{subfigure}
%
\begin{subfigure}[b]{0.09\textwidth}
    \includegraphics[clip, trim=2.35cm 1.75cm 4.5cm 0cm,width=\textwidth]{img/convergence/16640.pdf}
    \caption{16640}
    \label{fig:convergence_16640}
\end{subfigure}
%
\begin{subfigure}[b]{0.09\textwidth}
    \includegraphics[clip, trim=2.35cm 1.75cm 4.5cm 0cm,width=\textwidth]{img/convergence/16641.pdf}
    \caption{16641}
    \label{fig:convergence_16641}
\end{subfigure}
%
\begin{subfigure}[b]{0.09\textwidth}
    \includegraphics[clip, trim=2.35cm 1.75cm 4.5cm 0cm,width=\textwidth]{img/convergence/16643.pdf}
    \caption{16643}
    \label{fig:convergence_16643}
\end{subfigure}
%
\begin{subfigure}[b]{0.09\textwidth}
    \includegraphics[clip, trim=2.35cm 1.75cm 4.5cm 0cm,width=\textwidth]{img/convergence/16644.pdf}
    \caption{16644}
    \label{fig:convergence_16644}
\end{subfigure}
%
\begin{subfigure}[b]{0.09\textwidth}
    \includegraphics[clip, trim=2.35cm 1.75cm 4.5cm 0cm,width=\textwidth]{img/convergence/16645.pdf}
    \caption{16645}
    \label{fig:convergence_16645}
\end{subfigure}
%
\begin{subfigure}[b]{0.09\textwidth}
    \includegraphics[clip, trim=2.35cm 1.75cm 4.5cm 0cm,width=\textwidth]{img/convergence/16649.pdf}
    \caption{16649}
    \label{fig:convergence_16649}
\end{subfigure}
%
\begin{subfigure}[b]{0.09\textwidth}
    \includegraphics[clip, trim=2.35cm 1.75cm 4.5cm 0cm,width=\textwidth]{img/convergence/16654.pdf}
    \caption{16654}
    \label{fig:convergence_16654}
\end{subfigure}
%
\begin{subfigure}[b]{0.09\textwidth}
    \includegraphics[clip, trim=2.35cm 1.75cm 4.5cm 0cm,width=\textwidth]{img/convergence/16660.pdf}
    \caption{16660}
    \label{fig:convergence_16660}
\end{subfigure}
%
\begin{subfigure}[b]{0.09\textwidth}
    \includegraphics[clip, trim=2.35cm 1.75cm 4.5cm 0cm,width=\textwidth]{img/convergence/16664.pdf}
    \caption{16664}
    \label{fig:convergence_16664}
\end{subfigure}
%
\begin{subfigure}[b]{0.09\textwidth}
    \includegraphics[clip, trim=2.35cm 1.75cm 4.5cm 0cm,width=\textwidth]{img/convergence/16670.pdf}
    \caption{16670}
    \label{fig:convergence_16670}
\end{subfigure}
%
\begin{subfigure}[b]{0.09\textwidth}
    \includegraphics[clip, trim=2.35cm 1.75cm 4.5cm 0cm,width=\textwidth]{img/convergence/16674.pdf}
    \caption{16674}
    \label{fig:convergence_16674}
\end{subfigure}
%
\begin{subfigure}[b]{0.09\textwidth}
    \includegraphics[clip, trim=2.35cm 1.75cm 4.5cm 0cm,width=\textwidth]{img/convergence/16680.pdf}
    \caption{16680}
    \label{fig:convergence_16680}
\end{subfigure}
%
\begin{subfigure}[b]{0.09\textwidth}
    \includegraphics[clip, trim=2.35cm 1.75cm 4.5cm 0cm,width=\textwidth]{img/convergence/16684.pdf}
    \caption{16684}
    \label{fig:convergence_16684}
\end{subfigure}
%
\begin{subfigure}[b]{0.09\textwidth}
    \includegraphics[clip, trim=2.35cm 1.75cm 4.5cm 0cm,width=\textwidth]{img/convergence/16690.pdf}
    \caption{16690}
    \label{fig:convergence_16690}
\end{subfigure}
%
\begin{subfigure}[b]{0.09\textwidth}
    \includegraphics[clip, trim=2.35cm 1.75cm 4.5cm 0cm,width=\textwidth]{img/convergence/16694.pdf}
    \caption{16694}
    \label{fig:convergence_16694}
\end{subfigure}
%
\begin{subfigure}[b]{0.09\textwidth}
    \includegraphics[clip, trim=2.35cm 1.75cm 4.5cm 0cm,width=\textwidth]{img/convergence/16700.pdf}
    \caption{16700}
    \label{fig:convergence_16700}
\end{subfigure}
%
\begin{subfigure}[b]{0.09\textwidth}
    \includegraphics[clip, trim=2.35cm 1.75cm 4.5cm 0cm,width=\textwidth]{img/convergence/16705.pdf}
    \caption{16705}
    \label{fig:convergence_16705}
\end{subfigure}
%
\begin{subfigure}[b]{0.09\textwidth}
    \includegraphics[clip, trim=2.35cm 1.75cm 4.5cm 0cm,width=\textwidth]{img/convergence/16711.pdf}
    \caption{16711}
    \label{fig:convergence_16711}
\end{subfigure}
%
\begin{subfigure}[b]{0.09\textwidth}
    \includegraphics[clip, trim=2.35cm 1.75cm 4.5cm 0cm,width=\textwidth]{img/convergence/16714.pdf}
    \caption{16714}
    \label{fig:convergence_16714}
\end{subfigure}
%
\begin{subfigure}[b]{0.09\textwidth}
    \includegraphics[clip, trim=2.35cm 1.75cm 4.5cm 0cm,width=\textwidth]{img/convergence/16720.pdf}
    \caption{16720}
    \label{fig:convergence_16720}
\end{subfigure}
%
\begin{subfigure}[b]{0.09\textwidth}
    \includegraphics[clip, trim=2.35cm 1.75cm 4.5cm 0cm,width=\textwidth]{img/convergence/16723.pdf}
    \caption{16723}
    \label{fig:convergence_16723}
\end{subfigure}
%
\begin{subfigure}[b]{0.09\textwidth}
    \includegraphics[clip, trim=2.35cm 1.75cm 4.5cm 0cm,width=\textwidth]{img/convergence/16730.pdf}
    \caption{16730}
    \label{fig:convergence_16730}
\end{subfigure}
%
\begin{subfigure}[b]{0.09\textwidth}
    \includegraphics[clip, trim=2.35cm 1.75cm 4.5cm 0cm,width=\textwidth]{img/convergence/16733.pdf}
    \caption{16733}
    \label{fig:convergence_16733}
\end{subfigure}
%
\begin{subfigure}[b]{0.09\textwidth}
    \includegraphics[clip, trim=2.35cm 1.75cm 4.5cm 0cm,width=\textwidth]{img/convergence/16800.pdf}
    \caption{16800}
    \label{fig:convergence_16800}
\end{subfigure}
%
\begin{subfigure}[b]{0.09\textwidth}
    \includegraphics[clip, trim=2.35cm 1.75cm 4.5cm 0cm,width=\textwidth]{img/convergence/16900.pdf}
    \caption{16900}
    \label{fig:convergence_16900}
\end{subfigure}
%
\begin{subfigure}[b]{0.09\textwidth}
    \includegraphics[clip, trim=2.35cm 1.75cm 4.5cm 0cm,width=\textwidth]{img/convergence/16981.pdf}
    \caption{16981}
    \label{fig:convergence_16981}
\end{subfigure}
%
\begin{subfigure}[b]{0.09\textwidth}
    \includegraphics[clip, trim=2.35cm 1.75cm 4.5cm 0cm,width=\textwidth]{img/convergence/16983.pdf}
    \caption{16983}
    \label{fig:convergence_16983}
\end{subfigure}
%
\begin{subfigure}[b]{0.09\textwidth}
    \includegraphics[clip, trim=2.35cm 1.75cm 4.5cm 0cm,width=\textwidth]{img/convergence/16985.pdf}
    \caption{16985}
    \label{fig:convergence_16985}
\end{subfigure}
%
\begin{subfigure}[b]{0.09\textwidth}
    \includegraphics[clip, trim=2.35cm 1.75cm 4.5cm 0cm,width=\textwidth]{img/convergence/16987.pdf}
    \caption{16987}
    \label{fig:convergence_16987}
\end{subfigure}
%
\begin{subfigure}[b]{0.09\textwidth}
    \includegraphics[clip, trim=2.35cm 1.75cm 4.5cm 0cm,width=\textwidth]{img/convergence/16991.pdf}
    \caption{16991}
    \label{fig:convergence_16991}
\end{subfigure}
%
\begin{subfigure}[b]{0.09\textwidth}
    \includegraphics[clip, trim=2.35cm 1.75cm 4.5cm 0cm,width=\textwidth]{img/convergence/16993.pdf}
    \caption{16993}
    \label{fig:convergence_16993}
\end{subfigure}
%
\begin{subfigure}[b]{0.09\textwidth}
    \includegraphics[clip, trim=2.35cm 1.75cm 4.5cm 0cm,width=\textwidth]{img/convergence/16996.pdf}
    \caption{16996}
    \label{fig:convergence_16996}
\end{subfigure}
%
\begin{subfigure}[b]{0.09\textwidth}
    \includegraphics[clip, trim=2.35cm 1.75cm 4.5cm 0cm,width=\textwidth]{img/convergence/17000.pdf}
    \caption{17000}
    \label{fig:convergence_17000}
\end{subfigure}
%
\begin{subfigure}[b]{0.09\textwidth}
    \includegraphics[clip, trim=2.35cm 1.75cm 4.5cm 0cm,width=\textwidth]{img/convergence/17008.pdf}
    \caption{17008}
    \label{fig:convergence_17008}
\end{subfigure}
%
\begin{subfigure}[b]{0.09\textwidth}
    \includegraphics[clip, trim=2.35cm 1.75cm 4.5cm 0cm,width=\textwidth]{img/convergence/17016.pdf}
    \caption{17016}
    \label{fig:convergence_17016}
\end{subfigure}
%
\begin{subfigure}[b]{0.09\textwidth}
    \includegraphics[clip, trim=2.35cm 1.75cm 4.5cm 0cm,width=\textwidth]{img/convergence/17017.pdf}
    \caption{17017}
    \label{fig:convergence_17017}
\end{subfigure}
%
\begin{subfigure}[b]{0.09\textwidth}
    \includegraphics[clip, trim=2.35cm 1.75cm 4.5cm 0cm,width=\textwidth]{img/convergence/17018.pdf}
    \caption{17018}
    \label{fig:convergence_17018}
\end{subfigure}
%
\begin{subfigure}[b]{0.09\textwidth}
    \includegraphics[clip, trim=2.35cm 1.75cm 4.5cm 0cm,width=\textwidth]{img/convergence/17019.pdf}
    \caption{17019}
    \label{fig:convergence_17019}
\end{subfigure}
%
\begin{subfigure}[b]{0.09\textwidth}
    \includegraphics[clip, trim=2.35cm 1.75cm 4.5cm 0cm,width=\textwidth]{img/convergence/17020.pdf}
    \caption{17020}
    \label{fig:convergence_17020}
\end{subfigure}
%
\begin{subfigure}[b]{0.09\textwidth}
    \includegraphics[clip, trim=2.35cm 1.75cm 4.5cm 0cm,width=\textwidth]{img/convergence/17028.pdf}
    \caption{17028}
    \label{fig:convergence_17028}
\end{subfigure}
%
\begin{subfigure}[b]{0.09\textwidth}
    \includegraphics[clip, trim=2.35cm 1.75cm 4.5cm 0cm,width=\textwidth]{img/convergence/17029.pdf}
    \caption{17029}
    \label{fig:convergence_17029}
\end{subfigure}
%
\begin{subfigure}[b]{0.09\textwidth}
    \includegraphics[clip, trim=2.35cm 1.75cm 4.5cm 0cm,width=\textwidth]{img/convergence/17030.pdf}
    \caption{17030}
    \label{fig:convergence_17030}
\end{subfigure}
%
\begin{subfigure}[b]{0.09\textwidth}
    \includegraphics[clip, trim=2.35cm 1.75cm 4.5cm 0cm,width=\textwidth]{img/convergence/17033.pdf}
    \caption{17033}
    \label{fig:convergence_17033}
\end{subfigure}
%
\begin{subfigure}[b]{0.09\textwidth}
    \includegraphics[clip, trim=2.35cm 1.75cm 4.5cm 0cm,width=\textwidth]{img/convergence/17036.pdf}
    \caption{17036}
    \label{fig:convergence_17036}
\end{subfigure}
%
\begin{subfigure}[b]{0.09\textwidth}
    \includegraphics[clip, trim=2.35cm 1.75cm 4.5cm 0cm,width=\textwidth]{img/convergence/17043.pdf}
    \caption{17043}
    \label{fig:convergence_17043}
\end{subfigure}
%
\begin{subfigure}[b]{0.09\textwidth}
    \includegraphics[clip, trim=2.35cm 1.75cm 4.5cm 0cm,width=\textwidth]{img/convergence/17045.pdf}
    \caption{17045}
    \label{fig:convergence_17045}
\end{subfigure}
%
\begin{subfigure}[b]{0.09\textwidth}
    \includegraphics[clip, trim=2.35cm 1.75cm 4.5cm 0cm,width=\textwidth]{img/convergence/17054.pdf}
    \caption{17054}
    \label{fig:convergence_17054}
\end{subfigure}
%
\begin{subfigure}[b]{0.09\textwidth}
    \includegraphics[clip, trim=2.35cm 1.75cm 4.5cm 0cm,width=\textwidth]{img/convergence/17055.pdf}
    \caption{17055}
    \label{fig:convergence_17055}
\end{subfigure}
%
\begin{subfigure}[b]{0.09\textwidth}
    \includegraphics[clip, trim=2.35cm 1.75cm 4.5cm 0cm,width=\textwidth]{img/convergence/17061.pdf}
    \caption{17061}
    \label{fig:convergence_17061}
\end{subfigure}
%
\begin{subfigure}[b]{0.09\textwidth}
    \includegraphics[clip, trim=2.35cm 1.75cm 4.5cm 0cm,width=\textwidth]{img/convergence/17066.pdf}
    \caption{17066}
    \label{fig:convergence_17066}
\end{subfigure}
%
\begin{subfigure}[b]{0.09\textwidth}
    \includegraphics[clip, trim=2.35cm 1.75cm 4.5cm 0cm,width=\textwidth]{img/convergence/17070.pdf}
    \caption{17070}
    \label{fig:convergence_17070}
\end{subfigure}
%
\begin{subfigure}[b]{0.09\textwidth}
    \includegraphics[clip, trim=2.35cm 1.75cm 4.5cm 0cm,width=\textwidth]{img/convergence/17073.pdf}
    \caption{17073}
    \label{fig:convergence_17073}
\end{subfigure}
%
\begin{subfigure}[b]{0.09\textwidth}
    \includegraphics[clip, trim=2.35cm 1.75cm 4.5cm 0cm,width=\textwidth]{img/convergence/17075.pdf}
    \caption{17075}
    \label{fig:convergence_17075}
\end{subfigure}
%
\begin{subfigure}[b]{0.09\textwidth}
    \includegraphics[clip, trim=2.35cm 1.75cm 4.5cm 0cm,width=\textwidth]{img/convergence/17076.pdf}
    \caption{17076}
    \label{fig:convergence_17076}
\end{subfigure}
%
\begin{subfigure}[b]{0.09\textwidth}
    \includegraphics[clip, trim=2.35cm 1.75cm 4.5cm 0cm,width=\textwidth]{img/convergence/17078.pdf}
    \caption{17078}
    \label{fig:convergence_17078}
\end{subfigure}
%
\begin{subfigure}[b]{0.09\textwidth}
    \includegraphics[clip, trim=2.35cm 1.75cm 4.5cm 0cm,width=\textwidth]{img/convergence/17079.pdf}
    \caption{17079}
    \label{fig:convergence_17079}
\end{subfigure}
%
\begin{subfigure}[b]{0.09\textwidth}
    \includegraphics[clip, trim=2.35cm 1.75cm 4.5cm 0cm,width=\textwidth]{img/convergence/17085.pdf}
    \caption{17085}
    \label{fig:convergence_17085}
\end{subfigure}
%
\begin{subfigure}[b]{0.09\textwidth}
    \includegraphics[clip, trim=2.35cm 1.75cm 4.5cm 0cm,width=\textwidth]{img/convergence/17086.pdf}
    \caption{17086}
    \label{fig:convergence_17086}
\end{subfigure}
%
\begin{subfigure}[b]{0.09\textwidth}
    \includegraphics[clip, trim=2.35cm 1.75cm 4.5cm 0cm,width=\textwidth]{img/convergence/17087.pdf}
    \caption{17087}
    \label{fig:convergence_17087}
\end{subfigure}
%
\begin{subfigure}[b]{0.09\textwidth}
    \includegraphics[clip, trim=2.35cm 1.75cm 4.5cm 0cm,width=\textwidth]{img/convergence/17088.pdf}
    \caption{17088}
    \label{fig:convergence_17088}
\end{subfigure}
%
\begin{subfigure}[b]{0.09\textwidth}
    \includegraphics[clip, trim=2.35cm 1.75cm 4.5cm 0cm,width=\textwidth]{img/convergence/17094.pdf}
    \caption{17094}
    \label{fig:convergence_17094}
\end{subfigure}
%
\begin{subfigure}[b]{0.09\textwidth}
    \includegraphics[clip, trim=2.35cm 1.75cm 4.5cm 0cm,width=\textwidth]{img/convergence/17096.pdf}
    \caption{17096}
    \label{fig:convergence_17096}
\end{subfigure}
%
\begin{subfigure}[b]{0.09\textwidth}
    \includegraphics[clip, trim=2.35cm 1.75cm 4.5cm 0cm,width=\textwidth]{img/convergence/17097.pdf}
    \caption{17097}
    \label{fig:convergence_17097}
\end{subfigure}
%
    
      \caption{Evolution of training throughout epochs (cross-entropy, constraint loss, and accuracy). Left-hand axis show the accuracy metric (blue line) against the cross-entropy, and constraint loss in the right axis (orange line), for each epoch of the training phase in the horizontal axis.}
	\label{fig:peaks}
\end{figure*}


\subsection{Zero initialization}\label{subsec:zero}

Throughout the experimentation we set the parameters to zero in an attempt to simplify initialization (See Figure \ref{fig:zeros}). We leave the responsibility of pulling the parameters from zero to the constraint, taking advantage of the fact of that the constraint gradient depend on the input of the units. However, we have two problems: (1) all the units in the layer will share the same gradient and will became the same and, (2) During the first iterations $\xi^+ = \xi^-$, so the gradient from the positive constraint will cancel with the negative (recall Equation \ref{eq:definitionOfRho}).
\\\\
In order to \emph{break symmetry} of the layers, we use Dropout \cite{dropout}. We find that Dropout has a very strong effect (i.e. \emph{harmful} towards convergence of the learning process). In order to solve this, we introduced annealed Dropout \cite{dropoutAnnealing}, starting with a rate of $0.5$ for 500 epochs using linear decay. We would like to remark that zero initialization  contributes to the aggressiveness of the transient states during training, causing  several peaks in the constraint loss. We conjecture that this process depends on the position of the hyperplanes after dropout annealing. 

\begin{figure*}
  \centering
    \begin{subfigure}[b]{0.5\textwidth}
        \includegraphics[width=\textwidth]{img/zero/3000/09-conv2d_1-0.pdf}
        \caption{Input layer}
        \label{fig:zerosInput3000}
    \end{subfigure}~ %add desired spacing between images, e. g. ~, \quad, \qquad, \hfill etc. 
      %(or a blank line to force the subfigure onto a new line)
    \begin{subfigure}[b]{0.5\textwidth}
        \includegraphics[width=\textwidth]{img/zero/3000/57-09-output.pdf}
        \caption{Output layer}
        \label{fig:zerosOutput3000}
    \end{subfigure}
    
      
  \caption{Zero initialization plus \SepUnitPoint and annealed dropout} 
  \label{fig:zeros} 
\end{figure*}

\begin{table}[h!]
\begin{center}
\begin{tabular}{l|rr|rr}
\toprule
{}  & \multicolumn{2}{c}{Accuracy} & \multicolumn{2}{c}{Loss} \\
{}  & Train   & Val.  & Train  & Val.  \\
\midrule
\ReLU            &  0.5176 &      0.4 &  0.6925 &  0.6938 \\
\ReLUBN     &  0.8117 &      0.6 &  0.6331 &  0.6636 \\
\SepLayer &  1.0000 &      1.0 &  0.0000 &  0.0211 \\
\SepPoint    &  0.9294 &  0.8000 &  0.1765 &  0.6476 \\
\SepUnit    &  0.9058 &  0.8000 &  0.4161 &  1.5228 \\
\SepUnitPoint   &  0.9882 &  0.9333 &  0.6988 &  1.0810 \\
\bottomrule
\end{tabular}
\end{center}
\caption{Maximal performance experiment using the \moons dataset. From left to right, accuracy and loss (for \emph{train} and \emph{validation} sets) for \ReLU, \ReLUBN, and  \SepConstraint in all its variants.}
  \label{tab:moons}
\end{table}


\subsection{Width versus Depth relation}


\begin{figure*}
  \centering
    \begin{subfigure}[b]{0.3\textwidth}
        \includegraphics[width=\textwidth]{img/moons_grid/acc-relu.pdf}
        \caption{\ReLU}
        \label{fig:moons_grid_relu}
    \end{subfigure}
    ~ %add desired spacing between images, e. g. ~, \quad, \qquad, \hfill etc. 
      %(or a blank line to force the subfigure onto a new line)
    \centering
    \begin{subfigure}[b]{0.3\textwidth}
        \includegraphics[width=\textwidth]{img/moons_grid/acc-relu-bn.pdf}
        \caption{\ReLUBN}
        \label{fig:moons_grid_relubn}
    \end{subfigure}
    ~ %add desired spacing between images, e. g. ~, \quad, \qquad, \hfill etc. 
      %(or a blank line to force the subfigure onto a new line)
    \centering
    \begin{subfigure}[b]{0.3\textwidth}
        \includegraphics[width=\textwidth]{img/moons_grid/acc-sep-up-0-0001.pdf}
        \caption{\SepUnitPoint}
        \label{fig:moons_grid_up}
    \end{subfigure}
    ~ %add desired spacing between images, e. g. ~, \quad, \qquad, \hfill etc. 
      %(or a blank line to force the subfigure onto a new line)
    \\
    \begin{subfigure}[b]{0.3\textwidth}
        \includegraphics[width=\textwidth]{img/moons_grid/acc-sep-u-0-0001.pdf}
        \caption{\SepUnit}
        \label{fig:moons_grid_u}
    \end{subfigure}
    ~ %add desired spacing between images, e. g. ~, \quad, \qquad, \hfill etc. 
      %(or a blank line to force the subfigure onto a new line)
    \centering
    \begin{subfigure}[b]{0.3\textwidth}
        \includegraphics[width=\textwidth]{img/moons_grid/acc-sep-p-0-0001.pdf}
        \caption{\SepPoint}
        \label{fig:moons_grid_p}
    \end{subfigure}
    ~ %add desired spacing between images, e. g. ~, \quad, \qquad, \hfill etc. 
      %(or a blank line to force the subfigure onto a new line)
    \centering
    \begin{subfigure}[b]{0.3\textwidth}
        \includegraphics[width=\textwidth]{img/moons_grid/acc-sep-l-0-0001.pdf}
        \caption{\SepLayer}
        \label{fig:moons_grid_l}
    \end{subfigure}
    ~ %add desired spacing between images, e. g. ~, \quad, \qquad, \hfill etc. 
      %(or a blank line to force the subfigure onto a new line)
    
  \caption{Depth vs Width accuracy plot a for rectangular network using a grid (width from $2$ to $25$ and depth from $2$ to $150$),  trained using a Adam learning rate of $0.01$ in the \moons dataset. The color show the accuracy attained of each of the combinations of width and depth, the clearer the better. Notice how \ReLU, Figure \ref{fig:moons_grid_relu}, fails with networks deeper than 30 layers. In other hand, \ReLUBN, Figure \ref{fig:moons_grid_relubn}, manages to work until 70 layers deep. \SepUnitPoint,  Figure \ref{fig:moons_grid_up}, works significantly better than both, up to 120 layers. Notice how all the methods suffer from degradation from depth, which is partially alleviated by the use of greater width. This is consistent with \cite{simpnet} and \cite{densenet}. However, \SepUnitPoint is able to delay the apparition of the issue. This is especially visible when the number of units is small (from $2$ to $5$) where \ReLUBN fails to work whereas \SepUnitPoint does not. Regarding to the role of the constraint on its success, we find that \SepUnit, Figure \ref{fig:moons_grid_u}, allows the network to grow deeper, yet the accuracy can be lower following the linear decrease with the inverse of the width, which we blame on the inability of the \SepUnit constraint to address the \emph{dead point} issue. In the other hand, \SepPoint, Figure \ref{fig:moons_grid_p}, seems to perform well up to 50 layers, but it breaks down afterwards. Finally, \SepLayer , Figure \ref{fig:moons_grid_l} seems to suffer if the width is too large, performing well up to 70 layers.}
  \label{fig:moons_grid} 
\end{figure*}


\begin{figure*}
  \centering
  \begin{subfigure}[b]{0.2\textwidth}
        \includegraphics[width=\textwidth]{img/moons_grid/acc-sep-up-0-0001.pdf}
        \caption{Glorot initialization}
        \label{fig:moons_glorot_rho}
    \end{subfigure}
    ~ %add desired spacing between images, e. g. ~, \quad, \qquad, \hfill etc. 
      %(or a blank line to force the subfigure onto a new line)
      \centering
    \begin{subfigure}[b]{0.2\textwidth}
        \includegraphics[width=\textwidth]{img/moons_grid/acc-sep-up-0-0001-zero.pdf}
        \caption{Zero initialization with $\rho = 0.51$}
        \label{fig:moons_zeros_rho}
    \end{subfigure}
    ~ %add desired spacing between images, e. g. ~, \quad, \qquad, \hfill etc. 
      %(or a blank line to force the subfigure onto a new line)
    \centering
    \begin{subfigure}[b]{0.2\textwidth}
        \includegraphics[width=\textwidth]{img/moons_grid/acc-sep-up-0-0001-nm-0.pdf}
        \caption{Glorot initialization with negative margin to zero}
        \label{fig:moons_glorot_nm0}
    \end{subfigure}
    ~ %add desired spacing between images, e. g. ~, \quad, \qquad, \hfill etc. 
      %(or a blank line to force the subfigure onto a new line)
    \centering
    \begin{subfigure}[b]{0.2\textwidth}
        \includegraphics[width=\textwidth]{img/moons_grid/acc-sep-up-0-0001-nm-0.pdf}
        \caption{Zero initialization with negative margin to zero}
        \label{fig:moons_zeros_nm0}
    \end{subfigure}
    
  \caption{Glorot vs Zero initialization in a series of Depth vs Width accuracy plot a for rectangular network using a grid (width from $2$ to $25$ and depth from $2$ to $150$),  trained using a Adam learning rate of $0.01$, annealed dropout for $1000$ epoch with an initial rate of $0.01$ in the \moons dataset. The color show the accuracy attained of each of the combinations of width and depth, the clearer the better. We show two different approaches to enable zero initialization by the use of \SepUnitPoint constraint. In the first two figures, we use $\rho = 0.51$ in order to bias the constraints towards the positive side (See Equation \ref{eq:definitionOfRho}) and compare how this enables to break the symmetry of the positive and negative constraints thus enabling training with zero initialization, showing how performs equally well. The second two figures show the same comparison but using the negative margin to zero, thus removing the symmetry naturally. We see how it performs equally well.
  }
  \label{fig:moons_grid_zero} 
\end{figure*}
