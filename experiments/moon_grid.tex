\begin{figure*}
  \centering
    \begin{subfigure}[b]{0.3\textwidth}
        \includegraphics[width=\textwidth]{img/moons_grid/acc-relu.pdf}
        \caption{\ReLU training}
        \label{fig:moons_grid_relu}
    \end{subfigure}
    ~ %add desired spacing between images, e. g. ~, \quad, \qquad, \hfill etc. 
      %(or a blank line to force the subfigure onto a new line)
    \centering
    \begin{subfigure}[b]{0.3\textwidth}
        \includegraphics[width=\textwidth]{img/moons_grid/acc-relu-bn.pdf}
        \caption{\ReLUBN training}
        \label{fig:moons_grid_relubn}
    \end{subfigure}
    ~ %add desired spacing between images, e. g. ~, \quad, \qquad, \hfill etc. 
      %(or a blank line to force the subfigure onto a new line)
    \centering
    \begin{subfigure}[b]{0.3\textwidth}
        \includegraphics[width=\textwidth]{img/moons_grid/acc-sep-up-0-0001.pdf}
        \caption{\SepUnitPoint training}
        \label{fig:moons_grid_up}
    \end{subfigure}
    ~ %add desired spacing between images, e. g. ~, \quad, \qquad, \hfill etc. 
      %(or a blank line to force the subfigure onto a new line)
    \\
    \begin{subfigure}[b]{0.3\textwidth}
        \includegraphics[width=\textwidth]{img/moons_grid/val-acc-relu.pdf}
        \caption{\ReLU validation}
        \label{fig:moons_grid_relu}
    \end{subfigure}
    ~ %add desired spacing between images, e. g. ~, \quad, \qquad, \hfill etc. 
      %(or a blank line to force the subfigure onto a new line)
    \centering
    \begin{subfigure}[b]{0.3\textwidth}
        \includegraphics[width=\textwidth]{img/moons_grid/val-acc-relu-bn.pdf}
        \caption{\ReLUBN validation}
        \label{fig:moons_grid_relubn}
    \end{subfigure}
    ~ %add desired spacing between images, e. g. ~, \quad, \qquad, \hfill etc. 
      %(or a blank line to force the subfigure onto a new line)
    \centering
    \begin{subfigure}[b]{0.3\textwidth}
        \includegraphics[width=\textwidth]{img/moons_grid/val-acc-sep-up-0-0001.pdf}
        \caption{\SepUnitPoint validation}
        \label{fig:moons_grid_up}
    \end{subfigure}
    ~ %add desired spacing between images, e. g. ~, \quad, \qquad, \hfill etc. 
      %(or a blank line to force the subfigure onto a new line)
      \\
    \begin{subfigure}[b]{0.3\textwidth}
        \includegraphics[width=\textwidth]{img/moons_grid/acc-sep-u-0-0001.pdf}
        \caption{\SepUnit training}
        \label{fig:moons_grid_u}
    \end{subfigure}
    ~ %add desired spacing between images, e. g. ~, \quad, \qquad, \hfill etc. 
      %(or a blank line to force the subfigure onto a new line)
    \centering
    \begin{subfigure}[b]{0.3\textwidth}
        \includegraphics[width=\textwidth]{img/moons_grid/acc-sep-p-0-0001.pdf}
        \caption{\SepPoint training}
        \label{fig:moons_grid_p}
    \end{subfigure}
    ~ %add desired spacing between images, e. g. ~, \quad, \qquad, \hfill etc. 
      %(or a blank line to force the subfigure onto a new line)
    \\
    \begin{subfigure}[b]{0.3\textwidth}
        \includegraphics[width=\textwidth]{img/moons_grid/val-acc-sep-u-0-0001.pdf}
        \caption{\SepUnit validation}
        \label{fig:moons_grid_u}
    \end{subfigure}
    ~ %add desired spacing between images, e. g. ~, \quad, \qquad, \hfill etc. 
      %(or a blank line to force the subfigure onto a new line)
    \centering
    \begin{subfigure}[b]{0.3\textwidth}
        \includegraphics[width=\textwidth]{img/moons_grid/val-acc-sep-p-0-0001.pdf}
        \caption{\SepPoint validation}
        \label{fig:moons_grid_p}
    \end{subfigure}
    ~ %add desired spacing between images, e. g. ~, \quad, \qquad, \hfill etc. 
      %(or a blank line to force the subfigure onto a new line)
    
    
  \caption{Depth vs Width accuracy plot a for rectangular network using a grid (width from $2$ to $25$ and depth from $2$ to $150$),  trained using a Adam learning rate of $0.01$ in the \moons dataset. The color show the accuracy attained of each of the combinations of width and depth, the clearer the better. Notice how \ReLU, Figure \ref{fig:moons_grid_relu}, fails with networks deeper than 30 layers. In other hand, \ReLUBN, Figure \ref{fig:moons_grid_relubn}, manages to work until 70 layers deep. \SepUnitPoint,  Figure \ref{fig:moons_grid_up}, works significantly better than both, up to 120 layers. Notice how all the methods suffer from degradation from depth, which is partially alleviated by the use of greater width. This is consistent with \cite{simpnet} and \cite{densenet}. However, \SepUnitPoint is able to delay the apparition of the issue. This is especially visible when the number of units is small (from $2$ to $5$) where \ReLUBN fails to work whereas \SepUnitPoint does not. Regarding to the role of the constraint on its success, we find that \SepUnit, Figure \ref{fig:moons_grid_u}, allows the network to grow deeper, yet the accuracy can be lower following the linear decrease with the inverse of the width, which we blame on the inability of the \SepUnit constraint to address the \emph{dead point} issue. In the other hand, \SepPoint, Figure \ref{fig:moons_grid_p}, seems to perform well up to 50 layers, but it breaks down afterwards. Finally, \SepLayer , Figure \ref{fig:moons_grid_l} seems to suffer if the width is too large, performing well up to 70 layers.}
  \label{fig:moons_grid} 
\end{figure*}
