
\section{Separating constraints}\label{sec:constraint}
If we regard a DNN as a function in the form of equation \ref{eq:network}. such network depends on a collection $\theta$ of parameters organized as follows. 
\\\\
For a layer $\layer_k:\Real{n}\rightarrow\Real{m}$, we denote its collection of parameters $\theta_k$ of the form
\begin{equation}
    \theta_k = \{(\vec{w}^k_j,b^k_j)|j=1,\ldots,m\}
\end{equation}
thus, $\theta$ is composed of the union of the parameters of each layer:
\begin{equation}
    \theta = \displaystyle\bigcup_{k=1}^D\theta_k
\end{equation}
thus, given $F$ as a function depends on both its input $\vec{x}$. The parameters in collection $\theta$ are set during the training process via the minimization of some loss functional. That is we seek the solution of the following equation
\begin{equation}\label{eq:generalOptimizationProblem}
\argmin_{\theta} \mathcal{L}(\mathcal{T},\theta)
\end{equation}
for some loss functional that depends on the training set $T$ and $\theta$ (see for example \cite{LeCun06atutorial} for available choices in classification problems). 
\\\\
Conceptually, we wish to enforce \ReLU \emph{separability} based on the definition provided in equation \ref{eq:separabilityDefinition}. However, set theoretic functions are not differentiable \cite{Glorot10Initialization,lecun2015DeepLearningBig,munkres2000Topology}. 
\\\\
To bridge that gap, we will make an adaptation inspired on the technique of Support Vector Machines as presented for example in  \cite{Burges1998TutorialOnSVMForPatternRecognition} or \cite{Hearst1998SupportVectorMachines}. 
\\\\\
That is, given a fixed choice of weight vector  $\vec{w}\in\Real{n}$ and bias $b\in\mathbb{R}$, we define slack variables $\xi^{+},\xi^{-}\geq 0$ for a dataset $\mathcal{T}$ (via set $X$) as a pair of numbers such that
\begin{equation}\label{eq:basicConstraintFormulation}
\begin{array}{lcl}
    \max\{\vec{w}\cdot\vec{x}+b|\vec{x}\in X\}\geq 1-\xi^{+}\\
    \min\{\vec{w}\cdot\vec{x}+b|\vec{x}\in X\}\leq -1+\xi^{-}\\
\end{array}
\end{equation}
This slack variables can be added to the optimization problem \ref{eq:generalOptimizationProblem} as a multiobjective optimization problem using a technique known as scalarization \cite{boyd} (equation \ref{eq:marginOptimizationProblem}) as
\begin{equation}\label{eq:marginOptimizationProblem}
\argmin_{\theta,\xi^{+},\xi^{-}} \mathcal{L}(\mathcal{T},\theta) + \lambda g(\xi^{+},\xi^{-})
\end{equation}
where $\lambda$ is a \emph{hyper-parameter} that introduces a trade-off between the constraint fulfillment and the main task loss. Intuitively, we mean to \emph{balance} the effect of the constraints and solving the actual problem.
Also, we will require that function $g$ is non-negative, so that in the limit (i.e. when minimized), the original loss will keep the training process going. 
\\\\
The intuition behind the introduction of these constraints presented in Equation \ref{eq:basicConstraintFormulation} is as follows. 
\\\\
We introduce the constraints in pair according to the sign. $\xi^{+}$ intends to create at least one pre-activation value greater (or equal) than 1, in order to force the weights and biases to be chosen so that the lower part of the units is non-empty. 
\\\\
In symmetry, enforcing the constraint with $\xi^{-}$ promotes at least pre-activation of points of $X$ below -1. This intends to ensure that the upper parts of the units are non-empty. Notice how the inclusion of the \emph{max} is what carries this effect of \emph{at least one}, and what it makes different from SVM.
\\\\
Thus, since neither the upper, nor the lower part of the unit is empty (i.e. the units are neither dead nor affine in the sense of equations \ref{eq:deadUnitVersion1} and equation \ref{eq:affineUnit} respectively).
\\\\
Additionally, since introducing those constraints imply the the value of the weights now depend on it output instead of only on the loss functional, this enables to use \emph{zero-initialization}. However, notice that since values of $\xi^+$ and $\xi^-$ will be equal and cancel each other, we need to introduce a mechanism to break the tie. In our case we chose a convex combination of both that we name $g$. We use a negligible value (0.51 in practice) so it does not bias the problem.

\begin{equation}\label{eq:definitionOfRho}
    g(\xi^{+},\xi^{-}) = \rho\xi^{+}+(1-\rho)\xi^{-}
\end{equation}

What remains of this section is to choose the object of the constraints among the different structures of a network: by unit, by layer, and \emph{by point}.

\subsection{Unit based separation constraints \SepUnit}\label{subsec:sepUnit}
Within a \ReLU DNN $F$, choose a unit $u_j^k$ in layer $\layer_k:\Real{n_k}\rightarrow\Real{m_k}$. This unit will be defined univocally by its parameters:
\begin{equation}\label{eq:unitSepParameterWriting}
    u^k_j(\vec{x}) = u(\vec{x};\vec{w}^k_j,b^k_j)
\end{equation}
in order to enforce separation within this unit, we must substitute parameters in equation \ref{eq:basicConstraintFormulation} and create constraints of the form:
\begin{equation}
    \begin{array}{lcl}
    \max\{\vec{w}^k_j\cdot\vec{x}+b^k_j|\vec{x}\in X\}\geq 1-\xi^{+}_{jk}\\
    \min\{\vec{w}^k_j\cdot\vec{x}+b^k_j|\vec{x}\in X\}\leq -1+\xi^{-}_{jk}\\
\end{array}
\end{equation}
for $k=1,\ldots,D$ and $j$ the number of units per layer. As a consequence, we will need to introduce balancing parameters $rho_{jk}$ on function $g$. Individually, we will need to minimize functions of the form
\begin{equation}
    g(\xi^{+},\xi^{-}) = \rho_{jk}\xi^{+}_{jk}+(1-\rho_{jk})\xi^{-}_{jk}
\end{equation}
so that at a global scale we will have to add to the loss functional a term of the form:
\begin{equation}\label{eq:constraintLossForUnitSeparation}
    G = \sum_{k=1}^{D}\sum_{j=1}^{m_k}\rho_{jk}\xi^{+}_{jk}+(1-\rho_{jk})\xi^{-}_{jk}
\end{equation}
We name this the \emph{separation by unit} formulation \SepUnit. Notice that this constraint does not prevent from dead points, this is points that belong to the intersection of the lower sets of the units of a layer. That is, if $u_1,u_2,\ldots,u_m$ are the units of a layer $\layer:\Real{n}\rightarrow\Real{m}$, we say that $\vec{x}\in\Real{n}$ is \emph{dead} with regards to $\layer$  if

\begin{equation}\label{eq:deadPoint}
 \vec{x}\in \displaystyle\cap_{j=1})^n lower(u_j)
\end{equation}

\subsection{Point Based Separation \SepPoint}\label{subsec:sepPoint}

In order to avoid the presence of dead points (recall Equation \ref{eq:deadPoint}, we introduce the \emph{Point Based Separation Constraint} \SepPoint. Moreover, since in practice the entire dataset is not used for training, but \emph{batches} of it \cite{LeCun06atutorial}, a small batch size may cause inconsistencies within the training process. Under our formulation,  these issues would translate to unit penalization for not fulfilling \SepUnit on a \emph{per-batch} basis (due to the fact that the points that comply with the constraints may lie outside the batch).  We need to make our constraint formulation independent of batch choice by focusing on the values that constraints take on specific points. 
\\\\
That is, we will constraint the slacks $\xi^{\pm}$ to each point on the batch. That is given $\vec{x}\in X$, and $u_1,\ldots,u_m$ unit functions in a layer, we define slack variables $\xi^{-}_{xk},\xi^{+}_{xk}\geq 0$ in the context of the following constraints:
\begin{equation}\label{eq:pointSeparationConstraint}
\begin{array}{lcl}
    \displaystyle\max_{j=1,\ldots,m}\{\vec{w}^j_k\cdot\vec{x}+b^j_k\}\geq 1-\xi^{+}_{xk}\\\\
    \displaystyle\min_{j=1,\ldots,m_k}\{\vec{w}^j_k\cdot\vec{x}+b^j_k\}\leq -1+\xi^{-}_{xk}\\
\end{array}    
\end{equation}
so that the associated term to minimize will have the following form \ref{eq:convexCombinationOfConstraints}:
\begin{equation}\label{eq:convexCombinationOfConstraints}
    G = \sum_{k=1}^{D}\rho_{k}\xi^{+}_{xk}+(1-\rho_{k})\xi^{-}_{xk}
\end{equation}

Notice that \SepPoint does not prevent  dead units (as \SepUnit). We can combine both into \SepUnitPoint, to ensure that neither dead points nor dead units are present. 

\subsection{Layer Based Separation \SepLayer}\label{subsec:sepLayer}

As previous network architectures like \texttt{ResNet} and \texttt{DenseNet} have shown us, it is desirable sometimes to \emph{skip} connections in between layers, for \emph{accuracy} can improve dependent on depth \cite{resnet, densenet}. However, this growth in accuracy is not unbounded and past some point it degrades \cite{simpnet}. 
\\\\
Thus, it is important to provide a mechanism to forward the output of intermediate layers to the top of the network to which our constraint formulation can adapt and coalesce with existing techniques. Additionally, the number of constraints that we need to place, especially in \SepUnitPoint, is large and we would like to find a way to reduce the computational complexity.
\\\\
We suggest to relax the \SepUnit formulation for that aim and provide the reader with the \SepLayer formulation. The intuition here is to require that at least \emph{one} of the pre-activations of the entire batch is greater than 1 and another is less than -1, per layer.  
\\\\
With this change we allow the presence of affine units, but ensure that there is at least one pre-activation greater (or lesser) than 1 (-1). That is, given a layer $\layer_k$ we define the \emph{layer} margins $\xi^{-}_k$ and $\xi^{+}_k$, and rewrite equation \ref{eq:basicConstraintFormulation} as follows:
\begin{equation}\label{eq:layerSeparationConstraint}
\begin{array}{lcl}
    \displaystyle\max_{\vec{x}\in{X}}\max_{j=1,\ldots,m_k}\{\vec{w}^j_k\cdot\vec{x}+b^j_k\}\geq 1-\xi^{+}_k\\\\
    \displaystyle\min_{\vec{x}\in{X}}\min_{j=1,\ldots,m_k}\{\vec{w}^j_k\cdot\vec{x}+b^j_k\}\leq -1+\xi^{-}_k\\
\end{array}    
\end{equation}
We will need to introduce only $D$ constraints (one for each layer). In addition, the term to minimize alongside with the loss functional is as follows:
\begin{equation}\label{eq:constraintLossForLayerSeparation}
    G = \sum_{k=1}^{D}\lambda_k(\rho_{k}\xi^{+}_{k}+(1-\rho_{k})\xi^{-}_{k})
\end{equation}
we call this the \SepLayer formulation. Notice that we have introduced additional hyperparameters $\lambda_k$ to focus on the role specific layers.   


